%%%%%%%%%%%%%%%%%%%%%%%%%%%%%%%%%%%%%%%%%%%%%%%%%%%%%%%%%%%%%%%%%%%%%%%%%%%%%%%%
% Preámbulo                                                                    %
%%%%%%%%%%%%%%%%%%%%%%%%%%%%%%%%%%%%%%%%%%%%%%%%%%%%%%%%%%%%%%%%%%%%%%%%%%%%%%%%

\documentclass[11pt,a4paper,titlepage,twoside,openright,openbib]{report}
\usepackage{csquotes}
%\usepackage[
%bookmarksopen,
%bookmarksdepth=2,
%breaklinks=true
%]{hyperref}
%\PassOptionsToPackage{hyphens}{url}\usepackage{hyperref}
\usepackage[hyphens]{url}
\usepackage{hyperref}
\usepackage[section]{placeins}
\usepackage{lscape}
\makeatletter
\newcommand{\setword}[2]{%
  \phantomsection
  #1\def\@currentlabel{\unexpanded{#1}}\label{#2}%
}
\makeatother
%%% RELACIÓN DE VARIABLES A PERSONALIZAR %%%
%\def\lingua{gal}
\def\lingua{esp} % descomenta esta liña se redactarás a memoria en español
%\def\lingua{eng} % descomenta esta liña se redactarás a memoria en inglés
\def\nome{Amaro Castro Faci}                             % substitúe aquí o teu nome
\def\nomedirectorA{Outro Nome Completo}             % substitúe aquí o nome de quen dirixe
\def\titulo{PESCI: Plataforma de Entrega de Servicios Cloud para Investigación} % substitúe aquí o título do teu TFG  % descomenta a mención correspondente
%\def\mencion{COMPUTACIÓN}
%\def\mencion{ENXEÑARÍA DO SOFTWARE}
%\def\mencion{ENXEÑARÍA DE COMPUTADORES}
%\def\mencion{SISTEMAS DE INFORMACIÓN}
\def\mencion{TECNOLOXÍAS DA INFORMACIÓN}

%\def\renomearcadros{si} % descomenta esta liña se redactas a memoria en español e prefires que
                         % os "cuadros" e o "índice de cuadros" se renomeen
                         % a "tablas" e "índice de tablas" respectivamente

\usepackage{estilo_tfg}

% Lista de paquetes potencialmente interesantes (uso baixo demanda)

% \usepackage{alltt}       % proporciona o entorno alltt, semellante a verbatim pero que respecta comandos
% \usepackage{enumitem}    % permite personalizar os entornos de lista
% \usepackage{eurofont}    % proporciona o comando \euro
% \usepackage{float}       % permite máis opcións para controlar obxectos flotantes (táboas, figuras)
% \usepackage{hhline}      % permie personalizar as liñas horizontais en arrays e táboas
% \usepackage{longtable}   % permite construir táboas que ocupan máis dunha páxina
% \usepackage{lscape}      % permite colocar partes do documento en orientación apaisada
% \usepackage{moreverb}    % permite personalizar o entorno verbatim
% \usepackage{multirow}    % permite crear celdas que ocupan varias filas da mesma táboa
% \usepackage{pdfpages}    % permite insertar ficheiros en PDF no documento
% \usepackage{rotating}    % permite diferentes tipos de rotacións para figuras e táboas
% \usepackage{subcaption}  % permite a inclusión de varias subfiguras nunha figura
% \usepackage{tabu}        % permite táboas flexibles
% \usepackage{tabularx}    % permite táboas con columnas de anchura determinada

%%%%%%%%%%%%%%%%%%%%%%%%%%%%%%%%%%%%%%%%%%%%%%%%%%%%%%%%%%%%%%%%%%%%%%%%%%%%%%%%
% Corpo                                                                        %
%%%%%%%%%%%%%%%%%%%%%%%%%%%%%%%%%%%%%%%%%%%%%%%%%%%%%%%%%%%%%%%%%%%%%%%%%%%%%%%%

\begin{document}

 %%%%%%%%%%%%%%%%%%%%%%%%%%%%%%%%%%%%%%%%
 % Preliminares do documento            %
 %%%%%%%%%%%%%%%%%%%%%%%%%%%%%%%%%%%%%%%%

 \include{portada/portada}
 \iffalse
 \paxinaenbranco
 \dedicatoria{Dedicatoria} % escribe neste comando o teu texto de dedicatoria
 \paxinaenbranco
 \paxinaenbranco
 \begin{agradecementos}
 \blindtext % substitúe este comando polo teu texto de agradecementos
 \end{agradecementos}
 \paxinaenbranco
 \fi
 %%%%%%%%%%%%%%%%%%%%%%%%%%%%%%%%%%%%%%%%%%%%%%%%%%%%%%%%%%%%%%%%%%%%%%%%%%%%%%%%

\begin{abstract}\thispagestyle{empty}
    El Cloud Computing es un modelo que ofrece un conjunto de recursos (por ejemplo, redes, almacenamiento, aplicaciones, servicios y potencia de cálculo) que pueden ser aprovisionados bajo demanda de una forma rápida y sencilla, y minimizando el coste del servicio y el esfuerzo en cuanto a gestión de los recursos.\\\\
    Desde hace varios meses* Centro de Investigación en Tecnoloxías da Información e as Comunicacións (CITIC) de la Universidade da Coruña, ofrece un servicio de Cloud Computing al personal que allí trabaja. Este servicio consiste en que los usuarios pueden aprovisionar un conjunto de recursos, que se traducen en máquinas virtuales, del tamaño que necesiten para realizar tareas que no serían posibles en dispositivos convencionales.\\\\
    Actualmente el servicio es limitado y no está abierto a todos los usuarios del CITIC debido a la falta de una plataforma que centralice los accesos y que controle todo el ciclo de vida de los recursos aprovisionados. Para poder acceder, los usuarios deben hacer una petición a una persona encargada con las máquinas virtuales que necesitan y este debe activarlas. \\\\
    El objetivo principal de este proyecto es desplegar una plataforma sobre el servicio Cloud actual del CITIC, donde los usuarios puedan obtener recursos según sus necesidades de forma automática y simple, a la vez que se simplifica la gestión de las credenciales de usuarios enlazándolas con las credenciales de la UDC, y se establece un sistema que permita controlar y limitar, en cierta medida, el uso que se hace de los recursos y así evitar tener recursos ociosos. En definitiva, el objetivo es mejorar la eficiencia de esta Cloud y obtener todo su potencial.
    \item 
  \vspace*{25pt}
%\begin{multicols}{1}
\begin{description}
\item [\palabraschaveprincipal:] \mbox{} \\[-20pt]
\begin{itemize}
    \item Cloud Computing
    \item CITIC
    \item Virtualización
    \item SDDC
    \item Aprovisionamiento
\end{itemize}
\end{description}
%\end{multicols}

\end{abstract}

%%%%%%%%%%%%%%%%%%%%%%%%%%%%%%%%%%%%%%%%%%%%%%%%%%%%%%%%%%%%%%%%%%%%%%%%%%%%%%%%

 \paxinaenbranco

 \pagenumbering{roman}
 \setcounter{page}{1}

 \tableofcontents
 \listoffigures
 \listoftables
 \cleardoublepage
 
 \pagenumbering{arabic}
 \setcounter{page}{1}
 \bstctlcite{IEEEexample:BSTcontrol}

 %%%%%%%%%%%%%%%%%%%%%%%%%%%%%%%%%%%%%%%%
 % Capítulos                            %
 %%%%%%%%%%%%%%%%%%%%%%%%%%%%%%%%%%%%%%%%

 \chapter{Introdución}
\label{chap:introducion}
\lettrine{S}{egún} \textit{National Institute of Standards and Technology} (NIST), el Cloud Computing es un \textquote{modelo de recursos configurables y compartidos, accesibles através de la red bajo demanda y desde cualquier lugar en cualquier momento}\cite{DefCloudComputing}. Las principales características de este modelo son:
\begin{itemize}
    \item Autoservicio bajo demanda: El usuario puede aprovisionar recursos según sus necesidades y de forma automática sin requerir ninguna interacción humana con el proveedor del servicio.
    \item Acceso a través de red: El servicio es accesible a través de mecanismos estándar.
    \item Almacén de recursos: Los recursos se sirven a distintos consumidores al mismo tiempo siguiendo un modelo de tenencia múltiple. Estos se pueden gestionar de forma dinámica y permiten conocer su ubicación física a un nivel de abstracción alto.
    \item Elasticidad: Los recursos se pueden aprovisionar o liberar de forma elástica, es decir, que se pueden escalar de forma rápida según las necesidades del usuario.
    \item Servicio medido: El sistema Cloud es capaz de aportar información sobre los recursos que el cliente tiene aprovisionados (por ejemplo, almacenamiento, ancho de banda, procesamiento, y usuarios activos).
\end{itemize}
 
 El Centro de Investigación en Tecnoloxías da Información e as Comunicacións (CITIC) de la Universidade da Coruña ofrece un servicio de Cloud Computing para el personal que trabaja en sus instalaciones, que les permite aprovisionar recursos de un servidor en forma de máquinas virtuales con unas especificaciones determinas para realizar tareas que precisan gran capacidad de cómputo, de almacenamiento, o de red. Aunque el servicio ya está en uso, tiene grandes inconvenientes en la gestión y en el acceso que no permiten ofrecerlo de forma abierta a todos los usuarios del centro.\\
 
Actualmente el sistema cuenta con una plataforma donde los usuarios no pueden acceder y obtener los recursos bajo demanda de forma dinámica y sencilla, esto tiene que ser realizado por una persona encargada de recibir sus peticiones y de activar máquinas virtuales solicitadas, un proceso no automático. Este problema surge por la falta de un sistema con una interfaz intuitiva, ya que el portal de acceso actual solo cuenta con una vista de administrador del sistema con demasiadas opciones de configuración que el usuario final no debería tener disponibles. A través de dicha vista, los usuarios tampoco disponen de un directorio personal al que solo ellos tienen acceso para administrar sus recursos, con el portal actual tendrían acceso a todas las máquinas virtuales del sistema pudiendo eliminar o modificar cualquier recurso de otro usuario. Además, la plataforma tampoco incorpora un sistema de medición de consumo de los recursos obtenidos por un usuario, así como tampoco hay un sistema que permita limitar la cantidad de recursos que un usuario puede aprovisionar.\\

Este servicio Cloud no cumple con las características esenciales de un sistema de Cloud Computing, como el autoservicio bajo demanda, el acceso a través de red, la gestión dinámica y elástica, y la monitorización y medición de los recursos, por lo que se hace necesario desplegar una herramienta, o herramientas, para separar el entorno de administración del entorno de usuario y crear un servicio Cloud lo más aproximado posible al modelo NIST, con la intención de obtener el máximo potencial de la infraestructura mejorando su eficiencia para que sea un servicio útil.

%\lettrine{P}{rimer} capítulo da memoria, onde xeralmente se exporán as
%liñas mestras do traballo, os obxectivos, etc. Incluimos un par de
%exemplos de citas~\cite{ErlangBook,ErlangWebBook} e de referencias
%internas (sección \ref{sec:mostra}, páxina \pageref{sec:mostra}).


\section{Motivación}
La motivación para realizar este proyecto se basa en mejorar el servicio Cloud del CITIC para que aquellos usuarios que necesiten equipos de grandes prestaciones para sus tareas puedan conseguirlos de una forma sencilla y ágil al mismo tiempo que se mejora la gestión interna del servicio, y así reducir sus costes e incidencias a largo plazo. En definitiva, hacer que esta herramienta sea eficiente, útil y capaz de dar servicio a todos sus usuarios.


\section{Objetivos}
El objetivo principal de este proyecto es mejorar el servicio Cloud que ofrece el CITIC desplegando una herramienta, usando como base las que ya están funcionando actualmente, para mejorar su eficiencia y obtener el máximo potencial de la infraestructura física. Con esto se pretende que el servicio sea útil y accesible para todos los usuarios.\\\\
Los objetivos concretos se pueden resumir en los siguientes:
\begin{itemize}
    \item Centralizar y mejorar la gestión de usuarios integrando el sistema de cuentas de la UDC.
    \item Desplegar una herramienta que facilite el acceso a la plataforma, que permita a los usuarios gestionar sus recursos y que simplifique el aprovisionamiento de recursos.
    \item Implementar un sistema de facturación que permita limitar y controlar la cantidad de recursos que un usuario pueda obtener, y así evitar tener demasiados recursos ociosos.
    \item Documentar las soluciones desplegadas en el sistema para facilitar la transmisión de conocimiento a largo plazo.
\end{itemize}


 \chapter{Estado de los recursos}
\label{chap:estadoInfraestructuraSistema}
\lettrine{C}{on} el fin de contextualizar los recursos que se utilizarán en este trabajo, en este capítulo se expone la situación actual de toda la infraestructura en lo relacionado al software que está en funcionamiento, a los recursos físicos de los que se compone, y al estado actual de las herramientas que rodean a dichos recursos.

\section{Infraestructura}
La infraestructura física de este servicio de Cloud Computing, se encuentra localizada en el edificio del CITIC de la UDC, dentro de un rack alojado en su Centro de Proceso de Datos (CPD) \cite{citicUDC}. Está formado por 5 nodos \textit{Lenovo NeXtScale nx360 M5} y 3 nodos \textit{Dell EMC PowerEdge R740}. Ambos componentes dan flexibilidad en cuanto a la escalabilidad y ofrecen gran rendimiento de cómputo.\\

Especificaciones principales de los nodos:
\begin{itemize}
    \item Lenovo NeXtScale nx360 M5: 
        \begin{itemize}
            \item CPU: Dos Intel Xeon E5-2650
            \item Memoria: 128 GB
            \item Tarjeta  gráfica: Tesla M60
        \end{itemize}
            Más información: \href{https://lenovopress.com/tips1195-nextscale-nx360-m5-e5-2600-v3}{Especificaciones}
    \item Dell EMC PowerEdge R740:
        \begin{itemize}
            \item CPU: Dos Xeon Gold 6146
            \item Memoria: 384 GB
            \item Tarjeta gráfica: Tesla P40
        \end{itemize}
    Más información: \href{https://www.dell.com/es-es/work/shop/servidores-almacenamiento-y-redes/smart-value-poweredge-r740-server-standard/spd/poweredge-r740/per7400m}{Especificaciones}
\end{itemize}
El almacenamiento es independiente y está colocado en una ubicación distinta a la de los nodos. Esta conformado por 13 discos duros SSD de 3.84 TB de capacidad, obteniendo así una capacidad total de casi 50 TB, pero que utilizan el sistema de almacenamiento RAID 5 lo cual permite replicar los datos para conseguir mayor  integridad, tolerancia a fallos y ancho de banda, reduciendo la cantidad de almacenamiento utilizable a 34 TB. Estos discos forman un \textit{pool} de almacenamiento que se divide en cinco LUNs (\textit{Logical Storage Unit}) de 2 TB cada una, representadas en el sistema de virtualización como cinco \textit{datastores} diferentes que utilizan el sistema de archivos VMFS el cual optimiza el almacenamiento de máquinas virtuales.\\
Estos discos están colocados en varias cabinas que son accesibles por los nodos a través de dos switches para aportar redundancia. Para ello, las cabinas incorporan dos controladores SFP+ que se conectan a cada switch mediante dos puertos que aportan conectividad 10 Gb y, además, incorporan otros dos puertos con conectividad 1 Gb para la gestión de los discos. Estas conexiones utilizan los protocolos de red Ethernet y iSCSI , formando así, junto con el resto de componentes descritos, la estructura de una SAN [\ref{fig:esquemaentornoreal}].\\

La gestión del almacenamiento se realiza en la capa física, el nivel más bajo por lo que la configuración de cada LUN que utilizan las máquinas virtuales desplegadas se tiene que hacer antes de conocer los requisitos necesarios de lo que se vaya a desplegar en la capa software. Esto provoca que si se quiere desplegar una máquina virtual con más capacidad de almacenamiento o con una estructura RAID diferente haya que crear una nueva LUN que se adapte a los requisitos.
Esta gestión hace que el uso de recursos de almacenamiento no sea el óptimo ya que no permite ajustar de forma precisa y rápida cada configuración a los requisitos necesarios generando así mayor coste.

\iffalse
Los nodos acceden a los discos de almacenamiento a través de dos switches que al mismo tiempo dan conectividad entre los nodos formando una SAN. Los 10TB de capacidad están repartidos entre cinco DataStores, de 2TB cada uno. Una máquina virtual está alojada en un DataStore concreto pero puede tener ficheros alojados en varios almacenes de datos diferentes.
Las conexiones entre dispositivos son Ethernet 10 Gigabit combinado con el protocolo de transporte ISCSI
\fi

\section{Software}
\label{subsec:softwareinstalado}
Actualmente el servicio está basado en el software de la empresa VMware, uno de los principales proveedores de software de virtualización, siendo \emph{VMware vSphere} el software desplegado sobre la infraestructura. El primero se encarga de virtualizar parte de la infraestructura física y de proporcionar las herramientas necesarias para gestionarla. Sus principales componentes son los siguientes:
\begin{itemize}
    \item ESXi: Hipervisor propio de VMware, de tipo 1 o \textit{bare metal}[\ref{itm:baremetal}]. No requiere de sistema operativo para funcionar ya que funciona directamente sobre el hardware físico\cite{Esxi}. Este hipervisor está instalado en cada uno de los cinco nodos que forman la infraestructura.
    \item VMware vCenter Server: proporciona una plataforma centralizada para la gestión, operación, aprovisionamiento de recursos, y evaluación de rendimiento de las máquinas virtuales y los nodos. Incluye otros componentes\cite{componentesCenterServer}:
        \begin{itemize}
            \item Single Sign-On: es un servicio de autenticación. Permite que los componentes de vSphere se puedan comunicar sin tener que cada uno se tenga que autenticar de forma separada gracias a que crea un dominio de autenticación. También se encarga de la autenticación de usuarios, permitiendo usar un directorio de usuarios externo.\label{itm:singlesingonEX}
            \item vSphere License Server: permite gestionar e inventariar licencias para aquellos sistemas conectados a un \emph{Platform Services Controller}.
            \item VMware Certificate Authority: provee con un certificado a cada nodo. Está firmado por esta misma autoridad (VMCA).
            \item PostgreSQL: distribución de la base de datos PostgreSQL para vSphere.
            \item vSphere Web Client y vSphere Client: interfaz que permite conectarse a una instancia de vCenter Server para gestionar la infraestructura.
            \item vSphere ESXi Dump Collector: permite configurar un host ESXi para que guarde su memoria en un servidor externo en lugar de en un disco cuando hay algún fallo crítico.
            \item vSphere Syslog Collector: habilita logs de red.
            \item vSphere Auto Deploy: herramienta que permite desplegar gran cantidad de nodos físicos de forma automática.
            \item vSphere Update Manager Extension: centraliza y gestiona las actualizaciones de vSphere. 
        \end{itemize}
    
    \item Web Client: interfaz web que permite acceder a vCenter Server de forma remota.
    \item vMotion: permite la migración de máquinas virtuales de un servidor físico a otro de forma transparente.
    \item Storage vMotion: permite migrar los ficheros de una máquina virtual de un un datastore a otro, pudiendo repartirlos en diferentes datastores.
    \item High Availability (HA): provee alta disponibilidad para las máquinas virtuales. En caso de que un componente de la infraestructura falle recoloca la máquina en otra ubicación.
    \item Distributed Resource Scheduler (DRS): se encarga de balancear la carga de cómputo entre el hardware disponible. Ayuda a reducir el consumo de energía.
    \item Storage DRS: balancea la carga de almacenamiento y las operaciones I/O entre los diferentes datastores disponibles.
    \item Fault Tolerance: provee la disponibilidad continua de las máquinas virtuales habilitadas creando una copia de cada una  para usarla en caso de que la primera falle.
    \item Distributed Switch (VDS): habilita swithces virtuales que se encargan de gestionar el tráfico de los hosts ESXi.
    \item Virtual Machine File System (VMFS): vSphere proporciona un sistema de almacenamiento optimizado para tener alto rendimiento con sus hosts ESXi. Este es el sistema de archivos utilizado en los DataStores mencionados.
\end{itemize}

\begin{figure}[hp]
  \centering
  \includegraphics[width=0.75\textwidth]{imaxes/cap2recursos/contentVSphere}
  \caption{Elementos de la plataforma VMWare vSphere\cite{fotovSphere}}
  \label{fig:componentesVSphere}
\end{figure}
\begin{figure}[hp]
  \centering
  \includegraphics[width=1\textwidth]{imaxes/cap2recursos/recursosReal.png}
  \caption{Esquema de los recuros software y hardware del entorno}
  \label{fig:esquemaentornoreal}
\end{figure}

\FloatBarrier
\subsection{Estado de la tecnología}

En los últimos tiempos los servicios de IaaS (\textit{Infrastructure as a Service}) se han extendido de forma considerable con la aparición de software que permite la gestión de un sistema de Cloud Computing, como pueden ser VMware Cloud Foundation (2011), OpenStack (2010), o Apache CloudStack (2012). Estas herramientas construyen una infraestructura virtual sobre un entorno físico estandarizado que les permite administrar y automatizar la escalabilidad, el sistema de almacenamiento, la disponibilidad del servicio, la red, y la seguridad del servicio, con lo que se consigue reducir el coste y el tiempo de gestión y configuración, haciendo la infraestructura física se hace más eficiente.

A la hora de alcanzar los objetivos descritos en este proyecto se nos plantea la duda de que solución software desplegar ya que actualmente existen tres principales alternativas en el mercado, VmWare Cloud Foundation, OpenStack y Apache CloudStack. Cada una de ellas ofrece diferentes características con diferentes requisitos que se pueden adaptar mejor o peor al entorno de despliegue, pero después de comprobar esos aspectos tenemos claro que la solución elegida es \emph{VmWare Cloud Foundation}.

\subsubsection{VmWare Cloud Foundation}
Esta solución virtualiza todas las capas de la infraestructura[\ref{fig:infraCloudFound}] (red, computación y almacenamiento) combinando cuatro componentes principales, vSphere para gestionar el cómputo, vSAN para la gestión del almacenamiento, NSX para la gestión de la red, y vRealize para gestionar todas las operaciones que tienen lugar en el servicio, integrando todos los componentes para que la gestión de la infraestructura sea lo más simple posible. Este cojunto de herramientas convierten el CPD en un \emph{Software Defined Datacenter} (SDDC), un entorno donde todas las partes físicas de la infraestructura pasan a estar controladas a través de software haciendo más flexible, independiente y menos costosa la configuración de estos componentes[\ref{fig:sddcoverview}]. Las principales características de Cloud Foundation son:
\begin{itemize}
    \item \emph{Integración nativa}: todos sus componentes se integran de forma nativa entre ellos, así como con otros componentes de la empresa VmWare, minimizando las tareas de configuración y administración.
    \item \emph{Experiencia de usuario simple}: gracias a la gran cantidad de procesos automatizados.
    \item \emph{Escalabilidad modular}: el sistema y la infraestructura se puede escalar de forma sencilla.
    \item \emph{Cloud híbrida}: da la posibilidad de conectar una Cloud pública con una Cloud privada y así tratar ambas como una única Cloud.
\end{itemize}

\begin{figure}[h!]
  \centering
  \includegraphics[width=1\textwidth]{imaxes/cap2recursos/SDDCoverview.png}
  \caption{Partes virtualizadas en un SDDC.}
  \label{fig:sddcoverview}
\end{figure}
\\
\begin{figure}[h!]
  \centering
  \includegraphics[width=1\textwidth]{imaxes/cap2recursos/overviewCF.png}
  \caption{Cloud Foundation virtualiza toda la infraestructura.}
  \label{fig:infraCloudFound}
\end{figure}

Cloud Foundation permite reducir el tiempo de mantenimiento ya que todo está controlado por el software que integra todos los componentes, automatizando gran parte de las operaciones y el ciclo de vida de todos los elementos desde su creación, como puede ser el control de versiones de cada elemento, los perfiles de usuario, y las máquinas virtuales creadas, además de proporcionar una plataforma de acceso para que cada usuario pueda gestionar sus recursos.
También permite separar las cargas de trabajo mediante Dominios según el tipo de trabajo que se va a realizar, pudiendo acceder a cada uno de ellos de forma separada.




Para poder usar este software es necesaria la adquisición de licencias, estas se organizan por componente y por número de hosts sobre los que se va a instalar el producto, haciendo que el coste sea elevado, pero, a pesar de eso, se ha elegido este paquete principalmente por su integración nativa con los componentes ya instalados y porque su mantenimiento es más sencillo. \\
Si bien VMWare ofrece plugins para conectar sus componentes con otras soluciones, como es el caso de OpenStack \cite{opestackintegrated}, estos no ofrecen el rendimiento que da la integración nativa, además, en caso de recibir actualizaciones, habría que actualizar cada componente de forma individual aumentando el riesgo de incompatibilidades con el resto de elementos del sistema, mientras que Cloud Foundation gestiona todo el ciclo de vida de cada actualización para cada componente, permitiendo comprobar si existe alguna incompatibilidad con el resto de versiones antes de aplicar una actualización. En definitiva, VmWare Cloud Foundation simplifica el proceso de instalación, configuración, gestión, y mantenimiento, tanto para los usuarios como para el administrador del sistema.

\subsubsection{Componentes de VmWare Cloud Foundation \cite{componentesCloudFound}}
\label{subsubsect:cfcomponents}
\begin{itemize}
    \item \textbf{SDDC Manager}: gestiona el ciclo de vida de todos los componentes del sistema, incluyendo el proceso inicial de despliegue de Cloud Foundation, su configuración y aprovisionamiento, y las actualizaciones. Monitoriza los recursos físicos y lógicos de la infraestructura, facilita su configuración y permite añadir nuevos recursos cuando sea necesario. \iffalse Todo esto lo realiza mediante flujos de trabajo que facilitan la detección de orígenes de errores.\fi
    \item \textbf{vSphere}: ya está incluído en el servicio actual [\ref{subsec:softwareinstalado}].
    \item \textbf{vSAN}: componente clave que virtualiza el almacenamiento. Como ya se ha explicado, el almacenamiento del servicio actual está configurado con LUNs que se deben gestionar individualmente en una capa distinta a los componentes software, provocando que su configuración sea más compleja y costosa. Esto es lo que gestiona vSAN, trata toda la capacidad de almacenamiento como un único elemento, eliminando así la necesidad de tener que crear LUNs aisladas, consiguiendo abstraer la configuración de almacenamiento de la capa física en la capa de software y permitiendo establecer políticas desde cada máquina virtual para adecuarlo a las necesidades de cada una sin tener que editar la configuración real de los discos. Así el uso de almacenamiento es más eficiente, flexible y su configuración está integrada dentro del mismo servicio. \\
    En vSAN, en lugar de tratar el almacenamiento de forma independiente este pasa a estar ligado a cada host, es decir, cada uno de los nodos tiene asignados hasta cinco grupos de discos. Estos grupos de discos pueden ser híbridos, donde se combinan discos duros SSD y HDD, o All-Flash, donde todos los discos son SSD. Dentro de cada grupo hay \underline{dos tipos de discos} con distintas funciones, el disco de caché y el disco de capacidad\cite{operacionesVSAN}:
        \begin{itemize}
            \item \textbf{Caché}: Hay uno en cada grupo. Realiza la función de memoria caché y se encarga de escribir los datos persistentes en los discos de capacidad.
            \item \textbf{Capacidad}: Puede haber hasta siete discos en cada grupo. Almacena los datos persistentes del entorno.
        \end{itemize}
    En cada grupo de discos la gestión de la \underline{lectura y escritura} de datos se hace de la siguiente forma:
        \begin{itemize}
            \item \textbf{Lectura}: En el caso de la solución híbrida, si el dato que se busca no está en el disco de caché entonces se busca en los discos de capacidad y después se incorpora al disco de caché. Con la solución all-flash, los datos se leen siempre directamente de los discos de capacidad sin que estos sean escritos en el disco de caché dejando a este completamente libre para las operaciones de escritura. Gracias a esto, la estructura all-flash ofrece mayor rendimiento respecto a la híbrida.
            \item \textbf{Escritura}: Tanto en la solución híbrida con en la all-flash, el host ESXi primero escribe en el disco de caché, este le responde con una confirmación de escritura y más tarde vSAN se encarga de escribir ese dato en los discos de capacidad cuando el disco de caché está casi completo o cuando el dato lleva un tiempo sin ser utilizado.
        \end{itemize}
        \begin{figure}[h!]
            \centering
            \includegraphics[width=1\textwidth]{imaxes/cap2recursos/rendimientoVSAN.png}
            \caption{Almacenamiento All-Flash vs. Híbrido en vSAN}
            \label{fig:rendimientoVSAN}
        \end{figure}
        \FloatBarrier
    El acceso al almacenamiento desde cada nodo se hace a través de IP en una red donde están todos los nodos. \\ 
    Con esto se reducen las tareas de gestión del almacenamiento físico ya que ya no es necesario hacer ajustes en la capa física para cumplir unos requisitos en la capa software.
    \item \textbf{NSX}: otro de los componentes clave. Tiene un papel similar a vSAN, pero en este caso se encarga de virtualización de los componentes físicos de la red de la infraestructura, es decir, abstrae los componentes físicos de nuestro entorno en software, dando más libertad a la hora de establecer la topología y componentes físicos de la red. Incluye servicios como firewall, DNS, DHCP, VPN, NAT, enrutamiento, balanceo de carga, o switching, que permiten reducir la cantidad de dispositivos físicos de red.\\
    Principales \underline{componentes internos}\cite{componentesNSX}:
    \begin{itemize}
        \item \textbf{NSX Manager}: permite crear, configurar y administrar el resto de recursos de NSX. Su interfaz está integrada en vSphere.
        \item \textbf{NSX Controller}: contiene las tablas de ARP, MAC, VTEP y de enrutamiento.
        \item \textbf{NSX Virtual Switch}: gestiona los vSphere vSwitch Distibuted desplegados, proporcionando switching a los nodos ESXi.
        \item \textbf{NSX Logical Router Control}: se despliega cuando se crea un Distributed Logical Router. Se encarga de buscar adyacencias para generar tablas de enrutamiento que envía a NSX Manager y a los NSX Cotrollers para que informen a cada Distributed Logical Router en cada nodo ESXi.
        \item \textbf{NSX Edge}: proporciona servicios múltiples servicios como firewall perimetral de capa 2 y 3, SSL, NAT, DHCP, VPN, balanceo de carga y alta disponibilidad.
    \end{itemize}
    \item \textbf{vRealize Log Insight}: realiza la gestión de logs del servicio, dando visibilidad a todas las operaciones del servicio, generando análisis del sistema. Esto permite tener mayor conocimiento de los riesgos, eficiencia y uso de recursos, a parte de facilitar la búsqueda de orígenes de errores.\\
    Para su correcto funcionamiento \underline{requiere desplegar los siguientes componentes}:
    \begin{itemize}
        \item \textbf{Nodo Master}: cuando se despliega con el modo \textit{standalone} este el responsable de todas las actividades, incluyendo las consultas y gestión de logs. También participa en la gestión del ciclo de vida del cluster actualizando, eliminando y añadiendo nodos \textit{Worker}. 
        \item \textbf{Nodo Worker}: se generan para proporcionar alta disponibilidad facilitando la escalabilidad del entorno. En estos nodos se delegan las tareas de consulta y gestión de logs.
        \item \textbf{Load Balancer}: se encarga de centralizar y asegurar la entrada de logs en Log Insight simplificando la configuración para habilitar la alta disponibilidad. También balancea el tráfico de logs entrante entre los nodos existentes.
    \end{itemize}
\end{itemize}

**Decir como se puede implementar un sistema de facturación.*******\\
**Como se puede conectar los usuarios de la UDC.********












\iffalse
\subsubsection{OpenStack}
Es una plataforma constituída por tres proyectos, uno dedicado a la gestión de la red, otro a la gestión del cómputo, y otro a la gestión del almacenamiento. Para poder desplegarlo sobre los componentes de VmWare instalados en nuestra infraestructura son necesarios tres plugins específicos para el software de VmWare. OpenStack aporta una API a través de la cual es posible gestionar los recursos virtuales y el aproivisionamiento.

\subsubsection{Apache CloudStack}
\fi



 \chapter{Planificación}
\label{chap:planificacionProyecto}
\lettrine{E}{n} este capítulo se propone una planificación del proyecto con el fin de organizar su estructura y exponer sus costes temporales y económicos aproximados necesarios para su realización.

\section{Tareas}
\underline{Tarea 1}. Analizar como está formada la infraestructura, que componentes hardware y software la componen y cual es la función de cada uno de ellos.\\
En cuanto a la parte física se comprueban las especificaciones concretas del hardware de cómputo, almacenamiento y red. También como están organizados y estructurados tanto el sistema de almacenamiento y la red de la infraestructura. En la parte de software, se detallan las funciones de los principales programas y servicios que están instalados en el entorno.\\

\underline{Tarea 2}. Analizar y seleccionar una herramienta de las disponibles en el mercado que se adapte a las necesidades del servicio que se quiere construir y a las características de la infraestructura. La herramienta seleccionada debe permitir reducir el coste y la complejidad de los trabajos de mantenimiento y administración del servicio a la vez que el usuario final lo utiliza de forma sencilla. En este proceso también se debe tener en cuenta la compatibilidad y eficiencia de la nueva herramienta con los componentes ya existentes en el entorno.\\

\underline{Tarea 3}. Tarea que agrupa las tareas dedicadas al proceso de configuración de la infraestructura, configuración de la herramienta seleccionada y su instalación. Estas son las tareas 4, 5, 6, 7, 8, 9 y 10.

\underline{ Tareas 4, 5, 6, 7, 8, 9 y 10.}
Comprobación de requisitos, preparación del entorno, establecimiento de parámetros configuración, despliegue de la plataforma sobre la infraestructura existente y configuración de la plataforma después del despliegue. Antes de realizar la instalación de la nueva herramienta es necesario comprobar sus requisitos necesarios para que las capacidades del servicio final se adapten a las necesidades de uso (tareas 4 y 5). También se deben establecer los parámetros de configuración iniciales que se van a aplicar a la nueva plataforma (tarea 6). Durante el proceso de comprobación de requisitos puede surgir la necesidad de realizar cambios sobre las capacidades de la infraestructura y la configuración de los componentes ya existentes en el entorno inicial para que este se adapte a los requisitos de la nueva plataforma (tareas 7 y 8). Una vez el entorno está preparado para la herramienta pueda ser instalada entonces se efectúa el despliegue (tarea 9), posteriormente se configura y se comprueba el funcionamiento del nuevo servicio (tarea 10).\\

\underline{Tarea 11.} Fin de la instalación y configuración de la plataforma. Marca el final del despliegue y configuración del nuevo servicio en la infraestructura.

\iffalse
\underline{xxx}. Configuración de los parámetros de la plataforma una vez desplegada. Finalizado el proceso de la tarea anterior, se comprueba el correcto funcionamiento de la plataforma y se configura de forma que esta se adapte a las necesidades de uso del servicio.\\
\fi

\underline{Tarea 12}. Diseñar una integración de la nueva plataforma con el sistema de autenticación de la UDC para que los usuarios finales del servicio se puedan autenticar sin necesitar nuevas credenciales. Para ello es preciso comprobar el método de acceso al directorio de usuarios de la UDC y la forma de conectarlo con la plataforma desplegada para, posteriormente, realizar un diseño de la solución. Este proceso requiere realizar una solicitud de acceso a los servicios internos de la UDC.\\

\underline{Tarea 13}. Implementación y despliegue de la integración para la autenticación de usuarios con sus credenciales de la UDC. Durante este proceso puede ser necesario realizar cambios sobre la configuración de perfiles de usuarios que está establecida en la plataforma.\\

\underline{Tarea 14}. Análisis del uso que harán los usuarios del servicio para establecer políticas sobre el uso de recursos. Para realizar este cálculo, primero se debe analizar el uso previo al despliegue del nuevo servicio que los usuarios hacen de la infraestructura y, después, estimar el uso que pueden llegar a realizar una vez el servicio sea accesible. Hay que tener en cuenta la cantidad de usuarios que lo utilizan, que lo van a utilizar y la cantidad de recursos que se emplean y que se van a emplear. Una vez obtenida una estimación, se realiza un diseño de las políticas que se van a aplicar.\\

\underline{Tarea 15}. Diseño de un sistema de facturación/valoración de los recursos del servicio en base a las políticas de uso establecidas. Basándose en las políticas establecidas en la tarea 14, se debe pensar como se pueden aplicar sobre el servicio. Esto puede ser a través de una herramienta externa, en ese caso sería necesario realizar un desarrollo, o integrando la configuración en los parámetros de configuración de la plataforma.\\
La intención de este sistema es limitar la cantidad de recursos que un usuario puede aprovisionar permitiendo aumentar la eficiencia de los recursos físicos reduciendo la cantidad de recursos ociosos.\\

\underline{Tarea 16}. Implementación y despliegue del sistema de facturación/valoración diseñado. Para implementar este sistema puede que sea necesario realizar el desarrollo de una herramienta si se determina que no es posible establecerlo a través de los parámetros de configuración de la plataforma. \\

\underline{Tarea 17, 18 y 19}. Recopilación de la información necesaria para la realización de cada tarea. La información de apoyo se debe obtener de documentaciones, artículos, vídeos o libros de  fuentes fiables como empresas desarrolladoras de los productos utilizados o expertos especializados. El objetivo la recopilación de información es obtener conocimiento sobre las herramientas con las que se está trabajando para luego tener una base que facilite la realización de las tareas descritas. Esto se realiza desde el comienzo del proyecto hasta su finalización para tener claros los conceptos que se desarrollan y para conocer los detalles del trabajo que hay que realizar en cada tarea.\\

\underline{Tarea 20, 21 y 22}. Redacción de la memoria del proyecto. Se escribe un documento con todos los detalles de todas las tareas realizadas durante el proyecto, incluyendo los cambios realizados en la infraestructura, las configuraciones establecidas y como se lleva a cabo cada proceso del proyecto. Su objetivo es transmitir el conocimiento adquirido durante el proyecto sobre como realizar el despliegue de una plataforma de virtualización y los beneficios que esta puede tener.La escritura de este documento se realiza a la vez que completa cada tarea para detallar los pasos realizados en cada caso, por lo que su duración es igual a la duración total de todo el proyecto.\\


La duración total del proyecto se estima en 101 días. teniendo en cuenta que el estudiante trabaja durante 4 horas diarias. El coste mostrado se refiere al coste correspondiente al estudiante si trabaja por 25 €/hora[Fig. \ref{fig:estadisticasproyecto}]. 

\begin{landscape}
\begin{figure}[hp]
  \centering
  \includegraphics[width=1.5\textwidth]{imaxes/extras/diagramaGranttt.png}
  \caption{Diagrama de Grantt sobre la planificación del proyecto.}
  \label{fig:tareasproyecto}
\end{figure}
\end{landscape}
\begin{figure}[h!]
  \centering
  \includegraphics[width=1\textwidth]{imaxes/extras/estadisticasProyecto.png}
  \caption{Estadísticas sobre la planificación del proyecto.}
  \label{fig:estadisticasproyecto}
\end{figure}
\FloatBarrier

\section{Costes}
Los principales costes del proyecto son aquellos relacionados con los trabajadores que lo llevan a cabo y las licencias necesarias para cada componente de VMware Cloud Foundation en la infraestructura\footnote{Los componentes que se especifican son aquellos que son obligatorios para desplegar VMware Cloud Foundation.}.\\
Cada componente de VMware Cloud Foundation requiere su propia licencia\cite{licenses}. Estos componentes son SDDC Manager, VMware vSphere, VMware vCenter, VMware vSAN, VMware NSX for vSphere y VMware vRealize Log Insight. El precio de cada licencia dependerá del número de CPUs físicas sobre las que se va usar esta plataforma por lo que, como en la infraestructura hay un total de ocho hosts con dos CPUs cada uno, \underline{el precio por cada componente} es el siguiente:
\begin{itemize}
    \item \textbf{SDDC Manager}: 1800€\footnote{Para la edición \textit{Advanced} de VMware Cloud Foundation.} por CPU y 6500€ anuales por soporte. El precio total de la licencia es de 28800€ por 16 CPUs.
    \item \textbf{VMware vSphere}: 4000€\footnote{Para la edición \textit{Standard} de VMware vSphere.} por CPU. El precio total de la licencia es de 64000€ por 16 CPUs y el precio anual por las tareas de soporte es de 16000€.
    \item \textbf{VMware vCenter}: 6000€ por CPU. El precio total de la licencia es de 96000€ por 16 CPUs y el precio anual por las tareas de soporte es de 24000€.
    \item \textbf{VMware vSAN}: 4000€\footnote{Para la edición \textit{Advanced} de VMware vSAN.} por CPU. El precio total de la licencia es de 64000€ por 16 CPUs y el precio anual por las tareas de soporte es de 16000€.
    \item \textbf{VMware NSX for vSphere}: 5300€\footnote{Para la edición \textit{Advanced} de NSX.} por CPU. El precio total de la licencia es de 84400€ por 16 CPUs y el precio anual por las tareas de soporte es de 21100€.
    \item \textbf{VMware vRealize Log Insight}: 1500€ por CPU. El precio total de la licencia es de 24000€ por 16 CPUs y el precio anual por las tareas de soporte es de 6000€.
\end{itemize}

El precio total de todas las licencias necesarias para el entorno, teniendo en cuenta que hay 16 CPUs, sería igual a 361200€, y el precio total por las tareas de soporte sería igual a 83100€ anuales.\\

En caso de que ya estén instalados algunos de los componentes entonces solo se requieren licencias para aquellos componentes que aún no están en el entorno. En el caso del entorno inicial, los componentes que ya están instalados son VMware vSphere, VMware vCenter Server. Esto hace que el \underline{coste real para implementar VMware Cloud Foundation} en el entorno sea igual a 201200€, ya que solo son necesarias licencias para los componentes SDDC Manager, VMware vSAN, VMware NSX for vSphere y VMware vRealize Log Insight. El coste total de la instalación y mantenimiento de la plataforma VMware Cloud Foundation sobre la infraestrructura del CITIC es el siguiente:

    \begin{itemize}
        \item \textbf{Licencias}: 201200€ en total.
        \item \textbf{Soporte}: 83100€ anuales.
        \item \textbf{Sueldo empleado}: 5031,25€ en total.
    \end{itemize}



 \chapter{Metodología}
\label{chap:Metodologia}

%
\section{Prueba de Concepto}
Para la realización de este proyecto se ha decidido crear un entorno virtual dentro de la infraestructura real donde está localizado el servicio, para poder desplegar el software Cloud Foundation y probarlo sin afectar a la configuración ni al funcionamiento general del servicio. La parte física de este entorno está formada por cuatro máquinas virtuales que equivalen a cuatro nodos físicos con el hipervisor ESXi instalado en cada uno. También cuenta con una máquina virtual con vCenter Server para permitir el acceso al entorno, y otra máquina virtual con Windows Server 2016 donde se habilitan servicios de red como DNS o NTP.
*****ESTRUCTURA DEL ENTORNO.
\subsection{Especificaciones del entorno}
Estas son las características de hardware y software del nuevo entorno de pruebas.
\begin{itemize}
    \item Cuatro nodos con la siguiente configuración:
    \begin{itemize}
        \item Hipervisor: VMware ESXi, 6.7.0
        \item Procesador: Intel(R) Xeon(R) CPU E5-2650 v4 @ 2.20GHz
        \item Memoria: 24GB
        \item Discos de almacenamiento: un disco duro de 100GB y tres discos duros de 200GB
    \end{itemize}
    \item Almacenamiento: \\
        Este entorno, a diferencia del real, está basado en vSAN lo cual unifica los discos duros de cada host en un único almacén de datos dividido en cuatro grupos de discos donde el disco de 100GB representa el disco de memoria caché y los tres discos de 200GB representan la memoria de almacenamiento. La capacidad total útil del clúster es de 2,34TB.
    \item Software:
    **que hay instalado
\end{itemize}

\section{Conceptos previos}
En este apartado se describe la arquitectura de VMware Cloud Foundation y como estructura sus componentes\footnote{Se describen solo aquellos componentes que se utilizarán en el despliegue de Cloud Foundation.} internamente.

%%%%%%%%%%%%%%%%%%%%%%%%%%%%%%
\iffalse
En este apartado se explican aquellos conceptos de VMware Cloud Foundation necesarios para entender su funcionamiento, configuración y requisitos de la infraestructura previos al despliegue del servicio.
\fi
%%%%%%%%%%%%%%%%%%%%%%%%%%%%%%%%


%% Workload Ddomains %&%&%%%%
%%%%%%%%%%%%%%%%%%%%%%%%%%%%
\subsection{Workload Domain}
Un \textit{workload domain} consiste en una instancia lógica de un SDDC que abarca todos o parte de los recursos de uno o más clusters, cuya función es aislar el flujo de trabajo de un usuario, aplicación o un determinado tipo de tareas. Cada \textit{workload domain} se extiende sobre varios hosts y contiene su propia instancia de vCenter Server, vSAN y NSX, lo cual permite establecer políticas de control únicas para todos los \textit{workload domains} y específicas para cada uno de ellos a la vez que se simplifica la complejidad de la infraestructura. Existen \underline{tres tipos} de \textit{workload domains} que permiten aislar las tareas de gestión de la infraestructura del resto de flujos de trabajo. 

%% MANAGEMENT DOMAIN
\subsubsection{Management Domain}
\label{subsubsec:domainManagement}
Este \textit{workload domain} se crea y configura automáticamente durante el proceso de despliegue de una instancia de VMware Cloud Foundation. Su función es gestión de todos los componentes de VMware Cloud Foundation, de toda la infraestructura y del resto de \textit{workload domains} existentes en el entorno a tráves de políticas establecidas desde un único punto. Los componentes dedicados a este \textit{workload domain} son SDDC Manager, vCenter Server, dos Platform Services Controllers que son redundantes, vRealize Log Insight y un vSwitch Distributed [Fig. \ref{fig:esquemaCFComponentes}].\\
Cuando se \underline{despliega Management Domain se crean y configuran} de forma automatizada por SDDC Manager las siguientes máquinas virtuales (VM) de cada componente de Cloud Foundation:
\begin{itemize}
    \item Una VM de \textbf{SDDC Manager}: 4 vCPU, 16 GB de memoria, 800 GB de almacenamiento.
    \item Una VM de \textbf{vCenter Server}: 4 vCPU, 16 GB de memoria, 290 GB de almacenamiento.
    \item Dos instancias de \textbf{Platform Services Controller} (cada una): 2 vCPU, 4 GB de memoria, 60 GB de almacenamiento.
    \item Una VM de \textbf{NSX Manager}: 4 vCPU, 16 GB de memoria, 60 GB de almacenamiento.
    \item Tres VM de \textbf{NSX Controller} (cada una): 4 vCPU, 4 GB de memoria, 28 GB de almacenamiento.
    \item Tres VM de \textbf{vRealize Log Insight}, una en cada nodo: 8 vCPU, 16 GB de memoria, 1312 GB de almacenamiento.
\end{itemize}


%% VIRTUAL INF. DOMAIN
\begin{subsubsection}{Virtual Infrastructure Domain (VI) y Virtual Desktop Infrastructure Domain (VDI)}
\label{subsubsec:domainVI}
Este tipo de \textit{workload domain} se crea manualmente y bajo demanda desde el \textit{management domain} para dar servicio a las necesidades de cada usuario o para crear diferentes entornos con finalidades distintas. Su configuración de hardware y lógica se especifican durante su proceso de creación, permitiendo indicar la cantidad de hosts, cantidad de almacenamiento, políticas de rendimiento y disponibilidad, o configuración de la red disponibilidad, todo para satisfacer las necesidades de los usuarios que realizarán sus trabajos en él. El acceso a un \textit{VI domain} se realiza a través de vSphere Client donde su administrador puede gestionar todos los recursos asociados con ese \textit{workload domain}. Cada Virtual Infrastructure Domain cuenta con sus propios vCenter Server y un NSX Manager dedicados, que se ejecutan desde el \textit{management domain} de la infraestructura, y \textit{datastore} vSAN dedicado [Fig. \ref{fig:esquemaCFComponentes}]. La diferencia entre un \textit{virtual infrastructure domain} y \textit{virtual desktop infrastructure domain} es que el segundo incorpora el producto VMware Horizon View que, resumiendo, permite desplegar escritorios virtuales.\\
Cuando se \underline{despliega un \textit{virtual infrastructure domain} se crean y configuran} de forma automatizada por el componente SDDC Manager las siguientes máquinas virtuales (VM) de cada componente de VMware Cloud Foundation:
\begin{itemize}
    \item Una VM de \textbf{vCenter Server} en Management Domain: 8 vCPU, 24 GB de memoria, 500 GB de almacenamiento.
    \item Una VM de \textbf{NSX Manager} en Management Domain: 4 vCPU, 16 GB de memoria, 60 GB de almacenamiento.
    \item Tres VM de \textbf{NSX Controller} en el VI Domain creado (cada una):  4 vCPU, 4 GB de memoria, 28 GB de almacenamiento.
\end{itemize}

\end{subsubsection}
\begin{figure}[h!]
  \centering
  \includegraphics[width=1\textwidth]{imaxes/conceptosPrevios/WDomainStructure.JPG}
  \caption{Como se estructuran y conectan los componentes de VMware Cloud Foundation.}
  \label{fig:esquemaCFDominios}
\end{figure}
\begin{figure}[h!]
  \centering
  \includegraphics[width=1\textwidth]{imaxes/conceptosPrevios/workloadDomains.png}
  \caption{Componentes dedicados de cada \textit{workload domain}.}
  \label{fig:esquemaCFComponentes}
\end{figure}


\FloatBarrier

%&%%%%%%%%%%%%%%%%%%%%%%%%%%%%%%%%%%%%%%%%%%%%%%%%%%%%%%%%
%% ARQUITECTURA
\subsection{Arquitectura}
La arquitectura de VMware Cloud Foundation tiene dos posibles modelos de despliegue dependiendo del número de hosts sobre los que se despliega VMware Cloud Foundation.

%% ESTANDAR
\subsubsection{Modelo estándar}
Este modelo se utiliza cuando VMware Cloud Foundation se despliega en un entorno con siete o más hosts. Está formado por un \underline{management domain} que se despliega en uno de los hosts y que contiene todos los componentes de gestión de toda la infraestructura [Sec. \ref{subsubsec:domainManagement}], y por \underline{virtual infrastructure domains} que se crean bajo demanda sobre el resto hosts disponibles [Sec. \ref{subsubsec:domainVI}] y una vez creado se pueden ampliar o reducir sus capacidades según sea necesario. El máximo número de \underline{virtual infrastructure domains} que se pueden desplegar en una instacia de VMware Cloud Foundation es 15.

\begin{figure}[h!]
  \centering
  \includegraphics[width=0.75\textwidth]{imaxes/conceptosPrevios/arquitect_standarCF.png}
  \caption{Esquema del modelo de arquitectura estándar.}
  \label{fig:modelostandard}
\end{figure}

\begin{figure}[h!]
  \centering
  \includegraphics[width=0.85\textwidth]{imaxes/conceptosPrevios/standardArch.png}
  \caption{Estructura de los componentes en una arquitectura estándar.}
  \label{fig:standardarch}
\end{figure}
\FloatBarrier
%%%%%%%%%%%%%%%%%%%%%
%%  CONSOLIDADO
\subsubsection{Modelo consolidado}
Este modelo se despliega cuando el entorno esta formado por menos de siete nodos. En este modelo el flujo de trabajo que corresponde al \underline{virtual infrastructure domain} y al \textit{management domain} en el despliegue estándar, está colocado dentro del mismo \underline{workload domain}, es decir, solo existe un único \textit{workload domain} denominado \textit{management domain}. Este modelo se puede convertir en un modelo estándar creando un \textit{virtual infrastructure domain}. Los flujos de trabajo se mantienen aislados gracias a sus respectivos almacenes de recursos[Fig. \ref{fig:modeloconsolidated}].

\begin{figure}[h!]
  \centering
  \includegraphics[width=0.25\textwidth]{imaxes/conceptosPrevios/modelConsolidated.png}
  \caption{Esquema del modelo de arquitectura consolidado.}
  \label{fig:modeloconsolidated}
\end{figure}

\begin{figure}[h!]
  \centering
  \includegraphics[width=0.85\textwidth]{imaxes/conceptosPrevios/consolidatedArch.png}
  \caption{Estructura de los componentes de una arquitectura consolidado.}
  \label{fig:consolidatedArch}
\end{figure}
\FloatBarrier


%%%% DISEÑO ARQUI. FÍSICA %%%%%
\begin{subsection}{Diseño arquitectura física}

\subsection{Tipos de clusters, zonas de disponibilidad y regiones}
Un SDDC puede estar formado por múltiples clusters que pueden ser de diferentes tipos de clusters con diferentes propósitos. Un cluster puede ocupar uno o más \textit{racks} dependiendo del nivel de escalabilidad que se requiera. Según su función, cada \textit{workload domain} se puede colocar en un cluster diferente para gestionar la alta disponibilidad y el ciclo de vida según sus necesidades. Un cluster puede ser:
\begin{itemize}
    \item \textbf{Management Cluster}: Es aquel que contiene el \textit{management domain}.
    \item \textbf{Shared Edge y Compute Cluster}: contiene el \textit{virtual infrastructure domain} con las máquinas virtuales de los usuarios y, además, incorpora los servicios de NSX necesarios para comunicarse con redes externas y con otros \textit{workload domains}.
    \item \textbf{Compute Cluster}: solo contiene el \textit{virtual infrastructure domain} con las máquinas virtuales de los usuarios.
    \item \textbf{External Storage}: se centra en proveer almacenamiento de tipo NFS, iSCSI o Fiber Channel.
\end{itemize}

Un SDDC puede estar distribuído en una o más \underline{zonas de disponibilidad}. Estas son zonas aisladas que evitan la propagación de fallos de hosts individuales a través de toda la infraestructura para establecer mayor disponibilidad. A su vez, varias \underline{zonas de disponibilidad} están agrupadas en \underline{regiones}, entornos separados por grandes distancias para tener recuperación ante desastres.\\
*******************IMAGEN AV****************\\

\subsubsection{Propiedades de la red}
El \underline{transporte de la red} de un SDDC con VMware Cloud Foundation se puede implementar en la capa dos, configurando los dispositivos físicos para funcionar directamente con los componentes de VMware, o en la capa tres del modelo OSI para usar dispositivos de transporte lógico. Cada tipo de configuración tiene sus ventajas y desventajas, pero la configuración más común en este tipo de entornos es usar el transporte de red sobre la capa 3 da más libertad a la hora de gestionar los recursos físicos de red.\\


Como VMware Cloud Foundation abstrae la red física en una red virtual, la red física debe cumplir ciertos requisitos para que la red virtualizada sea robusta. Esta se debe mantener simple con configuraciones comunes en todos los switches, uso de VLANs y uso de enrutamiento dinámico, también debe ser escalable en cuanto a cantidad de hosts, ancho de banda, o posibles rutas para alcanzar un destino. Además, se debe tener en cuenta que cada tipo de tráfico tiene características diferentes, como por ejemplo el tráfico dedicado al almacenamiento a través de IP que suele usar mayor ancho de banda, por ello es necesario distinguir cada tipo de tráfico en la capa 2 con el protocolo CoS o en la capa 3 con el protocolo DSCP. Los elementos que se deben configurar son los siguientes:
\begin{itemize}
    \item \textbf{Top of Rack Physical Switches} (TOR): se trata de un switch que está colocado dentro de un rack y a él se conectan todos los hosts que se encuentran dentro de ese rack que a su vez los conecta con el resto de la infraestructura. Estos switches se deben configurar de forma redundante para ofrecer alta disponibilidad, los puertos que se conectan a los hosts deben estar configurados como puertos troncales de forma manual, proveer servicio DHCP a cada VLAN usada por los puertos \textit{vmkernel} de \textit{management} y VXLAN, y configurar los puertos para que acepten \textit{jumbo frames}. Además, se deben configurar todas las VLANs necesarias.
    \item \textbf{Conectividad entre Regiones}: 
    \item \textbf{Conectividad entre Zonas de dispobilidad}:
\end{itemize}

\subsubsection{Características de un host ESXi}
Los hosts ESXi que se desplieguen en un cluster deben tener características físicas idénticas para hacer la infraestructura más manejable,incluyendo la configuración de almacenamiento y red.
Dos de sus interfaces de red (NIC) conla misma velocidad deben estar conectadas a la VLAN \textit{trunk}. Esto junto con el componente VMware vSphere Distributed Switch permite que el tráfico se redirija de forma óptima y que las conexiones sean redundantes. Todas las conexiones deben terner una velocidad mínima de 10 Gbit.

\end{subsection}

%%%%%DISEÑO ARQ. VIRTUAL
\begin{subsection}{Diseño arquitectura virtual}
La infraestructura virtual de un SDDC está formada por dos regiones, cada una con su propio \textit{management domain} que incluye el \textit{management cluster}, y un \textit{virtual infrastructure domain} que contiene el \textit{shared cluster} y un \textit{compute cluster}.

\subsubsection{Características de la red virtual}
La red de esta infraestructura está virtualizada por VMware NSX for vSphere que utiliza los componentes vCenter Server, NSX Manager, NSX Controllers y NSX logical switch para establecer comunicaciones y aislar el tráfico de cada flujo de trabajo. El tráfico se divide en datos de flujos de trabajo donde se segrega el tráfico que pertenece a diferentes flujos, datos de control donde se agrupa el tráfico para la configuración de dispositivos virtualizados de routing, switching y firewalls, y datos de gestión donde se incluye el tráfico dedicado a la gestión de los recuros como la gestión de máquinas virtuales.\\

Estos componentes incluyen diversos servicios que suponen una abstracción lógica de dispositivos físicos de red y que VMware Cloud Foundation utiliza para formar su infraestructura virtual. Estos servicios son los siguientes:
\begin{itemize}
    \item \textbf{Logical Switches}: permite crear segmentos abstractos que representan dominios de broadcasting donde se colocan determinadas máquinas virtuales. Su tráfico tiene asignado una única VLAN y están mapeados en todos los hosts lo cual simplifica la movilidad de las máquinas virtuales entre los hosts.
    \item \textbf{Universal Distributed Logical Router}: realiza las funciones de enrutamiento entre máquinas virtuales. Son controlados por una máquina virtual y utilizan protocolos de enrutamiento dinámico como BGP y OSPF.
    \item \textbf{Designated Instance}: host ESXi encargado de resolver las solicitudes del protocolo ARP. Es elegido por NSX Controller y se elige un host por cada VLAN existente.
    \item \textit{Edge Services Gateway}: es el encargado de proveer conectividad a través de la infraestructura física para que los componentes se puedan conectar a redes externas.
    \item \textbf{Logical Firewall}: provee mecanismos de seguridad que se asemejan a las funciones de los firewalls físicos pero con la ventaja de la virtualización, lo cual hace que su configuración sea más flexible.
    \item \textbf{Logical Load Balancer}: distribuye el tráfico entre los servidores para que el uso de recursos sea el óptimo.
\end{itemize}

\end{subsection}
\begin{subsection}{Gestión de las operaciones de la arquitectura}
En este apartado se define como se gestionan en VMware Cloud Foundation las tareas de administración de todas las partes de la infraestructura. Estas tareas se agrupan en la gestión del ciclo de vida y la recopilación de información sobre el estado de cada componente existente.

\subsubsection{Gestión del ciclo de vida}
Elementos que se encargan de administrar el ciclo de vida de los componentes:
\begin{itemize}
    \item \textbf{vSphere Update Manager}: pensado para la adminstración del entorno vSphere. Permite administrar las actualizaciones que reciben los hosts ESXi, instalar software de terceros en los hosts ESXi y actualizar el hardware virtual de las máquinas virtuales. Existe una instancia de este producto por cada vCenter Server. En su arquitectura existe otro elemento llamado \textbf{Update Manager Download Service} (UMDS) y representa un entorno con acceso a la red pública donde se descargan los archivos requeridos por vSphere Update Manager, ya que este se encuentra en un entorno aislado. Esto incrementa la seguridad y permite compartir estos archivos entre distintas instancias de vSphere Update Manager.
    
    \item \textbf{vRealize Suite Licfecycle Manager}: elemento encargado de administrar el ciclo de vida de todos los productos de vRealize de forma automatizada, esto incluye el despliegue y actualización de vRealize Operations Manager, vRealize Log Insgiht y vRealize Automation. Este se conecta a un servicio externo para descargar los archivos que se necesiten. La instancia de eeste producto reside en la la Zona de Disponibilidad 1 dentro de la Region A.
    
\end{itemize}

\subsubsection{Monitorización}
En VMware Cloud Foundation la monitorización de la infraestructura es realizada por vRealize Log Insight que provee gestión y análisis de los logs de la infraestructura. Este componente conecta con todos los prodcutos de VMware y recoge información sobre las alarmas, tareas y eventos que tienen lugar usando el protocolo syslog.\\

Su modelo de despliegue está formado por un nodo \textit{Master} y dos nodos \textit{Worker}, además, se puede habilitar alta disponibilidad su balanceador de carga \textit{Integrated Load Balancer} (ILB), este último es quien recibe las peticiones de logs de los clientes y luego las retransmite a los nodos \textit{worker} y \textit{master} para obtener la información. El producto vRealize Log Insight se despliega en la Zona de Disponibilidad 1 dentro de la Region A.





\end{subsection}


%%% CLOUD BUILDER
\subsection{Cloud Foundation Builder VM Support}
\label{subsec:cloudBuilder}
El despliegue de la plataforma VMware Cloud Foundation se realiza través de una máquina virtual llamada Cloud Foundation Builder. Esta máquina recoge la configuración que se indica en la hoja de parámetros, los valida, , despliega y configura el \underline{Management Domain}. Al final del proceso, transfiere el inventario y el control del sistema al componente SDDC Manager y esta máquina virtual puede ser eliminada.\\

\subsection{Red, almacenamiento y servicios necesarios para el despliegue}

\subsubsection{Servicios internos}
\label{subsubsec:servInterno}
Durante el despliegue Cloud Foundation, hay servicios y puertos deben estar habilitados en cada nodo ESXi para Cloud Foundation Builder (Fig. \ref{fig:puertosCB}) y SDDC Manager (Fig. \ref{fig:puertosSDDC}) puedan acceder a todos los componentes.

%%%%%%%%%%%%%%%%%%%%%%%%%%%%%%%%%%%%%%%%%%%%%%%
\iffalse
\begin{figure}[h!]
  \centering
  \includegraphics[width=0.7\textwidth]{imaxes/conceptosPrevios/puertosentradaCB.png}
  \includegraphics[width=0.7\textwidth]{imaxes/conceptosPrevios/puertossalidaCB.png}
  \caption{Servicios y puertos de entrada y salida habilitados para Cloud Foundation Builder.}
  \label{fig:puertosCB}
\end{figure}

\begin{figure}[h!]
  \centering
  \includegraphics[width=0.7\textwidth]{imaxes/conceptosPrevios/puertosentradaSDDC.png}
  \includegraphics[width=0.7\textwidth]{imaxes/conceptosPrevios/puertossalidaSDDC.png}
  \caption{Servicios y puertos de entrada y salida habilitados para SDDC Manager.}
  \label{fig:puertosSDDC}
\end{figure}
\fi
%%%%%%%%%%%%%%%%%%%%%%%%%%%%%%%%%%%%%%%%%%%%%%%%%%%

\begin{figure}[h!]
  \centering
  \includegraphics[width=1\textwidth]{imaxes/conceptosPrevios/firewall_CloudBuilder.png}
  \caption{Servicios y puertos de entrada y salida habilitados para Cloud Foundation Builder.}
  \label{fig:puertosCB}
\end{figure}
\begin{figure}[h!]
  \centering
  \includegraphics[width=1\textwidth]{imaxes/conceptosPrevios/firewall_SDDC.png}
  \caption{Servicios y puertos de entrada y salida habilitados para SDDC Manager.}
  \label{fig:puertosSDDC}
\end{figure}

\FloatBarrier

\subsubsection{Servicios externos}
\label{subsubsec:servExterno}
En este apartado se describen aquellos servicios necesarios para el despliegue de VMware Cloud Foundation. Estos servicios deben estar configurados antes del despliegue de VMware Cloud Foundation para que todos los componentes se puedan comunicar antes, durante y después del despliegue. Se pueden configurar en servidores instalados en máquinas virtuales dentro del entorno y estas deben tener acceso a la red \textit{VM Network}.\\
\begin{itemize}
    
    \item \textbf{Dynamic Host Configuration Protocol (DHCP)}: Permite configurar automáticamente cada puerto VMkernel de un nodo con IPv4. 
    
    \item \textbf{Domain Name Server (DNS)}: Un servidor DNS debe estar disponible para todos los componentes desde el momento de despliegue. También se debe especificar el nombre del dominio. Con esto los componentes pueden obtener tanto el nombre como la dirección IP de un elemento en la red. Los componentes y servicios de la plataforma que usan este servicio son NTP, Platform Services Controller, instancias de vCenter Server, instancias de NSX Manager y vRealize Log Insight.
    
    \item \textbf{Network Time Protocol (NTP)}: Permite la sincronizar el tiempo en todos los nodos de la infraestructura. Durante el despliegue se debe especificar la dirección IP de al menos un servidor NTP.
    
    \item \textbf{Router}: Se requiere un router con el protocolo de enrutamiento \textit{\textbf{Border Gateway Protocol}} (BGP) habilitado y configurado para compartir todas las rutas por toda la red del entorno.
    Este router debe tener configuradas cuatro interfaces para las distintas redes que se generan en la arquitectura de VMware Cloud Foundation.
    
\end{itemize}

Se puede encontrar más información sobre otros servicios opcionales en el siguiente \href{https://docs.vmware.com/en/VMware-Cloud-Foundation/3.9/com.vmware.vcf.planprep.doc_39/GUID-F022BD3C-F11C-4EE6-83EA-ABE016E7A9B9.htm}{enlace}.
\FloatBarrier

%%%%% RED 
\subsubsection{Red física}
\label{subsubsec:redFisica}
La red física debe admitir las siguientes características:
\begin{itemize}
    \item \textbf{VLAN}: etiquetado de redes VLAN (802.1Q)
    \item \textbf{Jumbo Frames}: MTU mínimo igual a 1600, aunque se recomienda que sea igual a 9000.
\end{itemize}

\subsubsection{Red lógica}
\label{subsubsec:redLogicaCF}
\iffalse
Antes del despliegue es necesario especificar varias redes que más tarde Cloud Foundation usará para automatizar la configuración de puertos \ref{Word:vmkernel} cuando se añade un nuevo host al entorno o se crea un VI Domain. Estas redes son  una dedicada al servicio \underline{vSAN}, otra dedicada a \underline{vMotion}, ota de dedicada a la gestión de los componentes y otra dedicada al tráfico de las máquinas virtuales. Los datos a especificar son la etiqueta VLAN, MTU, IP de la red, máscara, gateway y el rango de direcciones IP. Una vez creadas solo se puede modificar el rango de direcciones IP.\\
Es importante utilizar VLANs e IPs ya que es en lo que se basa Cloud Foundation para aislar cada tipo de tráfico de la infraestructura. La cantidad de subredes necesarias depende del número de Workload Domains que se creen, número de clusters y otros componentes opcionales.
\fi

Antes de iniciar el despliegue de VMware Cloud Foundation, el entorno sobre el que se instalará la plataforma debe tener configurado un vSphere Distributed Switch, una llamada \textit{Management Network} que se dedicará a la transmisión del tráfico generado entre los distintos componentes del entorno, y otra llamada \textit{VM Network} que se dedicará al tráfico generado por las máquinas virtuales que se desplieguen sobre la plataforma. La red \textit{Management Network} se puede configurar con etiquetado VLAN, en ese caso la red \textit{VM Network} debe tener la misma etiqueta VLAN. Los vSwitches creados deben tener un MTU igual a 1600 como mínimo. Además, es necesario conocer de antemano las direcciones IP que se asginarán a cada componente de la infraestructura durante el despliegue de la plataforma.


%%% ARQUITECTURA DE LA RED CLOUD FOUNDATION
\subsubsection{Arquitectura de la red}
En este apartado se describe la red que forma VMware Cloud Foundation cuando despliega las máquinas virtuales de su principales componentes y como estructura sus conexiones con los elementos ya existentes.\\
Como ya hemos visto, los diferentes servicios de esta plataforma se implementan en forma de máquinas virtuales que se conectan entre si a través de una red creada en base a los elementos de red virtualizados de VMware como \textit{Distributed Switches}, \textit{vSwitches}, \textit{vmkernels} y \textit{vmportgroups}. Esta red se denomina \textit{Management Network} y a través de esta todos 



\subsubsection{Almacenamiento}
Según la documentación de vSAN, cada host ESXi del entorno debe tener como mínimo un grupo de discos. Un nodo puede tener hasta cinco grupos de discos, dentro de los cuales debe haber al menos un disco de caché y un disco de capacidad. Cada grupo de discos puede contener hasta siete discos de capacidad.
En cuanto al tipo de disco que se debe utilizar, los discos de caché deben ser SSD y los discos de capacidad pueden ser SSD o HDD, según el modelo que se quiera implementar (All-Flash o híbrido).

%%%%%%%%%%%%%%%%%%%%%%%%%%%%%%%%%%%%%%%


\begin{section}{Requisitos}
En este apartado se describe aquello que debe cumplir la infraestructura física para que los componentes de VMware Cloud Foundation funcionen de forma adecuada y que la configuración y mantenimiento de los componentes físicos sea simple a la hora de expandir el entorno.

% Teniendo en cuenta las capacidades físicas de la infraestructura, se ha elegido el modelo consolidado para el despliegue de VMware Cloud Foundation sobre la infraestructura.     La principal razón por las que se escoge este modelo es por el número de hosts ESXi.
% En los siguientes apartados se describen la arquitectura que se genera y la infraestructura requerida en cada capa.
%%%%%%%%%%%%%%%%%%%%%%%%%%%%%%%%%%%%%%
\begin{subsection}{Cómputo}
\begin{subsubsection}{Hosts ESXi}
    Para realizar el despliegue del primer WD (el Management Domain) se requieren al menos cuatro\footnote{Se reserva la cuarta parte de los recursos para que el \textit{management domain} permanezca activo en caso de caída de alguno de los hosts.} hosts ESXi con al menos un 128 GB de memoria RAM y un disco de arranque de 32 GB cada uno\footnote{Según la configuración establecida para el producto vSAN ReadyNode \cite{host-requirements}}. Para cada WD adicional solo se requiere un mínimo de tres hosts y la cantidad de memoria RAM depende de la finalidad del WD, por lo tanto para implementar el modelo de arquitectura estándar se requieren al menos siete hosts ESXi. Cada uno de los hosts debe tener al menos dos interfaces de red físicas (NIC) que soporten al menos 10 Gbit/seg de velocidad.
    
\end{subsubsection}
\end{subsection}
%%%%%%%%%%%%%%%%%%%%%%%%%%%%%%%%%%%%%%%
\begin{subsection}{Almacenamiento}
    En el Management Domain es obligatorio el uso de un \textit{datastore} de VMware vSAN, este necesita al menos tres hosts con recursos de almacenamiento para funcionar\footnote{VMware vSAN requiere un mínimo de tres hosts mientras que el Management Domain requiere un mínimo de cuatro hosts.}. Se debe aplicar la configuración All-Flash con discos SSD. Basándose en los perfiles que VMware establece para su producto vSAN Ready Node\cite{host-requirements}, cada host debe tener al menos un grupo de dos discos donde la cantidad de almacenamiento para la capa de capacidad debe ser de 4 TB y para la capa de caché de 200 GB. VMware vSAN soporta discos con adaptadores SAS, SATA o SCSI y estos pueden estar configurados en modo \textit{pass-through} o RAID 0. En cuanto a esto, es preferible que los discos se configuren en modo \textit{pass-through} ya que permite que estos se puedan gestionar de forma independiente, sin tener que apagar los hosts cuando sea necesario retirar o añadir discos.
    Para WD adicionales se puede utilizar almacenamiento NFS en lugar de un \textit{datastore} de VMware vSAN, aunque la solución de VMware aporta mayor rendimiento y simplifica la administración de esta parte de la infraestructura física.
    % los discos de caché debe ser al menos un 10\% del tamaño total de los discos de capacidad,  y  \footnote{La capacidad de los discos descrita es la necesaria para desplegar el \textit{management domain} y un \textit{workload domain} adicional.}
\end{subsection}
%%%%%%%%%%%%%%%%%%%%%%%%%%%%%%%%%%%%%%%
\begin{subsection}{Red}
 \begin{subsubsection}{Switch Top Of Rack}
     Los hosts están colocados en racks, en un rack puede haber hosts pertenecientes a distintos WD. Para favorecer la alta disponibilidad y tolerancia a fallos de la infraestructura física, un rack debe tener dos switches Top Of Rack (TOR) y cada host debe tener una interfaz conectada a cada uno de ellos, una capa superior de switches conecta los switches TOR entre sí. Todas las conexiones de la red física deben soportar \textit{Jumbo frames} (MTU hasta 9000 Bytes), etiquetado \textit{Quality of Service} (QoS) de tráfico y el etiquetado VLAN, todo para dar soporte a las subredes del SDDC\footnote{Para el Management Domain, las subredes cuya VLAN debe ser configurada en la red física son la subred Management para tareas de administración, la subred dedicada a VMware vSAN, la subred dedicada a overlay y la subred dedicada a VMware vSphere vMotion.}. Todas las conexiones físicas deben tener, al menos, 10 Gbit/seg de velocidad.
    % Todos los switches TOR deben tener al menos dos interfaces 10 Gbit Ethernet como mínimo. 
 \end{subsubsection}
 \begin{subsubsection}{Servicios}
     En el SDDC se deben habilitar varios servicios requeridos por los componentes de VMware Cloud Foundation para su correcto funcionamiento.
     \begin{itemize}
         \item DNS: servidor de nombres para resolver todas las direcciones IP y \textit{hostnames} de los componentes del SDDC.
         \item DHCP: servidor para asignar de forma automática una dirección IP a los hosts que forman el SDDC.
         \item NTP: servidor de tiempo para sincronizar la hora de todos los componentes del SDDC.
         \item Router: se requiere para enrutar el tráfico que emiten todas las instancias del SDDC y para dar acceso a redes externas. Debe soportar enrutamiento dinámico BGP y debe tener configuradas las subredes y VLANS que se vayan a utilizar en la infraestructura.
         \item SMTP: servidor de correo utilizado por el componente VMware vRealize Automation.
         \item Active Directory: servidor de usuarios y grupos de usuarios que el SDDC utiliza como fuente para configurar el acceso a cada parte de la infraestructura virtual.
         \item Certificate Authority: se debe configurar una autoridad certificadora que genere certificados firmados para cada uno de los componentes de VMware Cloud Foundation. Permite establecer conexiones seguras cuando se accede a los componentes.
     \end{itemize}
 \end{subsubsection}
\end{subsection}


\end{section}
%%%%%%%%%%%%%%%%%%%%%%%%%%%%%%%%%%%%%%%
    %%%% DISEÑO ARQUI. FÍSICA %%%%%
% \begin{subsection}{Arquitectura e Infraestructura Físicas \cite{CFfisInfraestuctura}}
%     En este apartado se describen las principales características que tiene el entorno físico de un SDDC construído con VMware Cloud Foundation.


% \begin{subsubsection}{Red física}
% La topología de red en la capa física del SDDC de VMware Cloud Foundation se puede implementar mediante servicios de \underline{transporte} en la capa 2 o en la capa 3. El \underline{diseño en la capa 2} implica que la topología de la red incluya los dispositivos de capa 2 (\textit{Top of Rack Switches}) y los dispositivos de la capa 3 (routers, switches) [Fig. \ref{fig:transportlayer2}], por lo tanto las VLANs que se definan se deben implementar en la capa 2 y en la capa 3. Esto puede provocar problemas al aumentar el tamaño de la red ya que el número de VLANs disponible es más limitado, y problemas de compatibilidad ya que es posible que los dispositivos físicos tengan que ser del mismo proveedor. El \underline{diseño en la capa 3} implica que la topología de la red solo incluye a los dispositivos de capa 3 [Fig. \ref{fig:transportlayer3}]. Esto permite limitar la definición de VLANs a esa capa y el uso de enrutamiento dinámico con protocolos OSPF o BGP entre la capa 2 y 3. Así se consigue una mayor libertad a la hora de seleccionar los dispositivos físicos de red y que su configuración es más sencilla.
% \begin{figure}[h!]
%   \centering
%   \includegraphics[width=0.3\textwidth]{imaxes/conceptosPrevios/transportlayer2.png}
%   \caption{Límite de las capas 2 y 3 cuando la topología se implementa con dispositivos de capa 2.}
%   \label{fig:transportlayer2}
% \end{figure}
% \FloatBarrier
% \begin{figure}[h!]
%   \centering
%   \includegraphics[width=0.3\textwidth]{imaxes/conceptosPrevios/transportNetLayer3.png}
%   \caption{Límite de las capas 2 y 3 cuando la topología se implementa con dispositivos de capa 3.}
%   \label{fig:transportlayer3}
% \end{figure}
% \FloatBarrier



% Como VMware Cloud Foundation abstrae la red física en una red virtual, la red física debe cumplir ciertos requisitos para que la red virtualizada sea robusta. Esta se debe mantener simple con configuraciones comunes en todos los switches, uso de VLANs y uso de enrutamiento dinámico, también debe ser escalable en cuanto a cantidad de hosts, ancho de banda y cantidad de rutas redundantes. Además, se debe tener en cuenta que cada tipo de tráfico tiene características diferentes, como por ejemplo el tráfico dedicado al almacenamiento a través de IP que suele usar mayor ancho de banda, por ello es necesario distinguir cada tipo de tráfico con protocolos \textit{Quality of Service} (QoS). El marcado de cada tipo de tráfico se realiza en el hipervisor ESXi a través de un vSphere Distributed Switch que soporta QoS tanto en la capa 2 como en la capa 3. En la capa 2 se utiliza  un campo de tres bits llamado \textit{Class of Service} que representa la prioridad del \textit{frame} con un valor de cero a siete, presente en la cabecera Ethernet cuando se utiliza etiquetado VLAN, mientras que en la capa 3 se utiliza un campo de 6 bits en la cabecera IP llamado \textit{Differentiated Services Code Point}, perteneciente al protocolo \textit{DiffServ}, para clasificar cada paquete. Los \underline{principales componentes que se deben configurar} para dar conectividad entre los servidores son los siguientes:
% \begin{itemize}
%     \item \textbf{Top of Rack Physical Switches} (TOR): es un switch al que se conectan los hosts de un rack para tener conectividad con el resto de la infraestructura. Se recomienda que un host esté conectado a dos switches TOR y que estos se configuren de forma redundante para proveer alta disponibilidad y tolerancia a fallos de alguna de las conexiones. Cada switch TOR se conecta a otro par de switches que establece conexión entre todos los racks.
    
%     Los puertos del switch TOR que se conectan a los hosts deben estar configurados como puertos troncales de VLAN para que acepte todas las VLANs usadas por el host, se debe proveer servicio DHCP a cada VLAN usada y configurar los puertos para que acepten \textit{jumbo frames}. El marcado QoS del tráfico que realiza cada host ESXi debe ser aceptado y no puede ser modificado una vez abandona el host. 
%     % \iffalse Además, se deben configurar todas las VLANs y subredes que se utilizarán en la infraestructura de VMware Cloud Foundation.\fi  
    
%     \underline{Otros protocolos que se deben configurar} en los puertos que se conectan con los hosts son:
%     \begin{itemize}
%         \item \emph{Spanning Tree Protocol} (STP): protocolo que se encarga de gestionar las rutas de la red que son redundantes.
%         \item \emph{Trunking}: configurar cada enlace troncal con las VLANs que van a transmitir tráfico a través de él. Se debe establecer como VLAN nativa, aquella utilizada para transmitir el tráfico que no tiene etiqueta, VLAN de la red \textit{management}.
%         \item \emph{MTU}: configurar el MTU de cada VLAN para el transporte de paquetes \textit{jumbo frames}. Este valor será el que se use para configurar los hosts ESXi. Se recomienda establecerlo en 9000 bytes.
%         \item \emph{Multicast}: configurar el protocolo IGMP en cada switch TOR como enrutador (busca activamente que VLANs pertenecen a un grupo Multicast) y cada VLAN como miembros de IGMP (los hosts que forman parte del grupo indican su pertenencia a un grupo multicast de forma activa).
%     \end{itemize}
    

    
%     % \iffalse
%     % \item \textbf{Conectividad entre Regiones}: 
%     % \item \textbf{Conectividad entre Zonas de dispobilidad}:
%     % \fi
% \end{itemize}

% Los siguientes servicios usados por los componentes de VMware Cloud Foundation se deben configurar sobre la red física de la infraestructura para el correcto funcionamiento del SDDC\cite{CFexternalServices}:
% \begin{itemize}
%     \item \textbf{Servidor DNS}: se utiliza para obtener los nombres y direcciones de todas las máquinas virtuales que se creen, tanto en sentido \textit{fordward} (obtener una dirección IP a partir de un nombre) como en sentido \textit{reverse} (obtener un nombre a partir de una dirección IP). Además, este servicio debe ser configurado antes de realizar el despliegue de VMware Cloud Foundation. Este servicio es utilizado por el componente Platform Services Controller, vCenter Server, NSX Manager y vRealize Log Insight.
    
%     \item \textbf{Servidor DHCP}: permite asignar direcciones IP de forma dinámica a los puertos \textit{vmkernel} de cada host ESXi. Este debe ser accesible desde cada VXLAN de VMware NSX y es necesario establecer previamente las redes que se van a usar en VMware Cloud Foundation. Este servicio debe estar disponible antes de comenzar el despliegue del SDDC ya que es necesaria la asignación dinámica de IPs.
    
%     \item \textbf{Servidor NTP}: requerido por todos los componentes de VMware Cloud Foundation para mantener sus horas sincronizadas. Este servicio debe estar disponible en la infraestructura y configurado en cada host ESXi antes del despliegue de VMware Cloud Foundation, y debe ser alcanzable desde la red de \textit{management} y de vRealize. La derencia de tiempo entre los componentes de la infraestructura no debe ser mayor de cinco minutos.
    
%     \item \textbf{Router}: debe existir enrutamiento dinámico en la red desde la capa 3. Es requerido por NSX para establecer comunicación con los ESG. Este servicio debe estar configurado antes del comenzar enl despliegue de VMware Cloud Foundation. 
% \end{itemize}

% \end{subsubsection}


% \begin{subsubsection}{Host ESXi\cite{WDminRequierements}}
% Los hosts ESXi que se desplieguen en un cluster deben tener características físicas idénticas para hacer la infraestructura más manejable,  incluyendo la configuración de almacenamiento y red. Para desplegar VMware Cloud Foundation se requiere:
% \begin{itemize}
%     \item  Dos interfaces de red (NIC) de la misma velocidad que deben estar conectadas a la VLAN troncal de dos switches TOR. Configurando \textit{NIC teaming} en VMware Sphere Distributed Switch se consigue que el tráfico se distribuya por las interfaces de red disponibles de forma óptima y que exista tolerancia a fallos.
%     \item Todas las conexiones físicas del host deben tener al menos una velocidad igual a 10 Gbit.
%     \item Cada host debe tener al menos 192 GB de memoria RAM, de esa cantidad, 176 GB de memoria RAM son requeridos por las máquinas virtuales que gestionan el SDDC.
%     \item Un disco de arranque con un tamaño mínimo de 16 GB.
% \end{itemize}

% \end{subsubsection}


% \begin{subsubsection}{Almacenamiento físico}
% VMware Cloud Foundation utiliza VMware vSAN para proveer el almacenamiento de un SDDC. Para desplegar VMware Cloud Foundation, VMware vSAN requiere las siguientes características:
% \begin{itemize}
%     \item Mínimo de tres hosts con recursos de almacenamiento.
%     \item Determinar qué configuración de vSAN se va a utilizar, \textit{All-Flash} o \textit{Hybrid}. Se recomienda la solución \textit{All-Flash} ya que ofrece mayor rendimiento.
%     \item Para cada host con recursos de almacenamiento se debe cumplir que el disco de caché tenga un 10\% de la capacidad del almacenamiento persistente del grupo de discos, tener un mínimo de dos discos en la capa de capacidad, un controlador RAID y configurar habilitar vSphere High Availability  para apagar las máquinas virtuales de un host cuando este se encuentre aislado. El controlador RAID debe tener la característica \textit{pass-through} la cual permite que VMware vSAN muestre como discos individuales cada disco duro de un grupo de discos, esto facilita la gestión de cada disco y que se puedan realizar sustituciones sin detener el servicio.
%     \item La capacidad mínima de almacenamiento disponible para el modelo consolidado es de 800 GB. 
% \end{itemize}

% \end{subsubsection}

% \end{subsection}

\begin{section}{Despliegue de VMware Cloud Foundation}
En esta sección se describe el entorno y los procedimientos llevados a cabo para desplegar VMware Cloud Foundation sobre la infraestructura.
\begin{subsection}{Prueba de concepto}
Para poder realizar la instalación sin afectar al funcionamiento actual del CPD del CITIC, se habilita un entorno aislado. Con esto se evita la aparición de fallos que pueden ser críticos para la infraestructura y a la vez permite explorar esta nueva plataforma sin tener en cuenta los riesgos de hacerlo sobre la infraestructura real.\\
 ************DESCRIPCION DEL ENTORNO (INFRAESTRUCTURA)*********\\
\end{subsection}
\begin{subsection}{Preparación del entorno}
En esta sección se describe la arquitectura y configuración de todos los componentes de la infraestructura del entorno donde se va a realizar el despliegue. Las opciones de configuración se establecen de acuerdo con lo descrito en los apartados anteriores sobre el diseño de la infraestructura y arquitectura de VMware Cloud Foundation.
\\


*********SERVICIOS QUE SE CREAN (CONFIG DNS,DHCP, ROUTER...)***********\\

\subsubsection{VMware Cloud Foundation Builder}
El despliegue de la plataforma VMware Cloud Foundation se realiza través de una \textit{appliance} llamada VMware Cloud Foundation Builder. Esto es una máquina virtual que se instala en el entorno desde una archivo con formato \textit{.ova}, proporcionado por VMware, y que contiene el instalador de VMware Cloud Foundation. Este instalador se encarga de recibir los parámetros de configuración de la infraestructura donde se va a hacer el despliegue, los valida y a continuación despliega el \textit{management domain} del SDDC con todos sus componentes. Cuando termina el proceso, transfiere el inventario y el control del sistema al componente SDDC Manager y VMware Cloud Foundation Builder puede ser eliminado.\\
Esta máquina virtual tiene los siguientes requisitos:
\begin{itemize}
    \item \textbf{CPU}: 4 vCPU.
    \item \textbf{Memoria RAM}: 4 GB.
    \item \textbf{Almacenamiento}: 350 GB.
    \item Acceso al servidor DNS y NTP para validar la configuración de VMware Cloud Foundation.
    \item Acceso a la red \textit{management} para comunicarse con todos los hosts ESXi.
\end{itemize}


*********CLOUD BUILDER*************\\

*********HOJA DE PARÁMETROS********\\
\end{subsection}
\end{section}
%

\section{Despliegue del servicio}
En este apartado se expone el proceso de despliegue de VMware Cloud Foundation sobre el entorno creado específicamente para esto. Se muestran las configuraciones aplicadas y los cambios realizados sobre la infraestructura.

\subsection{Cumplimiento de requisitos}

\subsubsection{Almacenamiento}
Según la documentación de de VMware, el entorno debe tener una \ref{Word:capacidad} de almacenamiento mínima determinada para poder instalar los componentes de Cloud Foundation si se va a crear como mínimo un Management Domain y un VI Domain. La capacidad inicial del entorno es de 2,34TB de almacenamiento, 96 GBde memoria y 4 nodos ESXi, lo cual no es suficiente para cumplir con este requisito.
Además, Cloud Foundation requiere un disco de arranque en cada nodo de 16GB por lo que es necesario aumentar el tamaño de este disco de 8GB a 16GB.


\subsubsection{Redes}
Antes del despliegue es necesario especificar cuatro redes, una dedicada a vSAN, otra para vMotion, otra para la gestión de los componentes y una cuarta red para el tráfico de las máquinas virtuales (Sección \ref{subsubsec:redLogicaCF}). Inicialmente solo falta una red para habilitar el servicio de vMotion, el resto de redes necesarias ya están creadas y configuradas (Sección \ref{sec:redPrueba}). Esta nueva red también debe tener acceso al resto de componentes por lo que hay que añadir un nuevo adaptador de red (NIC) en el servidor externo. Además, también hay que añadir las nuevas rutas entre redes para que el servidor pueda realizar el enrutamiento del tráfico correctamente.\\
******FOTO NUEVOS ADAPTADORES DE RED*****\\
******FOTO NUEVA TABLA DE ENRUTAMIENTO*****\\

En cuanto a la red física, está configurada para aceptar el MTU igual a 1500 lo cual no cumple con el requisito mínimo (Sección \ref{subsubsec:redFisica}). El tráfico etiquetado si que está habilitado en la red física. Como se está trabajando sobre un entorno físico virtualizado entonces se puede modificar de forma sencilla. Además, cada nodo tiene cuatro \ref{Word:vmnic} por lo que cumple con el mínimo de dos NIC por nodo (Sección \ref{subsubsec:domainVI}).

\subsubsection{Servicios internos y externos}
El tráfico de los servicios y puertos especificados [Sec. \ref{subsubsec:servInterno}] deben estar habilitados en el firewall interno de cada nodo y su tráfico está permitido a través de la red. Inicialmente, el tráfico a cada puerto está habilitado desde cualquier dirección IP.\\
Inicialmente, los servicios externos (DNS, DHCP y NTP) están habilitados en un servidor alcanzable por todas las redes configuradas.
\iffalse
Como es necesario habilitar una nueva red para el componente vMotion entonces hay que añadir un nuevo adaptador de red en el servidor para que esta red pueda acceder a los servicios y comunicarse con el resto de componentes.\\
\fi
Durante el proceso de despliegue de Cloud Foundation se instalarán nuevos componentes por lo que también es necesario configurar el DHCP añadiendo más rangos de direcciones IP, y el DNS para que todos los componentes y máquinas virtuales del entorno sean alcanzables desde todas las redes.\\
*****FOTO DE LO QUE SE HA AÑADIDO EN DHCP Y DNS *****\\

\iffalse
Los servicios externos (Sección \ref{subsubsec:servExterno}) se deben habilitar en un servidor alcanzable por cada nodo por lo que al añadir nuevas redes al entorno es necesario incluir más adaptadores de red en la máquina virtual con Windows Server para darles acceso al DNS, al DHCP y a NTP.
\fi
***RESUMEN DE COMO QUEDA ANTES DE EMPEZAR LA INSTALACIÓN CON LOS CAMBIOS REALIZADOS*******


\subsection{Despliegue VMware Cloud Foundation Builder}
Para instalar VMware Cloud Foundation, antes es necesario instalar VMware Cloud Foundation Builder (Sección \ref{subsec:cloudBuilder}). Esta máquina virtual se despliega desde vSphere Client a partir de un archivo \textit{ova} proporcionado por VMware, en este caso la versión que se instalará es \underline{2.2.0.0-14866160}.
Una vez deplegado Cloud Builder y antes de iniciar la instalación de Cloud Foundation, es necesario adjuntar una hoja de parámetros con los datos de configuración que se deben aplicar, tales como el DNS, redes, y credenciales de los componentes que se van a crear posteriormente.

\subsubsection{Configuración de la máquina virtual}

\subsubsection{Hoja de parámetros}
Esta hoja es un archivo \textit{xlsx} de Excel donde se indica la configuración sobre componentes que ya existen en el entorno y los que se van a desplegar. Estos datos permiten a Cloud Builder y SDDC Manager automatizar el proceso de instalación y configuración inicial de los componentes para generar el Management Domain. Estos datos se dividen en varias categorías:
\begin{itemize}
    \item \textbf{\textit{Prerequisite Checklist}}: muestra una lista de comprobaciones a realizar sobre la configuración inicial del entorno antes de iniciar el despliegue.
    \item \textbf{\textit{Management Workloads}}: permite introducir los números de las licencias necesarias para utilizar todos los componentes de VMware Cloud Foundation y una estimación de los recursos necesarios para el Management Domain.
    \item \textbf{\textit{Users and Groups}}: permite indicar las credenciales para cada usuario de cada componente de VMware Cloud Foundation. Estas credenciales deben cumplir un unos requisitos mínimos.
    \item \textbf{\textit{Hosts and Networks}}: en esta categoría se indican las direcciones IP de las redes requeridas y las direcciones IP de cada host ESXi. También se pueden indicar los fingerprint de SSH y thumprints de SSL de cada host ESXi para que estos sean validados antes de conectarse a ellos durante la instalación (esto es opcional).
    \item \textbf{\textit{Deploy Parameters}}\footnote{Todos los nombres y direcciones IP indicados deben estar configuradas en el DNS para que se puedan resolver a través de ese servicio.}: Aquí se indican los detalles de configuración que se van a desplegar. Existen varios apartados\cite{deploysheet}:
        \begin{itemize}
            \item \emph{Existing Infrastructure Details}: se indican los datos de configuración de servicios configurados inicialmente en el entorno. Estos son DNS, NTP, Single Sign-On y el almacén de vSAN.
            
            \item \emph{vSphere Infrastructure}: se indican los nombres y direcciones IP de las máquinas virtuales de los componentes vCenter Server y Platform Services Controller. 
            
            \item \emph{NSX}: se debe indicar las direcciones IP y los nombres de las máquinas virtuales del componente NSX que se van a desplegar.
            
            \item \emph{vRealize Log Insight}: se indican las direcciones IP y nombres de las máquinas virtuales del \textit{Load Balancer} y de los tres nodos de vRealize Log Insight (un nodo \textit{Master} y dos nodos \textit{Workers}).
            
            \item \emph{SDDC Manager}: se indica el nombre de la máquina virtual, la dirección IP y su máscara de red para el componente SDDC Manager.
        \end{itemize}
    \end{itemize}
Esta hoja de parámetros está adjuntada a este documento.\\
*****METER HOJA EXCEL*****



\subsection{Instalación}


\subsection{Configuración}

%\include{contido/...}
 %\include{contido/conclusions} ------Desmarcar

 %%%%%%%%%%%%%%%%%%%%%%%%%%%%%%%%%%%%%%%%
 % Apéndices, glosarios e bibliografía  %
 %%%%%%%%%%%%%%%%%%%%%%%%%%%%%%%%%%%%%%%%

 \appendix
 \appendixpage
 %\chapter{Material adicional}
\label{chap:adicional}

% \section{Script para limpiar VM de Ubuntu}
% \label{appendix:script-ubuntu-vm-clean}
% Shell script para la limpieza de una máquina virtual Ubuntu después de haber configurado cloud-init. Obtenido del siguiente repositorio \url{https://docs.vmware.com/en/vRealize-Automation/8.1/Using-and-Managing-Cloud-Assembly/GUID-57D5D20B-B613-4BDE-A19F-223719F0BABB.html}.
% \lstinputlisting[language=bash]{anexos/file/clean-ubuntu.sh}
\section{Diseño WD-Server para vRealize Automation}
\label{appendix:wd-server-blueprint}
Diseño elaborado para desplegar una VM con Windows Server 2016 y otra VM con CentOS 8 sobre una misma red en VMware vRealize Automation.
\lstinputlisting[language=python]{anexos/blueprint/windows-server-centOS8.yaml}
%\inputminted[fontsize=\footnotesize,breaklines]{yaml}{./anexos/blueprint/windows-server-centOS8.yaml}

\section{Diseño Wordpress-MySQL-Embedded para vRealize Automation}
\label{appendix:worpress-mysql-blueprint}
Diseño elaborado para desplegar una VM con CentOS 8 y el framework Wordpress preparado para crear un sitio web.
\lstinputlisting[language=python]{anexos/blueprint/worpress-mysql-CentOS8-1VM.yaml}
%\inputminted[fontsize=\footnotesize,breaklines]{yaml}{./anexos/blueprint/worpress-mysql-CentOS8-1VM.yaml}
 
%\include{anexos/...}

 %\chapter*{\nomeglosarioacronimos}
\addcontentsline{toc}{chapter}{\nomeglosarioacronimos}
\label{chap:acronimos}

%%%%%%%%%%%%%%%%%%%%%%%%%%%%%%%%%%%%%%%%%%%%%%%%%%%%%%%%%%%%%%%%%%%%%%%%%%%%%%%%
% Obxectivo: Lista de siglas, abreviaturas, acrónimos, etc. empregados         %
%            no documento, xunto cos seus respectivos significados.            %
%%%%%%%%%%%%%%%%%%%%%%%%%%%%%%%%%%%%%%%%%%%%%%%%%%%%%%%%%%%%%%%%%%%%%%%%%%%%%%%%

\begin{description}
 %\item [ERLANG/OTP] \emph{Erlang Open Telecom Platform}.
 \item [API] \emph{Application Programming Interface}
%  \item [AS] \emph{Autonomous System}
 \item [BGP] \emph{Border Gateway Protocol}
%  \item [BUM] \emph{Broadcast, Unknown Unicast, Multiacast}
%  \item [CA] \emph{Certificate Authority}
%  \item [CITIC] \emph{Centro de Invesitgación en Tecnoloxías da Información e as Comunicacións}
%  \item [CPD] \emph{Centro de Procesamiento de Datos}
 \item [DHCP] \emph{Dynamic Host Configuration Protocol}
 \item [DNS] \emph{Domain Name Server}
%  \item [DPM] \emph{vSphere Distributed Power Management}
%  \item [DR] \emph{Distributed Router}
%  \item [DRS] \emph{vSphere Distributed Resources Scheduler}
%  \item [FTT] \emph{Failures To Tolerate}
%  \item [HA] \emph{vSphere High Availability}
 \item [HDD] \emph{Hard Disk Drive}
 \item [IP] \emph{Internet Protocol}
 \item [iSCSI] \emph{Internet Small Computer System Interface}
 \item [LUN] \emph{Logical Unit Number}
%  \item [MD] \emph{Management Domain}
 \item [MTU] \emph{Maximum Transmission Unit}
 \item [NAT] \emph{Network Address Translation}
 \item [NFS] \emph{Network File System}
%  \item [NIST] \emph{National Institute of Standards and Technology}
 \item [NIC] \emph{Network Interface Card}
 \item [NTP] \emph{Network Time Protocol}
%  \item [N-VDS] \emph{NSX-T Virtual Distributed Switch}
%  \item [PSC] \emph{Platform Services Controller}
%  \item [QoS] \emph{Quality of Service}
 \item [RAID] \emph{Redundant Array of Independent Disks}
%  \item [SAN] \emph{Storage Area Network}
 \item [SDDC] \emph{Software Defined Data Center}
 \item [SFP] \emph{Small Form-factor Pluggable Transceptor}
 \item [SMTP] \emph{Simple Mail Transfer Protocol}
 \item [SSD] \emph{Solid-State Drive}
%  \item [SR] \emph{Service Router}
 \item [TB] \emph{TeraByte}
%  \item [TEP] \emph{Tunnel End Point}
%  \item [ToR] \emph{Switch Top of Rack}
%  \item [TN] \emph{Transport Node}
%  \item [TZ] \emph{Transport Zone}
 \item [UDC] \emph{Universidade da Coruña}
%  \item [UDP] \emph{User Datagram Protocol}
%  \item [VCF] \emph{VMware Cloud Foundation}
 \item [VLAN] \emph{Virtual Local Area Network}
%  \item [VLC] \emph{VMware Lab Constructor}
%  \item [vDS] \emph{vSphere Distribute Switch}
%  \item [VI] \emph{Virtual Infrastructure Domain}
 \item [VMFS] \emph{Virtual Machine File System}
 \item [VM] \emph{Virtual Machine}
%  \item [VNI] \emph{Virtual Network Identifier}
%  \item [vRA] \emph{VMware vRealize Automation}
%  \item [vRSLCM] \emph{VMware vRealize Lifecycle Manager}
%  \item [WD] \emph{Workload Domain}
%  \item [WSA] \emph{Workspace One Access}

\end{description}
%%%%%%%%%%%%%%%%%%%%%%%%%%%%%%%%%%%%%%%%%%%%%%%%%%%%%%%%%%%%%%%%%%%%%%%%%%%%%%%%

 \chapter*{\nomeglosariotermos}
\addcontentsline{toc}{chapter}{\nomeglosariotermos}
\label{chap:glosario-termos}

%%%%%%%%%%%%%%%%%%%%%%%%%%%%%%%%%%%%%%%%%%%%%%%%%%%%%%%%%%%%%%%%%%%%%%%%%%%%%%%%
% Obxectivo: Lista de termos empregados no documento,                          %
%            xunto cos seus respectivos significados.                          %
%%%%%%%%%%%%%%%%%%%%%%%%%%%%%%%%%%%%%%%%%%%%%%%%%%%%%%%%%%%%%%%%%%%%%%%%%%%%%%%%

\begin{description}
 \item [Tenencia múltiple] asdfasdfaasd
 \label{itm:tenenciamultiple}
 \item [SDDC]
 \label{itm:sddc}
 \item [Hipervisor baremetal]
  \label{itm:baremetal}
 \item [Máquina virtual]
  \label{itm:vm}
 \item [Datastore]
 \label{itm:datastore}
 \item [Modelo multi-tenant]
 \label{itm:multitenant}
 \item [RAID 5]
 \label{itm:raid5}
 \item [Almacén de datos]
 \label{itm:almacendatos}
 \item [LUN]
 \label{itm:lun}
 \item [Controlador SFP+]
 \label{itm:sfp}
 \item [SAN]
 \label{itm:san}
 \item [VMFS]
 \label{itm:vmfs}
 \item [Platform Services Controller (PSC)]
 \label{itm:psc}
 \item []
\end{description}

\bibliographystyle{IEEEtran}
\bibliography{\bibconfig,bibliografia/bibliografia}
 %\bibliographystyle{IEEEtran}
 %\bibliography{\bibconfig,bibliografia/bibliografia}
 \cleardoublepage
 
\end{document}

%%%%%%%%%%%%%%%%%%%%%%%%%%%%%%%%%%%%%%%%%%%%%%%%%%%%%%%%%%%%%%%%%%%%%%%%%%%%%%%%
