\chapter{Planificación}
\label{chap:planificacionProyecto}
\lettrine{E}{n} este capítulo se propone una planificación del proyecto con el fin de organizar su estructura y exponer sus costes temporales y económicos aproximados necesarios para su realización.

\section{Tareas}
Lo primero que hay que realizar para llevar a cabo el proyecto es analizar como está formada la infraestructura, que componentes hardware y software la forman y cual es la función de cada uno de ellos, para conocer la base sobre la que se va a trabajar. Se estiman alrededor de 15 horas para la realización de esta tarea.\\

Después, es necesario seleccionar una herramienta de las disponibles en el mercado que se adapte a las necesidades del servicio que se quiere construir y a las características de la infraestructura. Se estiman alrededor de 10 horas para esta tarea.\\

La siguiente tarea será diseñar las soluciones a implementar de aquellas características del servicio que no incluye la herramienta seleccionada ( perfiles de usuario y facturación). Se estiman alrededor de 20 horas para la realización de esta tarea.\\
Lo siguiente será desplegar la herramienta sobre la infraestructura y configurarla. Se estiman unas 17 horas de duración.\\

A continuación, se integra el sistema de autenticación de la UDC con el servicio desplegado. Su duración se estima en 20 horas.\\

Otra tarea a realizar consiste en analizar el uso de recursos que harán los usuarios del servicio. En base a este análisis se establecen unas políticas de uso de recursos, y finalmente se implementa en el servicio. Se estiman 20 horas para la realización de esta tarea.\\

Durante a todo el proyecto  se recopilará información sobre todo lo que engloba el proyecto de forma continua. Se estiman 50 horas repartidas a lo largo de la realización del proyecto.

Finalmente, para la redacción de la memoria del proyecto se estiman 60 horas que se reparten de forma continua durante la realización del trabajo.\\
Entonces, la duración total del proyecto se estima en 212 horas.

\section{Costes}