

\section{Despliegue del servicio}
En este apartado se expone el proceso de despliegue de VMware Cloud Foundation sobre el entorno creado específicamente para esto. Se muestran las configuraciones aplicadas y los cambios realizados sobre la infraestructura.

\subsection{Cumplimiento de requisitos}

\subsubsection{Almacenamiento}
Según la documentación de de VMware, el entorno debe tener una \ref{Word:capacidad} de almacenamiento mínima determinada para poder instalar los componentes de Cloud Foundation si se va a crear como mínimo un Management Domain y un VI Domain. La capacidad inicial del entorno es de 2,34TB de almacenamiento, 96 GB de memoria y 4 nodos ESXi, lo cual no es suficiente para cumplir con este requisito.
Además, Cloud Foundation requiere un disco de arranque en cada nodo de 16GB por lo que es necesario aumentar el tamaño de este disco de 8GB a 16GB.


\subsubsection{Redes}
Antes del despliegue es necesario especificar cuatro redes, una dedicada a vSAN, otra para vMotion, otra para la gestión de los componentes y una cuarta red para el tráfico de las máquinas virtuales (Sección \ref{subsubsec:redLogicaCF}). Inicialmente solo falta una red para habilitar el servicio de vMotion, el resto de redes necesarias ya están creadas y configuradas (Sección \ref{sec:redPrueba}). Esta nueva red también debe tener acceso al resto de componentes por lo que hay que añadir un nuevo adaptador de red (NIC) en el servidor externo. Además, también hay que añadir las nuevas rutas entre redes para que el servidor pueda realizar el enrutamiento del tráfico correctamente.\\
******FOTO NUEVOS ADAPTADORES DE RED*****\\
******FOTO NUEVA TABLA DE ENRUTAMIENTO*****\\

En cuanto a la red física, está configurada para aceptar el MTU igual a 1500 lo cual no cumple con el requisito mínimo (Sección \ref{subsubsec:redFisica}). El tráfico etiquetado si que está habilitado en la red física. Como se está trabajando sobre un entorno físico virtualizado entonces se puede modificar de forma sencilla. Además, cada nodo tiene cuatro \ref{Word:vmnic} por lo que cumple con el mínimo de dos NIC por nodo (Sección \ref{subsubsec:domainVI}).

\subsubsection{Servicios internos y externos}
El tráfico de los servicios y puertos especificados [Sec. \ref{subsubsec:servInterno}] deben estar habilitados en el firewall interno de cada nodo y su tráfico está permitido a través de la red. Inicialmente, el tráfico a cada puerto está habilitado desde cualquier dirección IP.\\
Inicialmente, los servicios externos (DNS, DHCP y NTP) están habilitados en un servidor alcanzable por todas las redes configuradas.
\iffalse
Como es necesario habilitar una nueva red para el componente vMotion entonces hay que añadir un nuevo adaptador de red en el servidor para que esta red pueda acceder a los servicios y comunicarse con el resto de componentes.\\
\fi
Durante el proceso de despliegue de Cloud Foundation se instalarán nuevos componentes por lo que también es necesario configurar el DHCP añadiendo más rangos de direcciones IP, y el DNS para que todos los componentes y máquinas virtuales del entorno sean alcanzables desde todas las redes.\\
*****FOTO DE LO QUE SE HA AÑADIDO EN DHCP Y DNS *****\\

\iffalse
Los servicios externos (Sección \ref{subsubsec:servExterno}) se deben habilitar en un servidor alcanzable por cada nodo por lo que al añadir nuevas redes al entorno es necesario incluir más adaptadores de red en la máquina virtual con Windows Server para darles acceso al DNS, al DHCP y a NTP.
\fi
***RESUMEN DE COMO QUEDA ANTES DE EMPEZAR LA INSTALACIÓN CON LOS CAMBIOS REALIZADOS*******


\subsection{Despliegue VMware Cloud Foundation Builder}
Para instalar VMware Cloud Foundation, antes es necesario instalar VMware Cloud Foundation Builder (Sección \ref{subsec:cloudBuilder}). Esta máquina virtual se despliega desde vSphere Client a partir de un archivo \textit{ova} proporcionado por VMware, en este caso la versión que se instalará es \underline{2.2.0.0-14866160}.
Una vez deplegado Cloud Builder y antes de iniciar la instalación de Cloud Foundation, es necesario adjuntar una hoja de parámetros con los datos de configuración que se deben aplicar, tales como el DNS, redes, y credenciales de los componentes que se van a crear posteriormente.

\subsubsection{Configuración de la máquina virtual}

\subsubsection{Hoja de parámetros}
Esta hoja es un archivo \textit{xlsx} de Excel donde se indica la configuración sobre componentes que ya existen en el entorno y los que se van a desplegar. Estos datos permiten a Cloud Builder y SDDC Manager automatizar el proceso de instalación y configuración inicial de los componentes para generar el Management Domain. Estos datos se dividen en varias categorías:
\begin{itemize}
    \item \textbf{\textit{Prerequisite Checklist}}: muestra una lista de comprobaciones a realizar sobre la configuración inicial del entorno antes de iniciar el despliegue.
    \item \textbf{\textit{Management Workloads}}: permite introducir los números de las licencias necesarias para utilizar todos los componentes de VMware Cloud Foundation y una estimación de los recursos necesarios para el Management Domain.
    \item \textbf{\textit{Users and Groups}}: permite indicar las credenciales para cada usuario de cada componente de VMware Cloud Foundation. Estas credenciales deben cumplir un unos requisitos mínimos.
    \item \textbf{\textit{Hosts and Networks}}: en esta categoría se indican las direcciones IP de las redes requeridas y las direcciones IP de cada host ESXi. También se pueden indicar los fingerprint de SSH y thumprints de SSL de cada host ESXi para que estos sean validados antes de conectarse a ellos durante la instalación (esto es opcional).
    \item \textbf{\textit{Deploy Parameters}}\footnote{Todos los nombres y direcciones IP indicados deben estar configuradas en el DNS para que se puedan resolver a través de ese servicio.}: Aquí se indican los detalles de configuración que se van a desplegar. Existen varios apartados\cite{deploysheet}:
        \begin{itemize}
            \item \emph{Existing Infrastructure Details}: se indican los datos de configuración de servicios configurados inicialmente en el entorno. Estos son DNS, NTP, Single Sign-On y el almacén de vSAN.
            
            \item \emph{vSphere Infrastructure}: se indican los nombres y direcciones IP de las máquinas virtuales de los componentes vCenter Server y Platform Services Controller. 
            
            \item \emph{NSX}: se debe indicar las direcciones IP y los nombres de las máquinas virtuales del componente NSX que se van a desplegar.
            
            \item \emph{vRealize Log Insight}: se indican las direcciones IP y nombres de las máquinas virtuales del \textit{Load Balancer} y de los tres nodos de vRealize Log Insight (un nodo \textit{Master} y dos nodos \textit{Workers}).
            
            \item \emph{SDDC Manager}: se indica el nombre de la máquina virtual, la dirección IP y su máscara de red para el componente SDDC Manager.
        \end{itemize}
    \end{itemize}
Esta hoja de parámetros está adjuntada a este documento.\\
*****METER HOJA EXCEL*****



\subsection{Instalación}


\subsection{Configuración}
