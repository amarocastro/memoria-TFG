\begin{subsubsection}{Diseño de la red del SDDC con VMware NSX-T}
    En un SDDC existe una red virtual que se define mediante software, también se le llama Software Defined Network (SDN). Esta red está desacoplada de la infraestructura física, lo cual permite modificar su configuración sin necesidad de realizar cambios en la infraestructura ni en la configuración de los dispositivos físicos, haciendo más simple su gestión y mantenimiento. Además, este tipo de arquitectura habilita la posibilidad de implementar diferentes configuraciones de red en tiempo reducido, proporcionando elasticidad y flexibilidad a la hora de administrar y obtener los recursos, tanto para el administrador como para el usuario final. 
    El componente encargado de mantener el SDN del SDDC es VMware NSX-T.

    Para mantener la disponibilidad de VMware NSX-T y balancear su carga de trabajo, se despliegan tres instancias de NSX-T Manager Appliance, aunque para reducir el consumo de recursos, en el entorno de prueba solo se creará una instancia de este componente llamada \textit{nsx-mgmt-1}. También se despliegan dos instancias del componente VMware NSX-T Edge, llamadas \textit{edge01-mgmt} y \textit{edge02-mgmt}.

    La virtualización de la red con VMware NSX-T se basa en dos componentes, Transport Zone (TZ) y Segment.
    \begin{itemize}
        \item Transport Zone: se trata de un contenedor dentro del cual se definen Segments. A una TZ se conectan TNs\footnote{Los Transport Nodes son los hosts físicos y cada instancia de VMware NSX-T Edge.} para acceder a los Segments. Cada TN puede estar conectado a varias Transport Zones.
        \item Segment: se trata de un dominio de broadcast de capa 2 que forma parte de una TZ. Las VMs situadas en un TN se pueden conectar a los Segments que existan en la TZ a la que ese TN esté conectado.
    \end{itemize}
    Una TZ se extiende en múltiples TN que pueden estar situados tanto en el mismo dominio broadcast a nivel físico, como en distintas partes de la red física del SDDC. Cuando el tráfico de un Segment debe salir de un TN a la red física para alcanzar su destino\footnote{Este paquete contiene la información de las VMs origen y destino que se están comunicando.}, este es encapsulado de nuevo en un paquete con la información de los TN origen y destino[Fig. \ref{fig:Frame-Segment-NSXT}]. Gracias a esta encapsulación, elementos que se encuentran en distintos entornos de la red física se pueden comunicar como si estuvieran directamente conectados el uno al otro. Así, es posible la creación de una misma red que se extienda por toda la infraestructura del SDDC, permitiendo comunicar componentes situados en distintas redes físicas, sin necesidad de modificar la configuración de los dispositivos físicos ni su topología. Esto hace necesario el uso de un protocolo de enrutamiento dinámico con BGP, tanto en la infraestructura física como en la red virtual, y así automatizar el proceso de configuración de nuevas redes virtuales.
    El tipo de encapsulación que se realiza sobre el tráfico de los Segments se define la configuración de cada TZ, esta puede  ser de tipo VLAN o de tipo Overlay usando el protocolo Geneve.
    \begin{itemize}
        \item TZ de tipo VLAN: se define una VLAN que se utilizará para identificar y encapsular el tráfico perteneciente a los Segments de una misma TZ. La VLAN que se defina debe estar configurada en la red física para que su tráfico sea aceptado.
        \item TZ de tipo Overlay Geneve: este protocolo se encarga de encapsular el tráfico saliente añadiendo una cabecera extra donde incluye un identificador. Cada Segment tendría su propio identificador.
    \end{itemize}
    \begin{figure}[h]
        \centering
        \includegraphics[width=0.8\textwidth]{imaxes/pruebaconcepto/Frame.png}
        \caption{Paquete de un Segment encapsulado cuando sale a la red física.}
        \label{fig:Frame-Segment-NSXT}
    \end{figure}
    \FloatBarrier
    En la imagen anterior se muestra como se encapsula un paquete perteneciente a un Segment cuando este sale de un host/TN al medio físico. En las cabeceras del paquete correspondiente al Segment tendrá la información sobre las VMs origen y destino que se están comunicando, mientras que las cabeceras que encapsulan a ese paquete contienen la información sobre los hosts/TNs origen y destino donde se encuentran las VMs que se están comunicando. La dirección IP utilizada por los hosts/TNs para enviar el tráfico de un Segment encapsulado, se denomina Tunnel End-Point (TEP) y se asigna mediante DHCP para automatizar su configuración cuando un nuevo host/TN es añadido al entorno.

    \begin{figure}[h]
        \centering
        \includegraphics[width=0.8\textwidth]{imaxes/pruebaconcepto/OverlayTZSegments.png}
        \caption{Segments de la Transport Zone \textit{mgmt-domain-m01-overlay-tz}}
        \label{fig:overlay-TZ-segments-NSXT}
      \end{figure}
      \FloatBarrier

      En la imagen anterior se muestra la Transport Zone de tipo Overlay, definida en el entorno de pruebas, con el nombre \textit{mgmt-domain-m01-overlay-tz}. A ella se conectan los seis TNs\footnote{Los seis TNs son los cuatro hosts y las dos instancias de NSX-T Edge}\footnote{Un TN utiliza el elemento NSX-T Virtual Distributed Switch (N-VDS) para conectarse a los Segments de una TZ} y contiene tres Segments. El Segment \textit{mgmt-xRegion01-VXLAN} se utiliza para desplegar aplicaciones cuyas instancias deben ser accesibles desde cada Region del SDDC. El Segment \textit{mgmt-Region01A-VXLAN} tiene como finalidad alojar aplicaciones que solo deben ser accesibles desde dentro de una misma Region. El Segment \textit{sddc-host-overlay} es utilizado por los componentes de VMware NSX-T para comunicarse con los TNs. Con cada Segment de tipo Overlay se genera un \textit{port group} con el mismo nombre en el vDS (se puede ver en la figura \ref{fig:port-groups-vSwitch-vSphere}) que se utilizan para transmitir su tráfico a la red física.
    %    VMware NSX-T genera en el vSphere vSwitch un \textit{port group} por cada \textit{segment} para poder conectar la VM de cada componente al \textit{port group} que le corresponda.

    \begin{figure}[h]
        \centering
        \includegraphics[width=0.8\textwidth]{imaxes/pruebaconcepto/VLANTZSegments.png}
         \caption{Segments de la  Transport Zone \textit{sfo01-m01-edge-uplink-tz}}
        \label{fig:VLAN-TZ-segments-NSXT}
    \end{figure}
    \FloatBarrier
    En la imagen anterior se muestra la Transport Zone de tipo VLAN, definida en el entorno de pruebas, con el nombre \textit{sfo01-m01-edge-uplink-tz}. A ella se conectan los dos TNs que son instancias de NSX-T Edge, \textit{edge01-mgmt} y \textit{edge02-mgmt}. Contiene dos Segments, \textit{VCF-edge-mgmt-cluster-segment-11} y \textit{VCF-edge-mgmt-cluster-segment-12}, que son utilizados por las instancias de NSX-T Edge para transmitir el tráfico de todas las redes virtuales gestionadas por VMware NSX-T hacia redes físicas externas al SDDC. Para ello utilizan utilizan los \textit{port groups} \textit{sddc-edge-uplink01} y \textit{sddc-edge-uplink02} (se pueden ver en la figura \ref{fig:port-groups-vSwitch-vSphere}) del vDS para transmitir su tráfico hacia las interfaces de red físicas de cada host. Ambas instancias forman la topología de red, que se muestra en la siguiente imagen, para comunicar las redes virtuales de VMware NSX-T con el router físico.
    \begin{figure}[h]
        \centering
        \includegraphics[width=0.4\textwidth]{imaxes/pruebaconcepto/UplinkDesign.png}
        \caption{Topología de red de las insterfaces \textit{uplink}.}
        \label{fig:Uplink-Design-Edge-NSXT} 
    \end{figure}
    \FloatBarrier
    Como se puede ver en la imagen anterior, los dos Segments son utilizados por las instancias de NSX-T Edge para mantener rutas redundantes hacia la red externa y así aumentar su disponibilidad. A parte, este componente también se encarga de proporcionar un conjunto de servicios de red a los componentes que están conectados a los Segments de VMware NSX-T. Para entregar estos servicios, internamente, VMware NSX-T forma una topología con una serie de routers virtuales.
    \begin{figure}[h]
        \centering
        \includegraphics[width=0.4\textwidth]{imaxes/pruebaconcepto/topologiaTwoTierRouting-Final.png}
        \caption{Topología virtual de VMware NSX-T}
        \label{fig:Uplink-Design-Edge-NSXT} 
    \end{figure}
    \FloatBarrier
    En la imagen anterior se muestra la topología que forman VMware NSX-T para proporcionar acceso a la red externa y para entregar otros servicios a los componentes situados en los dos Segments existentes. En esta topología hay tres routers, \textit{mgmt-domain-tier0-gateway}, \textit{mgmt-domain-tier1-gateway} y \textit{mgmt-domain-lb01-tier1-gateway}, de los cuales, el primero se encarga de gestionar la comunicación con los dispositivos de red físicos, y los dos restantes se encargan de enrutar el tráfico entre Segments y hacia el router de Tier 0, y de entregar distintos servicios a los componentes que residen en los Segments. Los servicios de red que se pueden habilitar en los dos routers de Tier 1 son NAT, Load Balancing, DNS y VPN.
\end{subsubsection}