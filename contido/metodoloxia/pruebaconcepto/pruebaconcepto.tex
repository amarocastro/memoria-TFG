\begin{section}{Prueba de concepto}
    \label{subsect:prueba-concepto}
Para no afectar al funcionamiento de los trabajos que se llevan a cabo en el CITIC, el proyecto se lleva a cabo en un entorno aislado formado por un host y un datastore, en el cual se despliegan todos los componentes de VCF con el fin mostrar y probar las capacidades y características de VMware Cloud Foundation. 
El proceso se realizará siguiendo la metodología Scrum, donde en cada ciclo se realiza el despliegue de uno o varios componentes y posteriormente se revisa su configuración y funcionamiento. Primero se instalarán los componentes base de VMware Cloud Foundation\footnote{Los componentes base de VCF son VMware vSphere, VMware vSAN y VMware NSX-T} usando el programa VMware Lab Constructor (VLC) v4.0.1\footnote{Herramienta que permite crear un generar de forma automatizada un entorno embebido para probar las funcionalidades de VMware Cloud Foundation.}. Después se instalarán los componentes de la suite VMware vRealize, uno dedicado a la gestión de usuarios del SDDC y otro que proporciona un servicio de aprovisionamiento de recursos. Finalmente, se comprobará el funcionamiento general del SDDC y las posibilidades que ofrece el servicio Cloud desplegado.

\begin{subsection}{Preparación}
    \begin{subsubsection}{Host ESXi}  
    
    Como base para la instalación se utiliza un servidor físico con el hipervisor ESXi instalado. Este host se utiliza para desplegar los componentes de VMware Cloud Foundation para crear un pequeño SDDC embebido para probar sus funciones. Este host cuenta con una memoria RAM de 192 GB, una CPU de 28,8 GHz y un \textit{datastore} con discos SSD con 2 TB de capacidad. Cuenta con dos interfaces físicas, una que conecta al host con el \textit{datastore} y otra a la que se conectan dos redes, una llamada \textit{Management Network} que permite acceder al host desde una VM para gestionarlo, y otra llamada \textit{VM Network} donde se conectan todas las VMs generadas por VLC y de los servicios que dan soporte a los componentes de VMware Cloud Foundation.
    \end{subsubsection}
    \begin{subsubsection}{Servicios}
      Todos los servicios requeridos por VMware Cloud Foundation se despliegan sobre el mismo servidor en forma de VMs. Una de las VMs es Windows Server 2016 que contiene un servidor DNS, un servidor NTP, un servidor Active Directory, un servidor SMTP y ejerce también como Certificate Authority. Otra VM contiene el sistema operativo VyOS que funciona como un router virtual y como servidor DHCP. Una última VM con Windows 10\footnote{Se refiere a ella como \textit{Jump Host}.} se requiere para ejecutar VLC y acceder al entorno embebido generado por VLC.
      El servidor DNS contiene el nombre y su respectiva dirección que un componente de VCF utilizará para que sus instancias se puedan comunicar con otras. Este servidor DNS implementa un único dominio que se denomina \textit{pesci.domain}. El servidor Active Directory proporciona un almacén de usuarios y grupos de usuarios a los cuales se les configura un rol dentro de cada componente de SDDC. Se utiliza este repositorio de usuarios en lugar del directorio real de la UDC para evitar posibles problemas del servicio. El router VyOS tiene configuradas todas las subredes y VLANs que VMware Cloud Foundation utiliza en la capa L3 de la infraestructura física y proporciona acceso a Internet, en las cuatro interfaces que conectan con las instancias de VMware NSX-T Edge utiliza enrutamiento dinámico BGP. El servidor DHCP se utiliza para asignar una dirección IP a las interfaces Tunnel EndPoint (TEP)\footnote{Más adelante se describirá la función de este elemento} de cada host ESXi.    
    \end{subsubsection}
    
    \begin{subsubsection}{VMware Lab Constructor}
      VLC genera en el host ESXi cuatro VMs que representan cuatro hosts ESXi. posteriormente, dentro de estos hosts VLC inicia la creación del \textit{management domain} de esta infraestructura embebida incluyendo todos los componentes de VMware Cloud Foundation. El diseño y configuración generados se describirá en las siguientes secciones.
      \begin{figure}[h!]
        \centering
        \includegraphics[width=0.6\textwidth]{imaxes/pruebaconcepto/hostFisico.png}
        \caption{Muestra la estructura generada por el instalador VLC. Cuatro hosts ESXi embebidos con los componentes de VMware Cloud Foundation cuyo tráfico circula a través del \textit{port group} VM Network.}
        \label{fig:estructura-generada-por-VLC}
      \end{figure}
      \FloatBarrier
  
      \begin{figure}[h]
        \centering
        \includegraphics[width=0.6\textwidth]{imaxes/pruebaconcepto/vSwitch0HostFisico.png}
        \caption{Máquinas virtuales en el host físico.}
        \label{fig:VMs-alojadas-host-fisico}
      \end{figure}
      \FloatBarrier
  
      En la imagen anterior se muestran las VMs que están funcionando sobre el host físico y que representan los componentes de la infraestructura física de un SDDC real, junto con el número de interfaces que se utilizan en cada una. Cada host ESXi generado por VLC cuenta con dos interfaces de red. El router VyOS, Jump Host y Windows Server 2016 se configuran antes del despliegue de VMware Cloud Foundation con VLC y se comunican con el entorno generado por VLC a través del \textit{port group} VM Network. El \textit{port group} Management Network se utiliza para acceder a la configuración del host físico a través de la dirección IP que se indica. Se utiliza la interfaz vmnic0 del host como salida del tráfico generado por el vSwitch0.
      \FloatBarrier
  
      \begin{figure}[h]
        \centering
        \includegraphics[width=0.4\textwidth]{imaxes/pruebaconcepto/RouterFisicoL3.png}
        \caption{Interfaces del router Vyos.}
        \label{fig:interfaces-router-fisico-L3}
      \end{figure}
      \FloatBarrier
  
      En la imagen anterior se muestra la configuración del router VyOS. Cada una de las interfaces se debe configurar antes del despliegue de VCF. Todas usan MTU de 9000 Bytes ya que la mayoría de componentes de VCF utilizan paquetes de red \textit{jumbo frame}. En las interfaces Eth2 y Eth3 el router utiliza enrutamiento dinámico BGP donde el AS local es 65001 y el AS remoto es AS 65003, configurado para anunciar a sus vecinos la red 10.0.0.0/24 Management Network. Las direcciones configuradas como \textit{neighbour} son: 172.27.11.2, 172.27.11.3, 172.27.12.2 y 172.27.12.3. En la dirección IP 172.27.254.199 de la interfaz eth0, el router proporciona un servidor DHCP que asigna direcciones IP en el rango 172.16.254.0 - 172.16.254.100.
      \FloatBarrier
  
      \begin{figure}[h]
        \centering
        \includegraphics[width=0.6\textwidth]{imaxes/pruebaconcepto/RedDesdeDentro.png}
        \caption{Topología de las redes del entorno desplegado.}
        \label{fig:red-L3-infraestructura-fisica}
      \end{figure}
      \FloatBarrier
  
      En la imagen anterior se muestran todos los componentes de VMware Cloud Foundation desplegados por VLC y los desplegados posteriormente para completar los objetivos del proyecto, como se conectan con los distintos servicios de red y a que redes se conectan. Las redes Mgmt-xRegion01-VXLAN y Mgmt-Region01A-VXLAN se corresponden a redes virtuales gestionadas por VMware NSX-T que no requieren ninguna configuración adicional en la capa 3 de la infraestructura física (esto se verá con detalle en el apartado de diseño de VMWare NSX-T).
      \FloatBarrier
    \end{subsubsection}
    
  \end{subsection}

\begin{subsection}{Diseño y configuración del Management Domain}
En esta sección se describen las funciones y configuración de los componentes desplegados en el entorno de pruebas con la ayuda de VLC. 

\begin{subsubsection}{Diseño de VMware vCenter Server}

El componente VMware vCenter Server es el punto de acceso y de control de todas las VMs localizadas en los hosts ESXi situados bajo su dominio. VMware vCenter Server funciona sobre una VM situada en el Management Domain. Esta instancia de vCenter Server contiene un dominio con un cluster vSphere que agrupa a los cuatro hosts ESXi que forman el Magement Domain. Estos hosts se denominan respectivamente \textit{esxi-1}, \textit{esxi-2}, \textit{esxi-3} y \textit{esxi-4}, y cada uno cuenta con 64 GB de memoria RAM y 19,9 GHz de CPU. Desde VMware vCenter Server el administrador gestiona los recursos de las VMs de cada componente, monitoriza los recursos, administra la creación y asignación de roles, permisos y usuarios, gestiona los grupos de discos que forman el \textit{datastore} de VMware vSAN, determina las redes a las que se conecta cada componente, establece la configuración de disponibilidad y recuperación ante fallos proporcionada por VMware vSphere, en definitiva, VMware vCenter Server es el punto desde donde se controla y administra el uso de recursos por parte de las VMs desplegadas. Además, integra el componente PSC, el cual controla la identidad y permisos de los administradores y aplicaciones que acceden a VMware vCenter, y gestiona el almacenamiento de licencias de VCF. El acceso a VMware vCenter Server se hace a través del componente web vSphere Client.
\begin{figure}[h]
  \centering
  \includegraphics[width=0.2\textwidth]{imaxes/pruebaconcepto/clusterVCenterServer.png}
  \caption{Dominio y cluster vSphere del Management Domain.}
  \label{fig:cluster-vCenter-Server}
\end{figure}
\FloatBarrier
En la imagen anterior se muestra el dominio (\textit{vcenter-mgmt.pesci.domain}), de la instancia de VMware vCenter Server, y el cluster vSphere (\textit{mgmt-cluster}) donde se alojan los componentes del Management Domain. Este cluster incluye los cuatro hosts ESXi y cuatro \textit{resource pools}, uno de ellos contiene las VMs de los componentes dedicados al Management Domain.
%  Con vCenter Server se simplifica la escalabilidad del SDDC, la gestión de actualizaciones para los componentes es más sencilla, permite determinar roles específicos y responsabilidades y permite aislar las redes de otras instancias de vCenter Server. Además, para gestionar vSpehere SSO Domain, VMware vCenter Server contiene embebido el componente PSC con todos los servicios necesarios. 
%  En caso de que existan varios \textit{Workload Domain} se puede habilitar el modo \textit{Enhanced Linked Mode} para poder gestionar todas las instancias de vCenter Server de forma centralizada desde un único vSphere Client.
% Por lo anterior, en el \textit{management domain} se despliega una instancia de VMware vCenter Server que incluye un cluster de VMware vSphere.

\end{subsubsection}


\begin{subsubsection}{Diseño almacenamiento VMware vSAN}
  
  Los hosts del Management Domain utilizan como almacenamiento un \textit{datastore} del componente VMware vSAN. Está formado por 16 discos SDD agrupados en cuatro grupos con configuración All-Flash, cada grupo está asociado a un host. Para mantener la disponibilidad del los ficheros almacenados en el \textit{datastore}, se establece la opción \textit{Failures-To-Tolerate} (FTT) igual a uno. De esta forma, VMware vSAN mantiene dos copias de los archivos generados por las VMs y las coloca en grupos de discos distintos, de forma que si ocurre un fallo en alguno de los hosts las VM seguirán teniendo acceso a sus archivos. Esta configuración equivale a tener un sistema de almacenamiento RAID 1, pero con la ventaja de que no se ha modificado la configuración del hardware y, si fuera necesario, se podría aumentar el número de réplicas simplemente editando el valor de FTT desde el portal de VMware vCenter Server. Como se muestra en la siguiente figura, VMware vSAN mantiene una copia del mismo archivo en dos hosts/grupos de discos diferentes, mientras la configuración física de cada grupo de discos es de tipo RAID 0. Las máquinas virtuales acceden al \textit{datastore} a través de una subred que utiliza su propia VLAN y a la que todos los hosts están conectados.
  \begin{figure}[h]
    \centering
    \includegraphics[width=0.8\textwidth]{imaxes/pruebaconcepto/vSANconfig.png}
    \caption{Ejemplo de como se almacena un archivo con VMware vSAN y FTT igual a uno}
    \label{fig:vSAN-config-FTT}
  \end{figure}
  \FloatBarrier
  Utilizar el sistema de almacenamiento de VMware vSAN supone una gran mejora respecto al sistema de almacenamiento basado en LUNs, utilizado actualmente en el CPD del CITIC. VMware vSAN monitoriza los dispositivos de almacenamiento y configura la redundancia de los archivos de forma dinámica y sencilla, permitiendo establecer una configuración específica según sea necesario, sin modificar los dispositivos físicos de almacenamiento. Con el sistema basado en LUNs, es obligatorio modificar la estructura y configuración de los dispositivos físicos cada vez que se quiera establecer una configuración de redundancia diferente en el sistema de almacenamiento, lo cual supone un gran coste para el administrador y un aumento de los riesgos. Si tomamos el ejemplo de la figura anterior, la redundancia el sistema gestionado por VMware vSAN, con FTT igual a 1, podría ser aumentada estableciendo la opción de configuración FTT igual a 2. Así, se crearía una nueva copia del archivo en un tercer grupo de discos, mientras la configuración física se mantiene igual.
  % sts participantes, soporta el fallo de un host lo cual permite dejar hosts fuera de servicio para tareas de mantenimiento. Esto es posible gracias a que con FTT (\textit{Failures-To-Tolerate}) igual a 1 se mantiene la redundancia de los datos almacenados en el \textit{datasotore}, en uno de los hosts. Cada grupo de discos cuenta con cuatro discos uno de ellos para caché, 16 discos en total. Para hacer disponible este servicio de almacenamiento, todos los hosts deben estar conectados a la subred generada para VMware vSAN y utilizar una VLAN para separar su tráfico.

\end{subsubsection}
    
\begin{subsubsection}{Diseño cluster VMware vSphere}
Como ya se ha mencionado, los cuatro hosts desplegados para el Management Domain están agrupados en un cluster de VMware vSphere. Gracias a dos funcionalidades de este componente, se establece una configuración para mantener activas las VMs desplegadas\footnote{Las VMs a las que se refiere son las instancias de cada componente de VCF.} dentro de este cluster. Entonces, VMware vSphere se encarga, de forma automatizada, de balancear el consumo de recursos y de recuperar el servicio de las VMs cuando alguna sufre un fallo. Estas funciones de VMware vSphere son:
\begin{itemize}
  \item vSphere High Availability: establece una cantidad de recursos que se reserva de los disponibles en el cluster vSphere, y se encarga de reiniciar una VM cuando deja de estar operativa. Para este cluster se establece una reserva el 25\% de la CPU total y el 25\% de la memoria RAM total. De esta forma, se asegura que una cuarta parte de los recursos disponibles están reservados para reiniciar, en un host diferente, una VM que ha dejado de funcionar.
  
  \item vSphere DRS: se encarga de migrar VMs de un host a otro dentro del cluster vSphere,con el objetivo de balancear la carga de trabajo entre los hosts disponibles. Usando este servicio se garantiza que cada VM obtiene la capacidad necesaria para funcionar correctamente, y aumenta la eficiencia del cluster al hacerse un mejor uso de sus recursos. Para realizar las migraciones entre hosts, vSphere DRS utiliza la funcionalidad vMotion, el cual permite mover una VM de un host a otro manteniendo el estado en el que se encontraba, y manteniendo activo el servicio de la VM. Por ejemplo, si el consumo de recursos de un host está alrededor del 100\%, vSphere DRS lo detecta e inicia la migración de la VM mediante vMotion, a otro host con recursos disponibles. 
\end{itemize}

Combinando estas dos funcionalidades, las tareas de mantenimiento se reducen ya que es VMware vSphere quien, de forma automatizada y transparente, se encarga de monitorizar el estado de VMs y hosts, de optimizar el uso de recursos y de asegurarse de que existen suficientes recursos para la ejecución de todos los flujos de trabajo.

% Dentro de un \textit{workload domain} pueden existir varios clusters vSphere con diferentes características según su finalidad. Los hosts ESXi que lo forman pueden ser de diferentes tamaños teniendo en cuenta que se pueden usar menos hosts ESXi de mayor capacidad o más hosts con menores prestaciones, el coste de cada host ESXi, el uso que se le va a dar al cluster y las características máximas y mínimas del cluster vSphere. Debido a la limitada cantidad de recursos que ofrece el host físico donde se realiza el despliegue, para el \textit{management domain} se utiliza un único cluster vSphere con de 4 hosts de los cuales se reserva un host para proveer redundancia. Todos los hosts ESXi cuenta con 64GB de memoria RAM menos uno que tiene 32 GB, y 19.9GHz de CPU. Dentro del cluster hay que configurar los servicios vSphere HA y vSphere DRS para proteger los componentes del SDDC. La configuración que se establece en el \textit{management domain} es la siguiente:
% En caso de que el \textit{management domain} esté extendido en dos AZ entonces se requieren 4 hosts en cada AZ para proporcionar redundancia y disponibilidad en caso de caída de una de las AZ.

% \begin{itemize}
%     \item \textbf{vSphere High Availability}: en este servicio la propiedad \textit{Admission Control Policy} permite establecer la cantidad de recursos reservados en caso de fallo y como se establece el cálculo de esos recursos. En el \textit{management domain} se configura para el fallo de al menos un host y reserva de recursos según un porcentaje, reservando así el 25\% de la CPU y el 30\% de la memoria RAM ya que funciona mejor cuando las VM usan mucha CPU y memoria. La otra propiedad que se debe habilitar para el correcto funcionamiento del servicio es \textit{VM and Application Monitoring}, que se encarga de reiniciar las VM en caso de caída.
    % que puede ser según el número hosts que pueden fallar en el cluster, según un porcentaje de reserva de rescursos o especificando el host donde se recolocan las VM del host caído.  RAM.  
%     \item \textbf{vSphere DRS}: 
% este servicio permite migrar VMs de un host ESXi a otro dentro del mismo cluster vSphere para equilibrar la carga de trabajo y mantener las VMs activas en caso de caída de alguno de los hosts. Se activa usando la opción por defecto \textit{Fully Automated} ya que aporta el mejor balance entre consumo de recursos y migraciones de VM innecesarias. Adicionalmente también se pueden establecer reglas para determinar el orden de encendido de las VMs pertenecientes a un mismo grupo. 
%     %En caso de que exista más de una AZ, se deben crear grupos de VM y de hosts de cada AZ para luego implementar reglas de afinidad para que las VM de una AZ no sean migradas a otra AZ ya que esto puede afectar al rendimiento de la VM. 
% \end{itemize}
% En el modelo consolidado se debe crear un único cluster con un mínimo de cuatro hosts ESXi ya que uno de los hosts se utiliza para asegurar la disponibilidad del almacenamiento vSAN cuando hay algún host inactivo. Este modelo proporciona capacidad de un único fallo por cluster.
\end{subsubsection}

\begin{subsubsection}{Diseño de red para el cluster vSphere}
  A parte de controlar la disponibilidad de los recursos, VMware vSphere también se encarga de gestionar las redes a las que se conecta cada VM, permitiendo separar cada tipo de tráfico en subredes y asignarles unas propiedades específicas. Para llevar esto a cabo y que las VMs puedan conectarse a la red externa y comunicarse con el resto de VMs, dentro del cluster vSphere se crea un vShpere Distributed Switch (vDS), en el cual se configuran puertos a los que se conectan las VMs alojadas en el cluster vSphere.
  % que las VMs tengan conectividad con la red externa y con el resto de VMs, en el cluster vSphere, se crea un vShpere Distributed Switch (vDS), en el cual se configuran puertos a los que se conectan las VMs alojadas en el cluster vSphere.
  \begin{figure}[h]
    \centering
    \includegraphics[width=0.5\textwidth]{imaxes/pruebaconcepto/distributedSwitchEntornoFinal.png}
    \caption{Contenido de vSphere Distributed Switch \textit{sddc-vds01}.}
    \label{fig:port-groups-vSwitch-vSphere}
  \end{figure}
  \FloatBarrier

  Como se muestra en la imagen anterior, el vDS creado para el cluster vSphere del Management Domain contiene varios puertos, donde hay VMs conectadas, y dos interfaces uplink (\textit{sddc-vds01-DVUplinks-10}). Estas dos interfaces, \textit{uplink1} y \textit{uplink2}, representan las interfaces de red físicas de cada host y son las que dan salida al tráfico de las VMs hacia la red física del entorno. Cada uno de los puertos tiene una función específica, estos son, \textit{sddc-vds01-mgmt}, dedicado al tráfico de configuración y gestión que los componentes de VCF envían entre sí, \textit{sddc-vds01-vmotion}, dedicado al tráfico de las migraciones de VMs entre hosts llevadas a cabo por la funcionalidad vMotion, \textit{sddc-vds01-vsan}, usado por las VMs para acceder al datastore de VMware vSAN el cual es el sistema de almacenamiento del entorno, \textit{sddc-edge-uplink01} y \textit{sddc-edge-uplink02}, puertos usados por los componentes de VMware NSX-T para dar salida, hacia la red física, al tráfico de las redes virtuales que gestiona este componente de VCF. Los demás puertos que se muestran en la imagen son generados de forma automática por VMware NSX-T. 
  En la configuración de cada puerto, se establece la VLAN que se asigna al tráfico que circula a través de él, las interfaces uplink por las que se transmite su tráfico hacia la red física y como se balancea la carga entre cada interfaz uplink, y la prioridad, respecto al resto de puertos, que se asigna al tráfico que circula por él. Los puertos, cuyo tráfico tiene mayor prioridad son, \textit{sddc-vds01-vsan} y \textit{sddc-vds01-vmotion},  para asegurarse de que obtienen el suficiente ancho de banda y así facilitar la transmisión de archivos de gran tamaño. 
  % re cada puerto se establecen varias propiedades En cuanto a las propiedades del tráfico, dentro del vDS se determina qué tráfico tiene más prioridad sobre los recursos de red, en este caso se establece el tráfico de los puertos \textit{sddc-vds01-vsan} y \textit{sddc-vds01-vmotion} como los de mayor prioridad, para asegurarse de que obtienen el suficiente ancho de banda y así facilitar la transmisión de archivos de gran tamaño. También, para cada puerto, se establece la VLAN con la que se etiqueta el tráfico, qué interfaces de red físicas se deben utilizar y como debe balancear el tráfico entre estas.
\\
  De esta forma, cada subred utilizada por los componentes de VCF es asociada con un puerto del vDS, y por lo tanto, las propiedades del tráfico de cada subred son configuradas a través de VMware vSphere. Esto, simplifica el proceso administración y configuración de las redes del entorno, ya que una vez configurados los dispositivos de red físicos, el router VyOS en este caso, con las direcciones IP, las etiquetas VLAN y MTU correspondientes a cada subred, la monitorización de la red y la configuración de la calidad del servicio se realizan desde VMware vSphere.
  
  

% Si bien en VMware Cloud Foundation existe VMware NSX-T, un componente dedicado únicamente a la administración de la red del SDDC, es desde VMware vSphere dónde se encuentran los elementos para establecer redes que separen cada tipo de tráfico de los componentes del SDDC. Estas redes se configuran en base a los siguientes aspectos:
% \begin{itemize}
%     \item Separar el tráfico de cada servicio para mejorar la eficiencia de la red y la seguridad. Así se puede ajustar las características de cada red, como el ancho de banda o la latencia, a las necesidades de cada servicio.
%     \item Utilizar un único vSphere Distributed Switch por cluster donde se añade un \textit{port group} por cada servicio.
%     % \item Mejorar el rendimiento usando NICs de tipo VMXNET3 en las máquinas virtuales.
%     \item Las NICs físicas de cada host ESXi conectados a un mismo vSphere Distributed Switch están conectadas también a la misma red física.
%     % \item Aquellas redes que se dedican a servicios de la infraestructura deben estar configuradas con puertos tipo \textit{vmkernel}.
% \end{itemize}
% Para el \textit{management domain} del SDDC se crea un único vSphere Distributed Switch llamado \textit{sddc-vds01} con la siguiente configuración:
% \begin{itemize}
    
%     \item Se establece un MTU igual 9000 Bytes para permitir el tráfico de \textit{jumbo frames} ya que son requeridos por algunos de los servicios.
    
%     \item Se habilita el servicio \textit{Network I/O} que permite establecer un nivel de prioridad a cada tipo de tráfico. Esto se realiza estableciendo limites de ancho de banda, políticas de balanceo de carga y reserva de recursos para un tipo de tráfico asociado a un servicio. Por cada tipo de tráfico hay cuatro aspectos que se pueden configurar que son \textit{Shares} (indica el \% de ancho de banda que se le da a un tipo de tráfico, el tipo de tráfico que tenga un mayor valor en \textit{Shares} tendrá más prioridad a la hora de usar los recursos), \textit{Reservation} (indica el valor de ancho de banda que se reserva para el tipo de tráfico) y \textit{Limit} (establece un valor máximo para el ancho de banda de un tipo de tráfico). En el \textit{management domain} los tipos de tráfico más relevantes que se deben configurar son los siguientes:
%     \begin{itemize}
%       \item \textit{Management Traffic}: el valor \textit{Shares} se establece al 50\% (\textit{Normal}) lo cual le da mayor prioridad que el resto de tipos. El resto de valores no se modifican.
%       \item \textit{vSphere vMotion Traffic}: el valor \textit{Shares} se establece al 25\% (\textit{Low}) ya que durante el estado normal del entorno este tipo de tráfico no es muy importante. El resto de valores no se modifican.
%       \item \textit{vSAN Traffic}: el valor \textit{Shares} se establece al 100\% (\textit{High}) para garantizar que este servicio recibe la cantidad de ancho de banda que necesita. El resto de valores no se modifican.
%       \item \textit{Virtual Machine Traffic}: el valor \textit{Shares} se establece al 100\% (\textit{High}) para garantizar que las VMs siempre tienen acceso a la red ya que son una parte importante del SDDC. El resto de valores no se modifican.
%     \end{itemize}
    
%     \item Para detectar errores de compatibilidad entre la configuración del vSphere Distributed Switch y la red física se habilita el servicio \textit{Health Check}. Este se encarga de comprobar si la configuración de cada VLAN y MTU se adapta a la configuración de la capa física.
    
%     \item Como puertos de salida \textit{Uplink} se configuran las interfaces físicas \textit{vmnic0} y \textit{vmnic1}. Como vDS es un componente distribuído, en cada host se usarán ambas interfaces de red como \textit{uplinks}.
    
% \end{itemize}
% En este vSpehere Distributed Switch para el Management Domain se configuran los siguientes \textit{port groups}, que son de tipo \textit{Distributed port group} y de tipo \textit{Uplink port group}. Además, el vDS está configurado sobre los cuatro hosts por lo tanto todos tienen acceso a todos los \textit{port groups}:
% \begin{itemize}
       
%         \item \textbf{Management port group}: es un \textit{Distributed port group} que comunica a todos los hosts ESXi entre si y transmite el tráfico entre los diferentes componentes de VMware Cloud Foundation, es decir, por este \textit{port group} circulan los comandos de configuración y gestión que los componentes del SDDC se envían entre ellos. Tiene el nombre \textit{sddc-vds01-mgmt}, a él se conectan las VMs \textit{vcenter-mgmt}, \textit{sddc-manager}, \textit{nsx-mgmt-1},\textit{edge01-mgmt} y \textit{edge02-mgmt}. Utiliza la subred con IP 10.0.0.0, máscara de red 255.255.255.0, VLAN 10 y MTU igual a 1500 Bytes. Esta red debe ser configurada también en la infraestructura física.
        
%         \item \textbf{vMotion port group}: es un \textit{Distributed port group} que está dedicado al tráfico del componente vSphere vMotion para realizar las migraciones de máquinas virtuales de un host a otro. Tiene el nombre \textit{sddc-vds01-vmotion} y utiliza la subred con IP 10.0.4.0, máscara de red 255.255.255.0, VLAN 10 y MTU igual a 8940 Bytes.
        
%         \item \textbf{vSAN port group}: es un \textit{Distributed port group} que está dedicado al servicio de almacenamiento VMware vSAN y por él los hosts acceden al almacenamiento del SDDC. Tiene el nombre \textit{sddc-vds01-vsan} y utiliza la subred con IP 10.0.8.0, máscara de red 255.255.255.0, VLAN 10 y MTU igual a 8940 Bytes.
        
%         \item \textbf{Edge Uplink port group}: es un \textit{Distributed port group} dedicado a las conexiones del component NSX-T Edge que se dedica a dar acceso a determinados servicios y para proporcionar a otros \textit{workload domain} conexión con la red externa. Están gestionados por VMware NSX-T ya que dan servicio a sus componentes. En el entorno existen dos \textit{port groups} para proporcionar redundancia y alta dispobilidad, uno llamado \textit{sddc-edge-uplink01} cuyas instancias están configuradas bajo la red con IP 172.27.11.0 y con máscara de red 255.255.255.0, y otro llamado \textit{sddc-edge-uplink02} cuyas instancias están configuradas bajo la red con IP 172.27.12.0 y máscara de red 255.255.255.0. Ambos \textit{port groups} están configurados como VLAN Trunk (por ellos puede circular tráfico de cualquier VLAN) y tienen un MTU de 8940 Bytes. En ambos hay configuradas las dos VMs llamadas \textit{edge01-mgmt} y \textit{edge02-mgmt}. Estas dos redes también se deben configurar en la infraestructura física.
        
%         \item \textbf{Uplink port group}: se trata de un \textit{Uplink port group} al que se le asignan las NICs físicas de cada host para establecer políticas sobre el tráfico que se dirige desde los hosts y VMs hacia fuera del vSphere Distributed Switch. Con el nombre \textit{sddc-vds01-DVUplinks-10}, en él están configuradas las dos NICs físicas de cada host, cada una en una interfaz \textit{uplink}.
        
% \end{itemize}
% \begin{figure}[h]
%   \centering
%   \includegraphics[width=0.4\textwidth]{imaxes/pruebaconcepto/distributedSwitchEntornoFinal.png}
%   \caption{Contenido de vSphere Distributed Switch \textit{sddc-vds01}.}
%   \label{fig:port-groups-vSwitch-vSphere}
% \end{figure}
% \FloatBarrier
% En la imagen anterior se muestran todos los \textit{Distributed Port Groups} y \textit{Uplink port group} que se alojan en el vSphere Distributed Switch (\textit{sddc-vds01}) dedicado al \textit{management domain}. En el \textit{port group} \textit{sddc-vds01-DVUplinks-10} se muestra como cada interfaz \textit{uplink} se mapea con una interfaz física (vmnic) de cada host ESXi. Los \textit{port groups} \textit{mgmt-Region01A-VXLAN}, \textit{mgmt-xRegion01-VXLAN} y \textit{sddc-host-overlay} son generados y administrados por el componente VMware NSX-T como se explicará más adelante. Cada \textit{port group} informa de cuantas VMs y hosts ESXi tiene conectados.

% La configuración que se aplica a cada \textit{Distributed port group} descrito anteriormente es la siguiente:
% \begin{itemize}
%   \item \textit{Port binding}: permite indidcar como se gestionan los puertos de un \textit{port group} cuando se añade o elimina una VM. Tiene dos opciones de configuración, la primera se denomina \textit{Static Port Binding} y su función consiste en asignar un puerto dentro del \textit{port group} a la VM que se conecta y solo se elimina cuando la VM es borrada. La segunda opción se denomina \textit{Ephemeral Port Binding} y consiste en que el puerto se asigna a la VM cuando esta se enciende y se elimina cuando se apaga o elimina. Para los \textit{port groups} \textit{sddc-vds01-vsan} y \textit{sddc-vds01-vmotion} se configura la opción \textit{Static Port Binding} ya que así se asegura que las VMs se conectan siempre al mismo puerto lo cual permite mantener datos históricos y hacer monitoreo a nivel de puerto. Para los \textit{port group} \textit{sddc-vds01-mgmt}, \textit{sddc-edge-uplink01} y \textit{sddc-edge-uplink02} se configura la opción \textit{Ephemeral Port Binding} ya que, como el tráfico que circula por ellos es el que gestiona todos los componentes del SDDC y dan acceso a otras redes externas, se elimina la dependencia con el estado de VMware vCenter Server permitiendo que la comunicación continúe aunque VMware vCenter Server no se encuentre operativo.

%   \item \textit{Load Balancing}: indica como se distribuye el tráfico de salida de cada VM/host que se encuentran en el \textit{port group} entre las NICs físicas. Se selecciona \textit{Route based on physical NIC load}, es decir, el tráfico de una VM se transmite por una única NIC por lo que si esa NIC física está saturada, se asignará otra NIC física a la VM.
  
%   \item \textit{Network failure detection}: esta opción permite establecer como debe determinar el \textit{port group} que alguna de las NICs físicas está fuera de servicio. Se selecciona \textit{Link status only} para que esto se determine según el estado que le transmite la NIC física, así se pueden detectar los fallos que ocurren en la red física.
  
%   \item \textit{Notify switches}: se habilita para permitir a los host enviar \textit{frames} a los switches físicos para que estos conozcan la localización de las VM que están funcionando en cada host.
  
%   \item \textit{Failback}: permite determinar como se reactiva una NIC cuando esta se recupera de un fallo. Se habilita para establecer que la NIC se marcará como activa inmediatamente después de que se haya recuperado. Esta opción se debería desactivar en caso de que el estado de la NIC sea inestable.
  
%   \item \textit{Failover Order}: permite determinar que uplinks se deben utilizar, los que se seleccionan como \textit{active} son los que se utilizarán por defecto, los que se seleccionan como \textit{stand by} se usarán cuando los uplinks marcados como \textit{active} se encuentren desactivados. Se seleccionan las dos interfaces \textit{uplink} disponibles en el estado \textit{active}. Para el \textit{port group} \textit{sddc-edge-uplink01} se selecciona la interfaz \textit{uplink1} como activa y se deja sin usar la interfaz \textit{uplink2}, mientras que se configura de forma contraria en el \textit{port group} \textit{sddc-edge-uplink02}.
% \end{itemize}

\end{subsubsection}


\begin{subsubsection}{Diseño de la red del SDDC con VMware NSX-T}
    En el entorno de pruebas existe una red virtual que se define mediante software mantenida por VMware NSX-T, que al estar desacoplada de la infraestructura física se puede gestionar sin necesidad de modificar la configuración de la red física. Esto implica que se pueden aplicar diferentes configuraciones de red de forma sencilla, mejorando y simplificando su administración y seguridad. La virtualización de la red con VMware NSX-T se basa principalmente en el concepto de Segment:
    % La virtualización de la red con VMware NSX-T se basa en dos componentes, Transport Zone (TZ) y Segment
    % % proporcionando elasticidad y flexibilidad a la hora de administrar y obtener los recursos, tanto para el administrador como para el usuario final. 
    % En el Management Domain se despliega una instancia de NSX-T Manager y dos instancias del componente NSX-T Edge.
    % Para mantener la disponibilidad de VMware NSX-T y balancear su carga de trabajo, se despliegan tres instancias de NSX-T Manager Appliance, aunque para reducir el consumo de recursos, en el entorno de prueba solo se creará una instancia de este componente llamada \textit{nsx-mgmt-1}. También se despliegan dos instancias del componente VMware NSX-T Edge, llamadas \textit{edge01-mgmt} y \textit{edge02-mgmt}.
    \begin{itemize}
        % \item Transport Zone: se trata de un contenedor dentro del cual se definen Segments. A una TZ se conectan TNs\footnote{Los Transport Nodes son los hosts físicos y cada instancia de VMware NSX-T Edge.} para acceder a los Segments. Cada TN puede estar conectado a varias Transport Zones.
        \item Segment: se trata de un dominio de broadcast de capa 2 (una subred) al cual las VMs se conectan.
    \end{itemize}
    Un Segment se extiende por diferentes hosts los cuales pueden estar en la misma red a nivel físico o en distintas partes de la infraestructura. De esta forma, las VMs situadas en hosts con acceso a un Segment pueden conectarse a él y comunicarse de forma directa con otras VMs situadas en un host conectado a una red física diferente. Es decir, con un Segment se pueden comunicar diferentes puntos de la infraestructura sin cambiar la estructura de la red física, ya que VMware NSX-T se encarga de encapsular el tráfico al salir de un host cuando su destino se encuentra en una red física diferente a la de origen, haciendo creer al destinatario y al remitente que se encuentran en la misma subred (figura \ref{fig:frame-nsx-t}).
    \begin{figure}[h]
        \centering
        \includegraphics[width=0.6\textwidth]{imaxes/pruebaconcepto/frame-nsx-t.png}
        \caption{Ejemplo de cómo la comunicación entre dos VMs a través de un Segment es realizada transportando el tráfico a través de diferentes redes físicas.}
        \label{fig:frame-nsx-t}
      \end{figure}
    \FloatBarrier
    En el entorno de pruebas existen cuatro Segments, \textit{Mgmt-Region01A-VXLAN} y \textit{Mgmt-xRegion01-VXLAN}, ambos dedicados a dar acceso a la red a los componentes de VMware vRealize Suite y a las VMs desplegadas por los usuarios, y \textit{VCF-edge\_mgmt-cluster\_segment\_11} y \textit{VCF-edge\_mgmt-cluster\_segment\_12}, utilizados para dar salida hacia el router VyOS al tráfico que proviene de los Segments anteriores.

    Los encargados de gestionar el enrutamiento entre Segments y hacia la red externa son las instancias de NSX-T Edge. Para ello, internamente forman una estructura de routers virtuales que a parte de realizar las tareas de enrutamiento proporcionan servicios de red como NAT, Load Balancing, DNS, DHCP, VPN y Firewall, y mantienen rutas redundantes hacia el dispositivo físico de red.
    \begin{figure}[h]
        \centering
        \includegraphics[width=0.6\textwidth]{imaxes/pruebaconcepto/estructura_NSX_T.png}
        \caption{Segments a los que se conecta cada host del entorno y cómo estos acceden a la red física a través de las VMs de NSX-T Edge.}
        \label{fig:estructura-NSXT}
      \end{figure}
    \FloatBarrier
    \begin{figure}[h]
        \centering
        \includegraphics[width=0.6\textwidth]{imaxes/pruebaconcepto/vrealize/topology-vrops.png}
        \caption{Topología virtual de las redes virtuales construídas en VMware NSX-T.}
        \label{fig:two-tier-topology} 
    \end{figure}
    \FloatBarrier
    En la primera figura \ref{fig:estructura-NSXT}, se muestra como cada host se conecta a los dos Segments disponibles para que las VMs que se residen en ellos puedan acceder a la red física, en última instancia a través del vDS desplegado en el cluster de VMware vSphere. En la segunda figura \ref{fig:two-tier-topology}, se muestra la misma estructura pero desde el punto de vista interno de VMware NSX-T. En ella se aprecian los dos Segments donde uno de ellos tiene tres VMs conectadas (componentes de VMware vRealize Suite), y tres routers virtuales, dos de tipo Tier-1 y uno de tipo Tier-0. Los routers Tier-1 proporcionan servicios de red y enrutamiento entre Segments, \textit{services-Tier1} contiene un servidor DHCP  para las VMs que se conecten al Segment \textit{Mgmt-Region01A-VXLAN}, mientras que el router de tipo Tier-0 se encarga de dirigir el tráfico hacia la red física (router VyOS) a través de cuatro conexiones que se corresponden con las que se conectan al vDS que se muestra en la figura \ref{fig:estructura-NSXT}. Para que el router VyOS tenga conocimiento de las subredes virtuales/Segments existentes, las instancias de NSX-T Edge se las comunica mediante el protocolo de enrutamiento dinámico BGP.
    \\
    Usando VMware NSX-T el administrador puede gestionar y crear redes para ser consumidas por los usuarios de la plataforma. La creación de estas redes virtuales se hace bajo demanda y no requiere ninguna configuración adicional en los dispositivos de la red física. Su gestión se realiza siempre desde VMware NSX-T, el cual permite monitorizarlas, controlar su seguridad y establecer servicios dedicados. Además, permite extender una red virtual sobre diferentes redes físicas, permitiendo acceder a las VMs conectadas a esa red virtual desde diferentes puntos del SDDC, lo cual implica que una VM se puede migrar de una localización a otra para aumentar su disponibilidad sin necesidad de cambiar su configuración de red. En la infraestructura actual del CITIC todo esto no es posible ya que las redes que se crean dentro del entorno deben configurarse previamente sobre la red física, y todos los servicios de red necesarios deben ser proporcionados también desde dispositivos físicos, es decir, no existe actualmente en el CITIC una plataforma que permita gestionar las redes de la infraestructura de una forma dinámica y sin un alto coste en tiempo y riesgos.
    % Una TZ se extiende en diferentes hosts que pueden estar situados en la misma red a nivel físico, o en distintas partes de la infraestructura del SDDC. Los hosts que estén conectados a una TZ tendrán acceso a los Segments (cada Segment equivale a una subred) generados en esa TZ. Así, se hace posible la creación de redes accesibles desde cualquier parte de la infraestructura del SDDC sin necesidad de modificar los dispositivos de red físicos.
    % En el entorno, existen dos TZs. Una de ellas, la TZ \textit{mgmt-domain-m01-overlay-tz}, contiene dos Segments, \textit{mgmt-Region01A-VXLAN} y \textit{mgmt-xRegion01-VXLAN}, los cuales son utilizados para conectar las instancias de los componentes de VMware vRealize Suite. La otra TZ disponible, \textit{sfo01-m01-edge-uplink-tz} contiene otros dos Segments, utilizados para dar salida hacia el router VyOS al tráfico que circula por la TZ anterior (\textit{mgmt-domain-m01-overlay-tz}). Para que el tráfico de cada Segment pueda circular por la red física de la infraestructura, VMware NSX-T lo encapsula cuando sale de un host para que este pueda llegar al host destinatario.
    % Los encargados de gestionar el enrutamiento entre Segments y hacia la red externa son las instancias de NSX-T Edge. Para ello, internamente forman una estructura de routers virtuales que a parte de realizar las tareas de enrutamiento, proporcionan servicios de red como NAT, Load Balancing, DNS, DHCP, VPN y Firewall, y mantienen rutas redundantes hacia el dispositivo físico de red.

    
    
    % VMware NSX-T se encarga de encapsular el tráfico de estas redes virtuales 
    
    % Cuando el tráfico de un Segment debe salir de un TN a la red física para alcanzar su destino\footnote{Este paquete contiene la información de las VMs origen y destino que se están comunicando.}, este es encapsulado de nuevo en un paquete con la información de los TN origen y destino[Fig. \ref{fig:Frame-Segment-NSXT}]. Gracias a esta encapsulación, elementos que se encuentran en distintos entornos de la red física se pueden comunicar como si estuvieran directamente conectados el uno al otro. Así, es posible la creación de una misma red que se extienda por toda la infraestructura del SDDC, permitiendo comunicar componentes situados en distintas redes físicas, sin necesidad de modificar la configuración de los dispositivos físicos ni su topología. Esto hace necesario el uso de un protocolo de enrutamiento dinámico con BGP, tanto en la infraestructura física como en la red virtual, y así automatizar el proceso de configuración de nuevas redes virtuales.
    % El tipo de encapsulación que se realiza sobre el tráfico de los Segments se define la configuración de cada TZ, esta puede  ser de tipo VLAN o de tipo Overlay usando el protocolo Geneve.
    % \begin{itemize}
    %     \item TZ de tipo VLAN: se define una VLAN que se utilizará para identificar y encapsular el tráfico perteneciente a los Segments de una misma TZ. La VLAN que se defina debe estar configurada en la red física para que su tráfico sea aceptado.
    %     \item TZ de tipo Overlay Geneve: este protocolo se encarga de encapsular el tráfico saliente añadiendo una cabecera extra donde incluye un identificador. Cada Segment tendría su propio identificador.
    % \end{itemize}
    % \begin{figure}[h]
    %     \centering
    %     \includegraphics[width=0.8\textwidth]{imaxes/pruebaconcepto/Frame.png}
    %     \caption{Paquete de un Segment encapsulado cuando sale a la red física.}
    %     \label{fig:Frame-Segment-NSXT}
    % \end{figure}
    % \FloatBarrier
    % En la imagen anterior se muestra como se encapsula un paquete perteneciente a un Segment cuando este sale de un host/TN al medio físico. En las cabeceras del paquete correspondiente al Segment tendrá la información sobre las VMs origen y destino que se están comunicando, mientras que las cabeceras que encapsulan a ese paquete contienen la información sobre los hosts/TNs origen y destino donde se encuentran las VMs que se están comunicando. La dirección IP utilizada por los hosts/TNs para enviar el tráfico de un Segment encapsulado, se denomina Tunnel End-Point (TEP) y se asigna mediante DHCP para automatizar su configuración cuando un nuevo host/TN es añadido al entorno.

    % \begin{figure}[h]
    %     \centering
    %     \includegraphics[width=0.8\textwidth]{imaxes/pruebaconcepto/OverlayTZSegments.png}
    %     \caption{Segments de la Transport Zone \textit{mgmt-domain-m01-overlay-tz}}
    %     \label{fig:overlay-TZ-segments-NSXT}
    %   \end{figure}
    %   \FloatBarrier

    %   En la imagen anterior se muestra la Transport Zone de tipo Overlay, definida en el entorno de pruebas, con el nombre \textit{mgmt-domain-m01-overlay-tz}. A ella se conectan los seis TNs\footnote{Los seis TNs son los cuatro hosts y las dos instancias de NSX-T Edge}\footnote{Un TN utiliza el elemento NSX-T Virtual Distributed Switch (N-VDS) para conectarse a los Segments de una TZ} y contiene tres Segments. El Segment \textit{mgmt-xRegion01-VXLAN} se utiliza para desplegar aplicaciones cuyas instancias deben ser accesibles desde cada Region del SDDC. El Segment \textit{mgmt-Region01A-VXLAN} tiene como finalidad alojar aplicaciones que solo deben ser accesibles desde dentro de una misma Region. El Segment \textit{sddc-host-overlay} es utilizado por los componentes de VMware NSX-T para comunicarse con los TNs. Con cada Segment de tipo Overlay se genera un \textit{port group} con el mismo nombre en el vDS (se puede ver en la figura \ref{fig:port-groups-vSwitch-vSphere}) que se utilizan para transmitir su tráfico a la red física.
    % %    VMware NSX-T genera en el vSphere vSwitch un \textit{port group} por cada \textit{segment} para poder conectar la VM de cada componente al \textit{port group} que le corresponda.

    % \begin{figure}[h]
    %     \centering
    %     \includegraphics[width=0.8\textwidth]{imaxes/pruebaconcepto/VLANTZSegments.png}
    %      \caption{Segments de la  Transport Zone \textit{sfo01-m01-edge-uplink-tz}}
    %     \label{fig:VLAN-TZ-segments-NSXT}
    % \end{figure}
    % \FloatBarrier
    % En la imagen anterior se muestra la Transport Zone de tipo VLAN, definida en el entorno de pruebas, con el nombre \textit{sfo01-m01-edge-uplink-tz}. A ella se conectan los dos TNs que son instancias de NSX-T Edge, \textit{edge01-mgmt} y \textit{edge02-mgmt}. Contiene dos Segments, \textit{VCF-edge-mgmt-cluster-segment-11} y \textit{VCF-edge-mgmt-cluster-segment-12}, que son utilizados por las instancias de NSX-T Edge para transmitir el tráfico de todas las redes virtuales gestionadas por VMware NSX-T hacia redes físicas externas al SDDC. Para ello utilizan utilizan los \textit{port groups} \textit{sddc-edge-uplink01} y \textit{sddc-edge-uplink02} (se pueden ver en la figura \ref{fig:port-groups-vSwitch-vSphere}) del vDS para transmitir su tráfico hacia las interfaces de red físicas de cada host. Ambas instancias forman la topología de red, que se muestra en la siguiente imagen, para comunicar las redes virtuales de VMware NSX-T con el router físico.
    % \begin{figure}[h]
    %     \centering
    %     \includegraphics[width=0.4\textwidth]{imaxes/pruebaconcepto/UplinkDesign.png}
    %     \caption{Topología de red de las insterfaces \textit{uplink}.}
    %     \label{fig:Uplink-Design-Edge-NSXT} 
    % \end{figure}
    % \FloatBarrier
    % Como se puede ver en la imagen anterior, los dos Segments son utilizados por las instancias de NSX-T Edge para mantener rutas redundantes hacia la red externa y así aumentar su disponibilidad. A parte, este componente también se encarga de proporcionar un conjunto de servicios de red a los componentes que están conectados a los Segments de VMware NSX-T. Para entregar estos servicios, internamente, VMware NSX-T forma una topología con una serie de routers virtuales.
    % \begin{figure}[h]
    %     \centering
    %     \includegraphics[width=0.4\textwidth]{imaxes/pruebaconcepto/topologiaTwoTierRouting-Final.png}
    %     \caption{Topología virtual de VMware NSX-T}
    %     \label{fig:two-tier-topology} 
    % \end{figure}
    % \FloatBarrier
    % En la imagen anterior se muestra la topología que forman VMware NSX-T para proporcionar acceso a la red externa y para entregar otros servicios a los componentes situados en los dos Segments existentes. En esta topología hay tres routers, \textit{mgmt-domain-tier0-gateway}, \textit{mgmt-domain-tier1-gateway} y \textit{mgmt-domain-lb01-tier1-gateway}, de los cuales, el primero se encarga de gestionar la comunicación con los dispositivos de red físicos, y los dos restantes se encargan de enrutar el tráfico entre Segments y hacia el router de Tier 0, y de entregar distintos servicios a los componentes que residen en los Segments. Los servicios de red que se pueden habilitar en los dos routers de Tier 1 son NAT, Load Balancing, DNS y VPN.
\end{subsubsection}
\end{subsection}
\begin{subsection}{Operaciones de la Arquitectura}
    En este punto ya se ha formado el SDDC, la configuración de la infraestructura física y de todos los componentes de VCF está preparada para desplegar la plataforma que habilite el servicio Cloud. Este último paso se completará con los servicios que proporciona VMware con los productos que agrupa en VMware vRealize Suite, estos proporcionarán un servicio de autenticación centralizado para los usuarios y servicio de aprovisionamiento de recursos. Se utilizarán tres componentes de VMware vRealize Suite, estos son vRealize Suite Lifecycle Manager (vRSLCM), Workspace One Access (WSA) y vRealize Automation (vRA). El despliegue de estos servicios dentro del entorno será iniciado desde el componente SDDC Manager y, para aprovechar las ventajas de VMware NSX-T y las redes virtuales existentes, utilizarán como red de acceso el Segment \textit{mgmt-xRegion01-VXLAN}.

    % El entorno ya está configurado para funcionar como un SDDC, a partir de este punto ya no es necesario realizar ninguna modificación en la infraestructura física ya que todas las tareas que se deben realizar están dentro del alcance de los componentes de VMware Cloud Foundation. Para finalizar la construcción del SDDC y habilitar un servicio donde los usuarios puedan aprovisionar recursos bajo demanda, se instalarán sobre el entorno desplegado las aplicaciones Workspace ONE Access \footnote{VMware vRealize Identity Manager} (WSA) y VMware vRealize Automation (vRA). La primera permite al administrador conectar con el servidor de usuarios Active Directory y gestionarlos para proveer un servicio de autenticación centralizado a múltiples aplicaciones como VMware vRealize Automation. La segunda aplicación permite a los usuarios aprovisionar recursos de forma automatizada desde un catálogo de recursos. VMware vRealize Suite Licfecycle Manager (vRSLCM) es el componente que permite administrar vRA y WSA, su instalación y actualizaciones, las contraseñas de administrador y sus certificados, para ello necesita comunicarse con VMware vCenter Server. Se desplegará una instancia de cada componente en el \textit{management domain} creado anteriormente y estarán conectadas al \textit{segment}/subred \textit{Mgmt-xRegion01-VXLAN}.
    
    \begin{subsubsection}{vRealize Suite Lifecycle Manager}
        vRSLCM es el primer componente que se instala (este proceso se hace desde SDDC Manager) ya que es el encargado de gestionar todo lo relacionado con el resto de productos de VMware vRealize Suite.  Como su nombre indica, su función es gestionar el ciclo de vida de los servicios de VMware vRealize Suite en el SDDC, incluyendo su despliegue, actualizaciones y gestión de las credenciales de administración, certificados y licencias, por lo tanto, este componente permite al administrador controlar de forma centralizada la configuración y seguridad de los servicios dedicados a las operaciones del SDDC. Para llevar a cabo sus funciones, vRSLCM debe mantener una comunicación con la instancia de VMware vCenter Server desplegada en el Management Domain.
        % este componente está dedicado a mantener su seguridad y configuración, y a controlar que servicios se encuentran en el entorno, todo para simplificar y facilitar las tareas del administrador. Para llevar a cabo sus funciones, vRSLCM debe mantener una comunicación con la instancia de VMware vCenter Server desplegada en el Management Domain.
        \begin{figure}[h]
            \centering
            \includegraphics[width=0.4\textwidth]{imaxes/pruebaconcepto/vrealize/diseno-vrlscm.png}
            \caption{Componentes con los que se comunica vRSLCM.}
            \label{fig:vrealize-components}
        \end{figure}
        \FloatBarrier        
        Durante el despliegue de WSA y vRA, desde vRSLCM se establece su configuración, indicando la licencia, credenciales del administrador, direcciones IP, configuración DNS y NTP, y certificados para que los usuarios puedan acceder de forma segura a través del navegador\footnote{El certificado de cada aplicación es generado manualmente desde la CA y luego subido a vRSLCM, que en este caso es la VM con Windows Server 2016.}. Además, se debe elegir la ubicación donde se van a desplegar las VMs de estos servicios, es decir, el dominio de VMware vCenter Server, el cluster vSphere, la red y el datastore para el almacenamiento.
        En el entorno de pruebas, de cada servicio se crea una instancia en el Management Domain, se colocan en el cluster vSphere (\nameref{subsubsec:diseno-vsphere}), utilizan el datastore de VMware vSAN (\nameref{subsubsec:diseno-vsan}) y están controladas por la instancia de VMware vCenter Server (\nameref{subsubsec:diseno-vcenter}). Como ya se ha mencionado, las instancias se conectan a un Segment controlado por VMware NSX-T (como se muestra en la figura \ref{fig:two-tier-topology}) para poder hacer uso de sus servicios de red.
        \begin{figure}[h]
            \centering
            \includegraphics[width=0.6\textwidth]{imaxes/pruebaconcepto/vrealize/config-istance-vridm.png}
            \caption{Apartado donde se muestra la configuración de la instancia de WSA en vRSLCM.}
            \label{fig:config-WSA}
        \end{figure}
        \FloatBarrier 
        
        % Dentro de vRSLCM, los despliegues son organizados por entornos pudiendo crearse un entorno por cada servicio o un entorno con todos los servicios. Existen dos modos de despliegue, uno donde se crean tres instancias del servicio para balancear la carga\footnote{El balanceo de la carga se realiza con el servicio de Load Balancing de VMware NSX-T.}, y otro donde solo se crea una instancia.

        % En el entorno, se despliega una instancia de cada servicio. Estas están colocadas bajo el dominio de la instancia de VMware vCenter en el cluster vSphere

        % Este se instala desde SDDC Manager, pero una vez instalado es utilizado para desplegar cualquier servicio de VMware vRealize Suite.
        
    \end{subsubsection}

    \begin{subsubsection}{Workspace One Access}
        WSA es el componente de VMware vRealize Suite que se integra con un directorio de usuarios para proporcionarles acceso a las aplicaciones que se despliegan en el entorno, vRA en este caso. Así, en el entorno de producción, WSA se utilizaría para que los usuarios del SDDC se pudieran conectar utilizando sus credenciales de la UDC, ya que estaría conectado al directorio de la universidad.        
        En el entorno de pruebas, se integra con el directorio de usuarios Active Directory situado en la VM con Windows Server 2016. Este Active Directory contiene perfiles de usuarios y grupos de usuarios organizados en unidades organizativas, los perfiles de usuario se añaden a grupos de usuarios, y la creación y mantenimiento de sus credenciales se realiza desde el propio Active Directory. Desde WSA se seleccionan los usuarios y grupos de usuarios que se quieran sincronizar para que estén disponibles en el SDDC y, posteriormente, desde cada aplicación se determina el nivel de acceso y permisos para cada usuario. Como norma general se deben asignar permisos a grupos de usuarios y no a perfiles individuales, de esta forma se reduce el tiempo de gestión y se simplifica la estructura, ya que para asignar nuevos permisos a un usuario solo sería necesario añadirlo al grupo correspondiente.        
        En Active Directory se han configurado varios usuarios que se sincronizan en WSA. Estos se utilizarán para mostrar las funcionalidades de vRA como si se tratase del entorno de producción con perfiles de usuarios de la UDC.
        \begin{figure}[h]
            \centering
            \includegraphics[width=0.6\textwidth]{imaxes/pruebaconcepto/vrealize/users-AD.png}
            \caption{Usuarios dentro de Active Directory}
            \label{fig:users-defined-AD}
        \end{figure}        
        Desde WSA se seleccionan aquellas unidades organizativas que se quieren sincronizar. Cada unidad contiene usuarios y grupos de usuarios, los usuarios que harán uso del servicio de aprovisionamiento están colocados en la unidad CITIC. 
        \begin{figure}[h]
            \centering
            \includegraphics[width=0.6\textwidth]{imaxes/pruebaconcepto/vrealize/syncing-users.png}
            \caption{Sincronización de usuarios desde Workspace One Access.}
            \label{fig:users-defined-WSA}
        \end{figure}
        \FloatBarrier
        \begin{figure}[h]
            \centering
            \includegraphics[width=0.6\textwidth]{imaxes/pruebaconcepto/vrealize/users-wsa.png}
            \caption{Usuarios sincronizados en Workspace One Access.}
            \label{fig:users-defined-WSA}
        \end{figure}
        \FloatBarrier
        El acceso a las aplicaciones está centralizado a través de una plataforma proporcionada por WSA. Cuando el usuario intenta acceder a vRA este es redirigido a una página web donde introduce sus credenciales, WSA comprueba los datos introducidos y vuelve a redirigir al usuario a la plataforma de vRA. Además, utilizando esta plataforma el administrador puede obtener estadísticas sobre que usuarios se autentican, a que servicios acceden y desde donde lo hacen. WSA permite al administrador editar la interfaz de la web de autenticación, pudiendo personalizar el icono y nombres que se muestran, y modificar los parámetros que se utilizan para autenticarse o mantener las sesiones, permitiendo definir si el usuario debe utilizar su cuenta de correo electrónico o nombre de usuario y si se utilizan cookies de sesión o persistentes. Por si esto fuera poco, también existe la posibilidad de crear políticas para controlar como se autentican los usuarios, desde donde pueden hacerlo y el tiempo de duración de la sesión. En la siguiente figura se muestran las reglas de la política por defecto que se aplica a los usuarios, se permite el acceso desde cualquier dirección IP, a través de un navegador web o la aplicación para dispositivos móviles Workspace One App, usando su contraseña y con un tiempo de sesión de 2160 u 8 horas dependiendo del punto de acceso.
        \begin{figure}[h]
            \centering
            \includegraphics[width=0.6\textwidth]{imaxes/pruebaconcepto/vrealize/default-policy.png}
            \caption{Política de autenticación por defecto.}
            \label{fig:default-policy}
        \end{figure}
        \FloatBarrier
        
        \begin{figure}[h]
            \centering
            \includegraphics[width=0.3\textwidth]{imaxes/pruebaconcepto/vrealize/wsa-login.png}
            \caption{Plataforma de autenticación de Workspace One Access}
            \label{fig:wsa-platform}
        \end{figure}
        \FloatBarrier        
        Con WSA el administrador tiene un mayor control sobre los usuarios y como estos acceden a los servicios de la plataforma Cloud, ya que gestiona desde un único punto todas las cuentas de usuarios y la seguridad del punto de acceso, pudiendo controlar que usuarios acceden y estableciendo medidas seguridad de forma sencilla e intuitiva. También, la gestión de las credenciales de cada usuario se separa de la gestión del acceso, ya que lo primero está controlado por Active Directory y lo segundo por WSA. De esta forma, la seguridad del entorno aumenta y las tareas del administrador se simplifican. Con esta plataforma se soluciona uno de los problemas de la infraestructura del CITIC, ya no es necesario crear cuentas manualmente para cada usuario que quiera acceder al servicio y su gestión se hace más dinámica, a la vez que se aumenta el nivel de seguridad.

        % Los usuarios que necesiten acceder a vRA deben estar registrados en el directorio de Workspace One Access. Este componente centraliza el acceso de todos los productos de VMware vRealize. Cuando se despliega se debe configurar un Active Directory que en el caso del entorno está situado en la VM con Windows Server 2016. Dentro del Active Directory existen grupos de seguridad y perfiles de usuario, un perfil de usuario contiene información como nombre, apellidos, dirección e-mail, nombre de usuario y contraseña\footnote{Se pueden configurar más campos pero los que se describen son los obligatorios a la hora de crear un usuario.}, y este puede formar parte de varios grupos de seguridad. Una vez configurado, cada aplicación se conectará a WSA y se podrán asignar roles para los grupos de seguridad y usuarios estableciendo así un nivel de acceso. Además, cada usuario registrado tendrá disponible un catálogo de aplicaciones en el portal de WSA cuyo administrador establecerá que aplicaciones están habilitadas para cada usuario o grupo, eso sí, para que el usuario pueda acceder a ella previamente se debe establecer un rol para ese usuario dentro de la aplicación.

        %*****USUARIOS QUE HAY EN EL ENTORNO Y EL ACCESO A CADA APLICACIÓN*******%
        % \begin{figure}[h]
        %     \centering
        %     \includegraphics[width=0.4\textwidth]{imaxes/vRealize_pruebaconcepto/usuariosDefinidos.png}
        %     \caption{Muestra los usuarios definidos en el Active Directory sincronizados en Workspace One Access.}
        %     \label{fig:users-defined-AD}
        % \end{figure}
        % \FloatBarrier
        % En la Figura \ref{fig:users-defined-AD} se muestran los dos usuarios definidos en el Active Directory y dos usuarios que se corresponden a los perfiles de administración de WSA, no se utilizarán grupos de seguridad para reducir la complejidad pero su configuración en las aplicaciones de VMware es igual que para los perfiles de usuario. En un entorno real existen usuarios que controlan a otros usuarios y establecen su nivel de acceso, a parte de los perfiles de administrador de cada aplicación. Para el entorno se define el perfil \textit{adminuser} que será el encargado de gestionar el acceso de dos usuarios (\textit{baseuser1} y \textit{baseuser2}) que serán los que consuman a las aplicaciones desplegadas (vRSLCM y vRA). El primero tendrá acceso y permisos de edición en las aplicaciones vRSLCM y vRA, mientras que los dos usuarios base solo podrán acceder a vRA y dentro de este el usuario admin definirá que servicios están habilitados para cada uno.

        %************************************************************************%

    \end{subsubsection}

    \begin{subsubsection}{VMware vRealize Automation}
        VMware vRealize Automation es el componente de VMware vRealize Suite que automatiza el aprovisionamiento de recursos del SDDC. Con esta plataforma los usuarios podrán elaborar diseños de los recursos que necesitan para posteriormente implementarlos y llevar a cabo sus trabajos. El proceso de implementación es automático y transparente para el usuario que puede modificar los requisitos de sus recursos según sea necesario. Al mismo tiempo, el administrador establece los límites del diseño/implementación y debe proveer en la infraestructura los medios que se usarán como base para los diseños.
        En vRA el aprovisionamiento de recursos se realiza a partir de diseños realizados por el usuario. Estos diseños, llamados Blueprints, son archivos con formato .yaml en los que se especifican los recursos que el usuario quiere obtener, estos recursos son VMs, redes y almacenamiento, y para cada recurso definido el usuario puede especificar sus características como el tamaño de una VM y su configuración interna (sistema operativo, instalación de librerías o servicios, redes a las que se conecta). Las blueprints se definen dentro de proyectos en los cuales existen jefes de proyecto y usuarios que utilizan o aprovisionan los recursos definidos en cada blueprint, de esta forma, tanto el administrador como el jefe de proyecto controlan qué usuarios acceden a los recursos de un proyecto. Además, el administrador será el encargado de limitar la cantidad de recursos disponibles para cada proyecto, y de establecer los recursos disponibles para el proyecto, es decir, qué sistemas operativos están disponibles, en qué redes se podrán realizar los despliegues y que datastore utilizarán para el almacenamiento.
        \begin{figure}[h]
            \centering
            \includegraphics[width=0.7\textwidth]{imaxes/vRealize_pruebaconcepto/ComponentesVRA.png}
            \caption{Uso y componentes de VMware vRealize Automation.}
            \label{fig:vra-components}
        \end{figure}
        \FloatBarrier
        Como se muestra en la figura anterior, en vRA existen tres componentes principales que son Service Borker, a través del cual los usuarios tienen acceso al catálogo de diseños disponibles en su proyecto para implementarlos, Cloud Assembly, desde donde se elaboran los diseños, automatiza el aprovisionamiento ordenado desde Service Broker y administra la infraestructura disponible, y Cloud Zone, punto desde donde vRA accede a la infraestructura para obtener los recursos, situada en este caso en el cluster vSphere del Management Domain.

        Para ordenar el aprovisionamiento de los recursos, antes de realizar cualquier implementación, es necesario configurar la infraestructura que se va a poner a disposición de los usuarios. 
        En VMware vCenter Server, dentro del cluster vSphere se define un \textit{resource pool} y una carpeta que se utilizarán para colocar las VMs que se desplieguen desde vRA. 
        La red que utilizarán las VMs generadas estará controlada por VMware NSX-T para aprovechar las ventajas de sus redes definidas por software y así automatizar su configuración y hacer uso de los servicios que ofrece. Para ello se añade un Segment al router virtual de Tier-1 que se muestra en la figura \ref{fig:two-tier-topology}, VMware NSX-T informa de la nueva ruta al router físico mediante BGP proporcionando así acceso a redes externas.
        \begin{figure}[h]
            \centering
            \includegraphics[width=0.4\textwidth]{imaxes/pruebaconcepto/vrealize/topology-for-vRA-NSXT.png}
            \caption{Nuevo Segment \textit{vra-deployments} en el router virtual Tier-1.}
            \label{fig:topology-nsx-t-vra}
        \end{figure}
        \FloatBarrier
        Las VMs que los usuarios generan están basadas en plantillas que son creadas previamente por el administrador. Antes de generar la plantilla el administrador debe crear una VM en VMware vCenter Server y proveerla con una configuración mínima para que el usuario pueda aplicar su propia configuración durante el despliegue, para este entorno se crea una VM con el sistema operativo Ubuntu Server 20.04.1. Una vez se termina el proceso de instalación y actualización del sistema operativo, se configura el servicio \textbf{cloud-init}\footnote{Se puede encontrar más información sobre cloud-init en el enlace: \url{https://cloudinit.readthedocs.io/en/latest/}} el cual permite inicializar la VM con la configuración indicada en la blueprint, como se verá más adelante. También se pueden preinstalar servicios como bases de datos para que el usuario solo tenga que configurarlo. Cuando la configuración de la VM está lista, se limpia limpian el hostname, archivos de logs, claves SSH y caché del servicio cloud-init ejecutando el siguiente script, para que en cada implementación se genere una VM distinta a partir de la misma plantilla. Finalmente, la VM se convierte a una plantilla que se almacena en VMware vCenter Server.
        \begin{figure}[h]
            \centering
            \includegraphics[width=0.4\textwidth]{imaxes/pruebaconcepto/vrealize/install-cloud-init.png}
            \caption{Configuración del servicio cloud-init en Ubuntu Server 20.}
            \label{fig:cloud-init-config}
        \end{figure}
        \FloatBarrier
        \begin{figure}[h]
            \centering
            \includegraphics[width=0.4\textwidth]{imaxes/pruebaconcepto/vrealize/config-istance-vridm.png}
            \caption{Creación de una plantilla a partir de la VM de Ubuntu Server 20.}
            \label{fig:template-ubuntu}
        \end{figure}
        \FloatBarrier

        Una vez se tienen los recursos que serán usados por los usuarios es necesario añadirlos a vRA. Para poder esos recursos, durante su configuración vRA se asigna un tag a cada uno con la forma \textit{key:value}. En una blueprint estos tags permitirán indicar a que recurso se está refiriendo. 
        Lo primero es configurar una Cloud Zone que proporcionará acceso a todos los recursos situados en el cluster vSphere. El tag utilizado para esta será \textit{cloud:private}. Posteriormente se definen los tamaños de VMs que se habilitan, también llamado Flavour Mapping. Se configuran cuatro tamaños, X-small (1 CPU y 512 MB de RAM), Small (2 CPU y 8 GB de RAM), Medium (8 CPU y 4 GB de RAM) y Large (8 CPU y 16 GB de RAM). Las redes disponibles se añaden mediante la creación de perfiles donde se definen los tags con los que se identifica la red, su gateway, su servidor DNS, el rango de IPs para asignar una IP a cada VM conectada a esa red. 
        % Internamente vRA se divide en tres componentes, Cloud Assembly, Service Broker y Cloud Zone.


        % El punto a través del cual los usuarios pueden aprovisionar sus recursos es vRealize Automation. Este producto provee el servicio cloud. 
        % \begin{figure}[h]
        %     \centering
        %     \includegraphics[width=0.8\textwidth]{imaxes/vRealize_pruebaconcepto/ComponentesVRA.png}
        %     \caption{Componentes de VMware vRealize Automation y tareas que realiza cada rol de usuario.}
        %     \label{fig:vra-components}
        % \end{figure}
        % \FloatBarrier
        % Internamente vRA se divide en varios servicios que permiten gestionar los diferentes aspectos de la cloud. Para centrarse en los objetivos de este proyecto solo se hace referencia a dos de esos servicios, el primero es Cloud Assembly el cual permite administrar la infraestructura disponible controlar el uso que se hace de esos recursos, y el segundo es Service Broker, utilizado por los usuarios para aprovisionar los recursos desde un catálogo de plantillas. La obtención de los recursos por parte del usuario se hace desplegando una serie de plantillas llamadas Blueprints diseñadas previamente, en donde se define un conjunto de VMs y recursos de red y de almacenamiento incluyendo otros aspectos como la configuración de cada uno de los recursos, como redes de la infraestructura que se utilizan, cantidad de almacenamiento, o la ubicación del despliegue en la infraestructura. Son ficheros de código con extensión \textit{.yaml} donde se indican etiquetas, aunque también se pueden diseñar con un editor gráfico. Estas plantillas están relacionadas con proyectos, una plantilla pertenece a uno o varios proyectos donde existe un coordinador de proyecto que se encarga de diseñar Blueprints y de administrar los usuarios miembros de ese proyecto. Los proyectos de vRA permiten limitar los recursos para que un conjunto de usuarios pueda desplegar los componentes definidos en las Blueprints disponibles, como la cantidad de memoria RAM, cantidad de instancias que se pueden desplegar y cantidad de almacenamiento, también aquellas redes que se pueden utilizar. Desde el punto de vista de vRA, la infraestructura se divide en Cloud Zones, las cuales son conjuntos de recursos situados en distintos proveedores Cloud que pueden ser públicos como AWS o Azure, o privados que solo pueden ser clusters vSphere. En el caso del entorno desplegado solo se tendrá una única Cloud Zone de tipo vSphere. En cada Cloud Zone se define como se deben distribuir los recursos aprovisionados sobre la infraestructura. 
        % Finalmente será el administrador de la infraestructura el que se encargue de proveer los recursos, administrar los proyectos disponibles, gestionar los coordinadores de cada proyecto y controlar y limitar el uso de los recursos.
        % Finalmente, vRA permite configurar tarjetas donde se puede definir el coste del aprovisionamiento de CPU, almacenamiento y memoria RAM, además del coste de uso de otros elementos como sistemas operativos, el uso de una determinada red o el uso de una determinada Cloud Zone. Estas tarjetas se asignan por proyecto para determinar el coste que tendrá el consumo de recursos por mes.


    \end{subsubsection}

    
\end{subsection}
\begin{subsection}{Servicio Cloud}
\label{subsec:plataforma-cloud}
    
    % A lo largo de esta sección se describe la configuración y el funcionamiento de VMware vRealize Automation con el fin de mostrar las sus características y las mejoras que implica su implementación. Primero se habilitarán los recursos necesarios dentro del servicio para que luego los usuarios hagan uso de ellos mediante la creación de dos proyectos e implementación de diseños. 

    \begin{subsubsection}{Preparación de los recursos}
    El aprovisionamiento de recursos con vRA se traduce en la creación de VMs a partir de plantillas creadas previamente por el administrador del SDDC a las que se les aplica una configuración determinada, y al uso de las subredes y almacenamiento disponibles en la infraestructura. 
    \\
    Con el objetivo de organizar las VMs creadas por los usuarios, en el cluster vSphere del entorno de pruebas se crean una carpeta y un \textit{resource pool} donde se colocarán las nuevas VMs que desplieguen los usuarios, como se muestra en la siguiente figura.
    \begin{figure}[h]
        \centering
        \includegraphics[width=0.6\textwidth]{imaxes/pruebaconcepto/vrealize/rp-vra.png}
        \caption{Resource pool (izquierda) y carpeta (derecha) creadas para alojar las VMs desplegadas desde vRA.}
        \label{fig:rp-folder-vra}
    \end{figure}
    \FloatBarrier
    % Para que las VMs creadas por los usuarios tengan acceso a la red, se crea un nuevo Segment en el router virtual Tier-1 de VMware NSX-T (Figura \ref{fig:two-tier-topology}) donde se alojará una subred dedicada exclusivamente a ser consumida por los usuarios. Al generar el Segment los componentes de VMware NSX-T comunican al router VyOS la nueva ruta mediante el protocolo de enrutamiento BGP, por lo tanto no es necesario aplicar ninguna configuración adicional en los recursos de red físicos.
    La red utilizada para dar acceso a las VMs creadas por los usuarios es el Segment \textit{Mgmt-Region01A-VXLAN} disponible en VMware NSX-T\footnote{Figura \ref{fig:two-tier-topology}}, el cual cuenta con un servidor DHCP\footnote{Como se observa en la figura \ref{fig:topology-segment-mgmt}, el servidor DHCP está gestionado por VMware NSX-T ya que forma parte de sus servicios de red.} y así poder establecer la configuración IP de forma automática de cada nueva VM que se conecte a este Segment. 
    \begin{figure}[h]
        \centering
        \includegraphics[width=0.7\textwidth]{imaxes/pruebaconcepto/vrealize/segment-MGMT.png}
        \caption{Segment utilizado para el despliegue de VMs con vRA (arriba) y la configuración del servidor DHCP definida en VMware NSX-T (abajo)}
        \label{fig:topology-segment-mgmt}
    \end{figure}
    \FloatBarrier
    % \begin{figure}[h]
    %     \centering
    %     \includegraphics[width=0.4\textwidth]{imaxes/pruebaconcepto/vrealize/router-vyos-bgp.png}
    %     \caption{Nueva ruta configurada en el router VyOS mediante BGP.}
    %     \label{fig:bgp-router-vyos}
    % \end{figure}
    % \FloatBarrier   
    Para que los usuarios tengan plantillas a partir de las cuales generar sus propias VMs, el administrador del SDDC debe crearlas antes. Este proceso consiste en crear una VM, inicializarla con la instalación de un sistema operativo y establecer una configuración base para finalmente generar una plantilla. En el entorno de pruebas se crean dos plantillas de dos sistemas operativos distintos desde VMware vCenter Server, una con Windows Server 2016 (figura \ref{fig:windows-server-installing}) y otra con CentOS 8 (figura \ref{fig:centos-installing}). Una vez instalados ambos sistemas se habilita al menos un método de acceso, SSH en el caso de CentOS y RDP en Windows Server, y se instala el servicio \textbf{cloud-init}\footnote{Ejemplos de uso y su documentación se pueden encontrar en el siguiente enlace: \url{https://cloudinit.readthedocs.io/en/latest/topics/examples.html}.} el cual permitirá a los usuarios finales ejecutar comandos de configuración durante el despliegue de una VM para adaptarla a sus requisitos. En sistemas operativos Windows este servicio se llama \textbf{cloudbase-init}\footnote{Su documentación se puede encontrar en el siguiente enlace:\url{https://cloudbase.it/cloudbase-init/}.}. Una vez se ha completada la configuración se deben ejecutar una serie de comandos, que en el caso de Windows Server son ejecutados directamente por el instalador de cloudbase-init a través del servicio \textbf{sysprep}, para limpiar el sistema y así generar una VM única cada vez que el usuario final utiliza la plantilla. Esto incluye el borrado de paquetes obsoletos y limpieza de logs, claves SSH e identificadores del sistema como direcciones MAC. Una vez se ha completado el proceso se genera una plantilla de cada VM (figura \ref{fig:templates}).
    % Las plantillas empleadas por los usuarios para generar VMs son generadas por el administrador del SDDC. Para crear una plantilla el administrador debe antes crear una VM, instalar en ella el sistema operativo deseado y establecer una configuración inicial. En el entorno de pruebas, dentro del cluster vSphere, se crea una VM con el sistema operativo Ubuntu Server 18.04, para inicializarla se instalan las actualizaciones correspondientes y se configura el servicio \textbf{cloud-init}, el cual permitirá al usuario especificar comandos en el diseño para inicializar automaticamente una VM con los requisitos que desee (instalación de paquetes, creación de usuarios, generación de claves SSH, configuración de red y mucho más\footnote{En el siguiente enlace se pueden encontrar más información sobre cloud-init y ejemplos sobre sus usos: \url{https://cloudinit.readthedocs.io/en/latest/topics/examples.html}}). Una vez configurada se procede a ejecutar un script\footnote{El script se puede encontrar en el anexo .} para limpiar la VM para que cada vez que se utilice la plantilla se genere una VM distinta. Finalmente la VM se convierte a una plantilla que se almacena en VMware vCenter Server. 
    % Siguiendo un procedimiento similar se crea una plantilla a partir de una VM con Windows Server 2016 y otra plantilla con CentOS 8.
    \begin{figure}[h]
        \centering
        \includegraphics[width=1\textwidth]{imaxes/pruebaconcepto/vrealize/instalador-windows.png}
        \caption{Instalación y preparación de la VM con Windows Server 2016 para la creación de una plantilla}
        \label{fig:windows-server-installing}
    \end{figure}
    \FloatBarrier
    \begin{figure}[h]
        \centering
        \includegraphics[width=1\textwidth]{imaxes/pruebaconcepto/vrealize/centos-installation.png}
        \caption{Instalación CentOS y comandos ejecutados para la creación de una plantilla.}
        \label{fig:centos-installing}
    \end{figure}
    \FloatBarrier
    \begin{figure}[h]
        \centering
        \includegraphics[width=0.4\textwidth]{imaxes/pruebaconcepto/vrealize/plantillas-creadas-vcenter.png}
        \caption{Plantillas de CentOS 8 y Windows Server 2016 creadas a partir de sus respectivas VMs.}
        \label{fig:templates}
    \end{figure}
    \FloatBarrier
    % Se crea otra VM con el sistema operativo Windows Server 2016. En este caso, en lugar de cloud-init se utiliza el servicio \textit{cloudbase-init}\footnote{La documentación de cloudbase-init se puede encontrar aquí: \url{https://cloudbase.it/cloudbase-init/}} que cumple la misma función que el anterior. Una vez completada la instalación y configuración de Windows Server 2016, desde VMware vCenter Server se convierte la VM en una plantilla.
    % \begin{figure}[h]
    %     \centering
    %     \includegraphics[width=0.6\textwidth]{imaxes/pruebaconcepto/vrealize/instalación-cloudbase-windows.png}
    %     \caption{instalación de cloudbase-init en Windows Server 2016.}
    %     \label{fig:cloudbase-init}
    % \end{figure}
    % \FloatBarrier 

    \end{subsubsection}

    \begin{subsubsection}{Configuración de VMware vRealize Automation}
        Para que los recursos de cómputo, red y almacenamiento de la infraestructura sean consumidos por los usuarios, es necesario habilitarlos en la plataforma de vRA. A medida que se integra cada recurso se le asigna uno o más tags para poder identificarlo y que el usuario lo pueda incluir en sus diseños.
        \\ 
        En la siguiente figura se muestran las plantillas creadas anteriormente en VMware vCenter Server. Cada vez que un usuario quiera crear una VM deberá indicar a partir de qué plantilla quiere generarla.
        \begin{figure}[h]
            \centering
            \includegraphics[width=0.7\textwidth]{imaxes/pruebaconcepto/vrealize/image-mappings.png}
            \caption{Plantillas de CentOS 8 y Windows Server 2016 disponibles en vRA.}
            \label{fig:image-mapping}
        \end{figure}
        \FloatBarrier
        Para habilitar el Segment \textit{Mgmt-Region01A-VXLAN}, en vRA se crea un perfil de red y dentro de este se añade la subred deseada. A esta se le asignan los tags \textit{subnet-cidr:10.50.0.0/24}, \textit{function:pro} y \textit{env:pro} como se muestra en la siguiente figura.
        % El Segment configurado en VMware NSX-T se añade como un perfil de red en vRA. Este Segment contiene un servidor DHCP configurado, pero existe la posibilidad de crear un rango de direcciones IP para asignar una IP estática a cada VM que utiliza este perfil. A la subred se le han asignado los tags .
        % , pero en este caso se utiEn este perfil se ha creado un rango de direcciones IP para que vRA asigne una dirección estática a cada VM que lo utilice, y se le han asignado los tags \textit{function: pro} y \textit{env: pro}.
        \begin{figure}[h]
            \centering
            \includegraphics[width=0.8\textwidth]{imaxes/pruebaconcepto/vrealize/net-profile-MGMT.png}
            \caption{Subred habilitada en vRA que se corresponde con el Segment \textit{Mgmt-Region01A-VXLAN} configurado en VMware NSX-T.}
            \label{fig:net-profile}
        \end{figure}
        \FloatBarrier
        Los recursos de cómputo se habilitan configurando una Cloud Zone que se muestra en la figura \ref{fig:cloud-zone}. Esta integra en vRA los recursos del cluster vSphere del entorno y permite establecer la política a seguir para escoger el host donde se debe desplegar cada VM\footnote{La opción DEFAULT escoge un host aleatoriamente.} y la carpeta y resource pool donde se deben colocar. A esta Cloud Zone se le han asignado los tags \textit{cloud: private} y \textit{region: management}, y al resource pool el tag \textit{resource:rpprivate}.
       \begin{figure}[h]
            \centering
            \includegraphics[width=0.8\textwidth]{imaxes/pruebaconcepto/vrealize/cloud-zone.png}
            \caption{Cloud Zone (izquierda) y resource pool (derecha) configurados para utilizar los recursos de cómputo y colocar las VMs desplegadas.}
            \label{fig:cloud-zone}
        \end{figure}
        \FloatBarrier
        Igual que con los recursos de red, para habilitar los recursos de almacenamiento se debe crear un perfil de almacenamiento como se muestra en la figura \ref{fig:storage-policy}. Este perfil integra al datastore vSAN utilizado por el cluster vSphere del entorno y se establece como el perfil por defecto para aprovisionar recursos de almacenamiento desde vRA. Al perfil se le asignan los tags \textit{cloud: private} y \textit{function: pro}.
        % Para el almacenamiento, se crea un perfil que se muestra en la siguiente imagen. Este perfil tiene como recurso de almacenamiento el datastore configurado para el Management Domain, y se establece como el perfil por defecto para el aprovisionamiento de recursos de almacenamiento. Se la han asignado los tags \textit{cloud: private} y \textit{function: pro}. 
        \begin{figure}[h]
            \centering
            \includegraphics[width=0.6\textwidth]{imaxes/pruebaconcepto/vrealize/datastore-policy.png}
            \caption{Perfil de almacenamiento configurado donde se indica el datastore utilizado para aprovisionar recursos de almacenamiento.}
            \label{fig:storage-policy}
        \end{figure}
        \FloatBarrier
        Además, se definen varios perfiles de tamaños para que los usuarios determinen el tamaño de sus VMs. En estos perfiles se define una cantidad de CPU y memoria RAM con el fin de estandarizar la cantidad de recursos que un usuario puede asignar a una VM. En el entorno de pruebas, estos tamaños van desde \textit{x-small} con 1 CPU y 512 MB de memoria RAM, hasta \textit{large} con 8 CPUs y 16 GB de memoria RAM, mostrados en la siguiente figura.
        \begin{figure}[h]
            \centering
            \includegraphics[width=0.6\textwidth]{imaxes/pruebaconcepto/vrealize/flavor-mapping.png}
            \caption{Perfiles donde se preestablecen la cantidad de recursos que puede tomar una VM.}
            \label{fig:falvor-mapping}
        \end{figure}
        \FloatBarrier
        Con el objetivo de establecer una valoración de los recursos que utilizan los usuarios, se hace uso de las tarjetas de cobro. Se define una única tarjeta que se aplicará a todos los proyectos que se creen en la plataforma, y a medida que se vayan desplegando VMs se generará un cálculo total en base al precio asignado a cada recurso, a la cantidad de recursos utilizados y al tiempo que el despliegue se mantiene activo. Tanto el administrador del SDDC como el usuario tendrán acceso a estadísticas sobre el gasto que se realiza y de la cantidad total de recursos utilizados. En la figura \ref{fig:pricing-card} se muestra la valoración establecida para el consumo de recursos, que es de 1 €/hora por cada CPU cuando la VM está encendida, 2 €/hora por cada GB de memoria RAM cuando la VM está encendida y 0,5 €/hora por GB de almacenamiento mientras el despliegue esté activo. 
        \begin{figure}[h]
            \centering
            \includegraphics[width=0.6\textwidth]{imaxes/pruebaconcepto/vrealize/pricing-card.png}
            \caption{Tarjeta de cobro para valorar los recursos consumidos por los usuarios.}
            \label{fig:pricing-card}
        \end{figure}
        \FloatBarrier

    \end{subsubsection}

    \begin{subsubsection}{Uso del servicio Cloud}
        La plataforma de vRA ya está lista para ser utilizada por los usuarios. Los usuarios del CITIC que la utilizarán se organizan en proyectos, donde existe al menos un coordinador o administrador de proyecto. Cuando un grupo de usuarios quiere utilizar el servicio Cloud primero debe comunicarlo al administrador del SDDC, el cual crea el proyecto correspondiente y habilita el acceso a cada usuario con sus correspondientes permisos.

        Como ya se ha visto en la sección \nameref{subsubsec:WSA}, en el entorno de pruebas se han configurado cinco usuarios que se dividen en dos proyectos, uno llamado Web-DB con el objetivo de que los usuarios pertenecientes puedan construir un sitio web bajo demanda, y otro llamado Server-Desktop donde sus integrantes puedan desplegar dos VMs para realizar cierto trabajo de investigación (figura \ref{fig:projects-vra}). El proyecto Web-DB lo forman el usuario \textit{User One}, \textit{User Two} y \textit{Manager One}, el cual es el coordinador del grupo, y el proyecto Server-Desktop está formado por \textit{User Two}, \textit{User Three} y \textit{Manager Two}, el cual será el coordinador de este segundo grupo. Entonces, a los usuarios \textit{Manager One} y \textit{Manager Two} se les asigna el rol Administrador de Proyecto, y al resto de usuarios el rol Miembro de Proyecto, los primeros podrán controlar los diseños disponibles en el catálogo del proyecto, qué usuarios tienen acceso y los despliegues que estos realicen, mientras que los miembros del proyecto podrán desplegar los diseños habilitados (figura \ref{fig:projects-vra}).
        \begin{figure}[h]
            \centering
            \includegraphics[width=0.6\textwidth]{imaxes/pruebaconcepto/vrealize/projects-vRA.png}
            \caption{Proyectos creados para dar acceso a los usuarios a vRA.}
            \label{fig:projects-vra}
        \end{figure}
        \FloatBarrier
        \begin{figure}[h]
            \centering
            \includegraphics[width=0.6\textwidth]{imaxes/pruebaconcepto/vrealize/users-DB.png}
            \caption{Usuarios del proyecto Server-Desktop (izquierda) y usuarios del proyecto Web-DB (derecha).}
            \label{fig:project-users}
        \end{figure}
        \FloatBarrier
        Durante la creación de los proyectos el administrador establece la cantidad máxima de CPU, memoria RAM y almacenamiento que pueden consumir en total los usuarios del proyecto. Como se trata de un entorno de pruebas en el proyecto Server-Desktop se establece un límite de 2 VMs, 10 GB de memoria RAM y 6 CPUs, y en el proyecto Web-DB un límite de 3 VMs, 10 GB de memoria RAM y 6 CPUs, de esta forma los usuarios del proyecto no podrán superar ninguno de los límites establecidos. Para obtener una valoración del consumo se asigna a cada proyecto la tarjeta de cobro creada anteriormente.
        % Cuando el administrador crea un proyecto establece la cantidad máxima de CPU, memoria RAM y almacenamiento que puede consumir en total. También se pueden establecer mediante el uso de tags, los recursos que los despliegues del proyecto deben utilizar por defecto.
        \\
        Una vez configurados ambos proyectos los administradores de cada uno pueden acceder y empezar a crear los diseños de los recursos que requieran sus usuarios a través del componente Cloud Assembly. El administrador del proyecto Server-Desktop, \textit{Manager Two}, crea el diseño con el nombre WD-Server que se muestra en la siguiente figura\footnote{En el anexo \ref{appendix:wd-server-blueprint} se encuentra el contenido del archivo .yaml donde se establece la configuración del diseño.}.
        \begin{figure}[h]
            \centering
            \includegraphics[width=0.8\textwidth]{imaxes/pruebaconcepto/vrealize/windows-centos-blueprint.png}
            \caption{Diseño WD-Server para el proyecto Server-Desktop.}
            \label{fig:server-desktop-blueprint}
        \end{figure}
        \FloatBarrier
        En el archivo .yaml del diseño WD-Server se define una VM con el sistema operativo Windows Server 2016 y otra con CentOS 8, y una red a la que ambas se conectan. Se establecen además unas credenciales para cada VM, cuyos datos son introducidos por el usuario cuando se despliega el diseño y así poder iniciar sesión en ellas mediante SSH, o RDP en el caso de Windows. Los tags que se utilizan en la definición de las VMs son \textit{cloud:private}, \textit{region:management} y \textit{resource:rpprivate}, y el tag \textit{subnet-cidr:10.50.0.0/24} en la definición de la red, por lo tanto ambas VMs utilizarán los recursos del cluster vSphere, el Segment definido en VMware NSX-T y el datastore vSAN del entorno. En cuanto a la configuración de las interfaces de red, se establece que se configuren de forma dinámica con el servidor DHCP disponible en el Segment. Una vez completado el diseño, \textit{Manager Two} publica el diseño en el catálogo del proyecto para que los usuarios puedan acceder a él. Durante la publicación se especifica la versión del diseño ya que este puede ser actualizado, como se muestra en la siguiente figura.
        \begin{figure}[h]
            \centering
            \includegraphics[width=0.6\textwidth]{imaxes/pruebaconcepto/vrealize/create-version-blueprint.png}
            \caption{Publicación en el catálogo de una nueva versión del diseño.}
            \label{fig:publication-version}
        \end{figure}
        \FloatBarrier
        De la misma forma que para el proyecto Server-Desktop, el administrador del proyecto Web-WD, \textit{Manager One}, crea el diseño de los recursos necesarios para que los usuarios del proyecto puedan generar un sitio web basado en Wordpress automatizando la configuración del entorno, con la idea de que una vez desplegados los recursos el usuario pueda trabajar inmediatamente y exclusivamente en su sitio web. En la siguiente figura se muestra el diseño creado para el proyecto Web-WD\footnote{En el anexo \ref{appendix:worpress-mysql-blueprint} se encuentra el contenido del archivo .yaml donde se establece la configuración del diseño.}.
        \begin{figure}[h]
            \centering
            \includegraphics[width=0.8\textwidth]{imaxes/pruebaconcepto/vrealize/wordpress-mysql-blueprint.png}
            \caption{Diseño Wordpress-MySQL-Embedded para el proyecto Web-WD.}
            \label{fig:web-WD-blueprint}
        \end{figure}
        \FloatBarrier
        En el archivo .yaml del diseño Wordpress-MySQL-Embedded se define una VM con el sistema operativo CentOS 8, una red a la cual se conecta y un disco de almacenamiento conectado a la VM. En la sección \textbf{cloudConfig} del diseño se definen una serie de comandos que se ejecutan durante la inicialización de la VM cuando se despliega. Estos comandos son ejecutados por el servicio \textbf{cloud-init} y con ellos primero se instalan los paquetes necesarios para ejecutar MySQL y el framework Wordpress en un servidor Apache, luego se crea una base de datos y se configura Wordpress. De esta forma una vez se complete un despliegue el sitio web estará listo para ser usado. Además también se incluyen en esa sección del diseño los atributos que permiten a cloud-init crear las credenciales para acceder a la VM mediante SSH. Este diseño utiliza los mismos tags que el proyecto Server-Desktop por lo tanto utilizará los mismos recursos de cómputo, red y almacenamiento. Finalmente, \textit{Manager One} publica el diseño en el catálogo.        
        % Durante el despliegue del diseño, en la VM se instala y configura el gestor de base de datos MySQL y el framework web Wordpress. Para ello, se hace uso de la propiedad \textbf{cloudConfig} la cual invoca al servicio \textbf{cloud-init} para la ejecución de los comandos definidos en el diseño, que en este caso se utilizan para descargar los paquetes de MySQL, Wordpress y sus dependencias, y posteriormente crear una base datos, configurar Wordpress para conectarse a ella y habilitar un servidor Apache para acceder al sitio web. Además, también incluye la definición de un usuario para inciar sesión en la VM mediante SSH y de las credenciales usadas para conectarse a la base de datos. Los tags utilizados son los mismos que en el proyecto anterior por lo tanto este diseño se desplegará en la misma ubicación. Finalmente, \textit{Manager One} publica el diseño en el catálogo.        
        \\
        Una vez completada la fase de diseño y publicación, los usuarios ya pueden acceder a la plataforma Cloud y comenzar a utilizar los recursos en base a los diseños disponibles. A continuación se muestra cómo los usuarios de cada proyecto acceden al servicio Cloud y utilizan los recursos. 
        El usuario \textit{User Three} perteneciente al proyecto Server-Desktop accede a la plataforma utilizando sus credenciales\footnote{En el caso del entorno real utilizaría sus credenciales de la UDC.}, una vez inicia sesión accede al componente Service Broker de vRA donde se le muestra el catálogo de diseños disponibles en el proyecto al que pertenece (figura \ref{fig:login-user-3-catalog}). Cuando inicia el despliegue del diseño WD-Server, se muestra un formulario donde introduce los datos de las credenciales de cada VM\footnote{Para la VM con CentOS es necesario indicar el hash de la contraseña ya que el SO lo interpreta de esta forma, generado en este caso con el comando \textit{openssl passwd -1 -salt SaltSalt VMware123!} desde el powershell de Windows siendo "VMware123!" la contraseña en texto plano.} que se va a crear y el nombre del despliegue (figura \ref{fig:login-user-3-form-deployment}). A continuación comienza el proceso de despliegue. En este punto vRA se encarga de crear, configurar y reservar los recursos descritos en el diseño sin que el usuario tenga que realizar ninguna operación adicional (figura \ref{fig:deployment-process-user-3}).
        \begin{figure}[h]
            \centering
            \includegraphics[width=0.7\textwidth]{imaxes/pruebaconcepto/vrealize/login-user-3-credentials.png}
            \caption{Inicio de sesión del usuario \textit{User Three} (izquierda) y catálogo de diseños disponibles en el proyecto Server-Desktop (derecha).}
            \label{fig:login-user-3-catalog}
        \end{figure}
        \FloatBarrier
        \begin{figure}[h]
            \centering
            \includegraphics[width=0.6\textwidth]{imaxes/pruebaconcepto/vrealize/deployment-user-3-Windows.png}
            \caption{Formulario para configurar el nuevo despliegue iniciado por el usuario \textit{User Three}.}
            \label{fig:login-user-3-form-deployment}
        \end{figure}
        \FloatBarrier
        \begin{figure}[h]
            \centering
            \includegraphics[width=0.8\textwidth]{imaxes/pruebaconcepto/vrealize/deployment-start-user-3-Windows.png}
            \caption{Tarjeta del despliegue iniciado por el usuario \textit{User Three} (arriba) y la monitorización de todas las tareas llevadas a cabo por vRA durante el despliegue (abajo).}
            \label{fig:deployment-process-user-3}
        \end{figure}
        \FloatBarrier
        Cuando la creación y configuración de los recursos se ha completado estos ya están listos para su uso. En el panel de control del despliegue se muestra información como direcciones IP de las VMs, discos de almacenamiento disponibles en cada VM, la configuración aplicada a las VMs durante el despliegue o las credenciales indicadas por el usuario para acceder a las VMs (figura \ref{fig:user3-panel-control}). Además, desde este punto es donde el usuario puede gestionar los recursos pudiendo encenderlos o apagarlos, añadir discos de almacenamiento, modificar el tamaño de la VM, crear copias de seguridad y añadir tags para cambiar la ubicación de los recursos (figura \ref{fig:user3-actions}). Para acceder a las VMs creadas, \textit{User Three} simplemente tiene que comprobar las direcciones IP que se han asignado y conectarse a la VMs mediante SSH o a través de un cliente de escritorio remoto en el caso de Windows Server 2016 (figura \ref{fig:vm-cent-win-connection}).
        \begin{figure}[h]
            \centering
            \includegraphics[width=0.8\textwidth]{imaxes/pruebaconcepto/vrealize/user3-info-centos.png}
            \caption{Panel de control de la VM CentOS creada por \textit{User Three} (izquierda) y panle de control de la VM Windows creada por \textit{User Three} (derecha).}
            \label{fig:user3-panel-control}
        \end{figure}
        \FloatBarrier
        \begin{figure}[h]
            \centering
            \includegraphics[width=0.7\textwidth]{imaxes/pruebaconcepto/vrealize/user3-vm-actions.png}
            \caption{Acciones que \textit{User Three} puede ejecutar sobre las VMs creadas.}
            \label{fig:user3-actions}
        \end{figure}
        \FloatBarrier
        \begin{figure}[h]
            \centering
            \includegraphics[width=0.7\textwidth]{imaxes/pruebaconcepto/vrealize/Windows-RDP.png}
            \caption{Conexión de \textit{User Three} mediante RDP a la VM con Windows Server 2016 (arriba) y mediante SSH a la VM con CentOS (abajo).}
            \label{fig:vm-cent-win-connection}
        \end{figure}
        \FloatBarrier
        En el proyecto Web-WD, el usuario \textit{User Two} accede a la plataforma de vRA y en el catálogo tiene disponibles dos diseños, WD-Server y Wordpress-MySQL-Embedded, ya que es miembro de los dos proyectos Server-Desktop y Web-WD (figura \ref{fig:catalog-user-2}). El objetivo de este usuario es montar un sitio web por lo tanto inicia el despliegue del diseño Wordpress-MySQL-Embedded. En el formulario de configuración \textit{User Two} introduce las credenciales que se deben configurar en la VM para acceder a ella y para configurar a la base de datos  (figura \ref{fig:catalog-user-2}), luego inicia el despliegue del diseño (figura \ref{fig:deployment-user-2}). Una vez generada la VM con CentOS el servicio cloud-init se inicia y ejecuta los comandos descritos en el diseño (figura \ref{fig:cloud-init-user-2}). Cuando este proceso se ha completado el usuario ya puede acceder al panel de control del despliegue (figura \ref{fig:control-panel-user2}), comprobar la dirección IP de la VM, acceder a Wordpress a través del navegador, realizar la configuración inicial de su sitio web y comenzar a editar artículos (figura \ref{fig:wordpress-user-2}).
        \begin{figure}[h]
            \centering
            \includegraphics[width=0.7\textwidth]{imaxes/pruebaconcepto/vrealize/user-two-catalog.png}
            \caption{Diseños disponibles para \textit{User Two} (izquierda). Formulario de configuración de un nuevo despliegue del diseño Wordpress-MySQL-Embedded (derecha).}
            \label{fig:catalog-user-2}
        \end{figure}
        \FloatBarrier
        \begin{figure}[h]
            \centering
            \includegraphics[width=0.7\textwidth]{imaxes/pruebaconcepto/vrealize/user-2-card-deploy.png}
            \caption{Despliegues user2-wordpress-blog iniciado por \textit{User Two}.}
            \label{fig:deployment-user-2}
        \end{figure}
        \FloatBarrier
        \begin{figure}[h]
            \centering
            \includegraphics[width=0.6\textwidth]{imaxes/pruebaconcepto/vrealize/cloud-init-commands-wordpress.png}
            \caption{Fragmento de la ejecución de cloud-init donde se instala el paquete php-json y se descargan los archivos para la instalación de Wordpress.}
            \label{fig:cloud-init-user-2}
        \end{figure}
        \FloatBarrier
        \begin{figure}[h]
            \centering
            \includegraphics[width=0.8\textwidth]{imaxes/pruebaconcepto/vrealize/user-2-deploy-fin.png}
            \caption{Panel de control del despliegue iniciado por \textit{User Two} una vez finalizado.}
            \label{fig:control-panel-user2}
        \end{figure}
        \FloatBarrier
        \begin{figure}[h]
            \centering
            \includegraphics[width=0.8\textwidth]{imaxes/pruebaconcepto/vrealize/wordpress-installation.png}
            \caption{Página de instalación de Worpress cuando \textit{User Three} accede por primera vez (izquierda). Primer artículo escrito por \textit{User Two} en su nuevo sitio web.}
            \label{fig:wordpress-user-2}
        \end{figure}
        \FloatBarrier
        % Despliegue iniciado por \textit{User Two} finalizado junto con la información sobre la VM generada (derecha)
        % Desde el componente Cloud Assembly de vRA, el administrador crea los dos proyectos y asigna respectivamente el rol Administrador de Poryecto a los usuarios \textit{Manager One} y \textit{Manager Two}, mientras que el resto de usuarios reciben el rol Miembro de Proyecto. Con esta asignación cada usuario solo podrá acceder a los proyectos donde se le haya asignado un rol, lo cual podrán hacer a través del componente Service Broker de vRA.
        % El administrador de cada proyecto se encargará del diseño de blueprints, de habilitar las blueprints en el catálogo del proyecto y de controlar los usuarios que son miembros del proyecto. Los miembros del proyecto podrán realizar despliegues a partir de las blueprints habilitadas por el administrador del proyecto.
        A medida que se despliegan los diseños las VMs creadas comienzan a consumir recursos. El administrador del SDDC y los usuarios pueden monitorizar el consumo desde el panel de control de cada despliegue, donde pueden acceder a estadísticas diarias, semanales y mensuales sobre el uso de CPU, memoria RAM, almacenamiento y red.
        \begin{figure}[h]
            \centering
            \includegraphics[width=0.8\textwidth]{imaxes/pruebaconcepto/vrealize/statistics-service-broker.png}
            \caption{Panel de control del despliegue User2-Wordpress-Blog con la vista de monitorización de la VM Web-DB-CentOS-test-303.}
            \label{fig:statistics-user-2}
        \end{figure}
        \FloatBarrier
        Las estadísticas en cuanto a la valoración de los recursos en base a la tarjeta de cobro establecida también es accedida desde el panel de control del despliegue bajo la pestaña "Price". En ella los usuarios pueden ver la valoración total diaria, semanal y mensual de los recursos consumidos en el despliegue, y también de forma detallada donde se desglosa la valoración total en la valoración del cómputo, almacenamiento y otros cargos adicionales que se puedan aplicar. Además, el administrador del SDDC también tiene acceso a estadísticas sobre la valoración total sobre el consumo de recursos de un proyecto. 
        
        De esta forma el administrador del SDDC puede asignar a cada proyecto o usuario una cuenta con una cantidad de dinero ficticio del la cual se vaya extrayendo de forma mensual o semanal la valoración del consumo de recursos realizada por la plataforma. Cuando la cuenta esté vacía o no tenga suficiente saldo para consumir más recursos el administrador del SDDC puede bloquear nuevos despliegues dentro del proyecto o del usuario correspondiente hasta que su cuenta vuelva a tener saldo, para así conseguir que haya recursos disponibles para todos los usuarios y evitar que se mantengan despliegues activos cuyas VMs no están siendo usadas.

    \end{subsubsection}

    % Para ordenar el aprovisionamiento de los recursos, antes de realizar cualquier implementación, es necesario configurar la infraestructura que se va a poner a disposición de los usuarios. 
    %     En VMware vCenter Server, dentro del cluster vSphere se define un \textit{resource pool} y una carpeta que se utilizarán para colocar las VMs que se desplieguen desde vRA. 
    %     La red que utilizarán las VMs generadas estará controlada por VMware NSX-T para aprovechar las ventajas de sus redes definidas por software y así automatizar su configuración y hacer uso de los servicios que ofrece. Para ello se añade un Segment al router virtual de Tier-1 que se muestra en la figura \ref{fig:two-tier-topology}, VMware NSX-T informa de la nueva ruta al router físico mediante BGP proporcionando así acceso a redes externas.
    %     \begin{figure}[h]
    %         \centering
    %         \includegraphics[width=0.4\textwidth]{imaxes/pruebaconcepto/vrealize/topology-for-vRA-NSXT.png}
    %         \caption{Nuevo Segment \textit{vra-deployments} en el router virtual Tier-1.}
    %         \label{fig:topology-nsx-t-vra}
    %     \end{figure}
    %     \FloatBarrier
    %     Las VMs que los usuarios generan están basadas en plantillas que son creadas previamente por el administrador. Antes de generar la plantilla el administrador debe crear una VM en VMware vCenter Server y proveerla con una configuración mínima para que el usuario pueda aplicar su propia configuración durante el despliegue, para este entorno se crea una VM con el sistema operativo Ubuntu Server 20.04.1. Una vez se termina el proceso de instalación y actualización del sistema operativo, se configura el servicio \textbf{cloud-init}\footnote{Se puede encontrar más información sobre cloud-init en el enlace: \url{https://cloudinit.readthedocs.io/en/latest/}} el cual permite inicializar la VM con la configuración indicada en la blueprint, como se verá más adelante. También se pueden preinstalar servicios como bases de datos para que el usuario solo tenga que configurarlo. Cuando la configuración de la VM está lista, se limpia limpian el hostname, archivos de logs, claves SSH y caché del servicio cloud-init ejecutando el siguiente script, para que en cada implementación se genere una VM distinta a partir de la misma plantilla. Finalmente, la VM se convierte a una plantilla que se almacena en VMware vCenter Server.
    %     \begin{figure}[h]
    %         \centering
    %         \includegraphics[width=0.4\textwidth]{imaxes/pruebaconcepto/vrealize/install-cloud-init.png}
    %         \caption{Configuración del servicio cloud-init en Ubuntu Server 20.}
    %         \label{fig:cloud-init-config}
    %     \end{figure}
    %     \FloatBarrier
    %     \begin{figure}[h]
    %         \centering
    %         \includegraphics[width=0.4\textwidth]{imaxes/pruebaconcepto/vrealize/config-istance-vridm.png}
    %         \caption{Creación de una plantilla a partir de la VM de Ubuntu Server 20.}
    %         \label{fig:template-ubuntu}
    %     \end{figure}
    %     \FloatBarrier

    %     Una vez se tienen los recursos que serán usados por los usuarios es necesario añadirlos a vRA. Para poder esos recursos, durante su configuración vRA se asigna un tag a cada uno con la forma \textit{key:value}. En una blueprint estos tags permitirán indicar a que recurso se está refiriendo. 
    %     Lo primero es configurar una Cloud Zone que proporcionará acceso a todos los recursos situados en el cluster vSphere. El tag utilizado para esta será \textit{cloud:private}. Posteriormente se definen los tamaños de VMs que se habilitan, también llamado Flavour Mapping. Se configuran cuatro tamaños, X-small (1 CPU y 512 MB de RAM), Small (2 CPU y 8 GB de RAM), Medium (8 CPU y 4 GB de RAM) y Large (8 CPU y 16 GB de RAM). Las redes disponibles se añaden mediante la creación de perfiles donde se definen los tags con los que se identifica la red, su gateway, su servidor DNS, el rango de IPs para asignar una IP a cada VM conectada a esa red. 
\end{subsection}
\end{section}