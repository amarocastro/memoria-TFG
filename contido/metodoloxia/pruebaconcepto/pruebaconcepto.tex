\begin{section}{Prueba de concepto}
Para no afectar al funcionamiento del servicio proporcionado por el CITIC y para mostrar y probar las capacidades de VMware Cloud Foundation, en lugar de utilizar un entorno real el proyecto se lleva a cabo en un entorno aislado de prestaciones reducidas. Siguiendo la metodología Scrum, primero se deplegará VMware Cloud Foundation con la herramienta VMware Lab Constructor (VLC)\footnote{Se utiliza la versión 4.0.1 del instalador.}, que genera de forma automatizada una infraestructura física embebida basada en el diseño propuesto por VMware y sobre la cual posteriormente despliega los componentes base de VCF. Después se añadirá el componente que permite la gestión de usuarios y finalmente el servicio de aprovisionamiento. Una vez desplegados todos los elementos se realizará una demostración del servicio.

\begin{subsection}{Preparación}
    \begin{subsubsection}{Host ESXi}  
    
    Como base para la instalación se utiliza un servidor físico con el hipervisor ESXi instalado. Este host se utiliza para desplegar los componentes de VMware Cloud Foundation para crear un pequeño SDDC embebido para probar sus funciones. Este host cuenta con una memoria RAM de 192 GB, una CPU de 28,8 GHz y un \textit{datastore} con discos SSD con 2 TB de capacidad. Cuenta con dos interfaces físicas, una que conecta al host con el \textit{datastore} y otra a la que se conectan dos redes, una llamada \textit{Management Network} que permite acceder al host desde una VM para gestionarlo, y otra llamada \textit{VM Network} donde se conectan todas las VMs generadas por VLC y de los servicios que dan soporte a los componentes de VMware Cloud Foundation.
    \end{subsubsection}
    \begin{subsubsection}{Servicios}
      Todos los servicios requeridos por VMware Cloud Foundation se despliegan sobre el mismo servidor en forma de VMs. Una de las VMs es Windows Server 2016 que contiene un servidor DNS, un servidor NTP, un servidor Active Directory, un servidor SMTP y ejerce también como Certificate Authority. Otra VM contiene el sistema operativo VyOS que funciona como un router virtual y como servidor DHCP. Una última VM con Windows 10\footnote{Se refiere a ella como \textit{Jump Host}.} se requiere para ejecutar VLC y acceder al entorno embebido generado por VLC.
      El servidor DNS contiene el nombre y su respectiva dirección que un componente de VCF utilizará para que sus instancias se puedan comunicar con otras. Este servidor DNS implementa un único dominio que se denomina \textit{pesci.domain}. El servidor Active Directory proporciona un almacén de usuarios y grupos de usuarios a los cuales se les configura un rol dentro de cada componente de SDDC. Se utiliza este repositorio de usuarios en lugar del directorio real de la UDC para evitar posibles problemas del servicio. El router VyOS tiene configuradas todas las subredes y VLANs que VMware Cloud Foundation utiliza en la capa L3 de la infraestructura física y proporciona acceso a Internet, en las cuatro interfaces que conectan con las instancias de VMware NSX-T Edge utiliza enrutamiento dinámico BGP. El servidor DHCP se utiliza para asignar una dirección IP a las interfaces Tunnel EndPoint (TEP)\footnote{Más adelante se describirá la función de este elemento} de cada host ESXi.    
    \end{subsubsection}
    
    \begin{subsubsection}{VMware Lab Constructor}
      VLC genera en el host ESXi cuatro VMs que representan cuatro hosts ESXi. posteriormente, dentro de estos hosts VLC inicia la creación del \textit{management domain} de esta infraestructura embebida incluyendo todos los componentes de VMware Cloud Foundation. El diseño y configuración generados se describirá en las siguientes secciones.
      \begin{figure}[h!]
        \centering
        \includegraphics[width=0.6\textwidth]{imaxes/pruebaconcepto/hostFisico.png}
        \caption{Muestra la estructura generada por el instalador VLC. Cuatro hosts ESXi embebidos con los componentes de VMware Cloud Foundation cuyo tráfico circula a través del \textit{port group} VM Network.}
        \label{fig:estructura-generada-por-VLC}
      \end{figure}
      \FloatBarrier
  
      \begin{figure}[h]
        \centering
        \includegraphics[width=0.6\textwidth]{imaxes/pruebaconcepto/vSwitch0HostFisico.png}
        \caption{Máquinas virtuales en el host físico.}
        \label{fig:VMs-alojadas-host-fisico}
      \end{figure}
      \FloatBarrier
  
      En la imagen anterior se muestran las VMs que están funcionando sobre el host físico y que representan los componentes de la infraestructura física de un SDDC real, junto con el número de interfaces que se utilizan en cada una. Cada host ESXi generado por VLC cuenta con dos interfaces de red. El router VyOS, Jump Host y Windows Server 2016 se configuran antes del despliegue de VMware Cloud Foundation con VLC y se comunican con el entorno generado por VLC a través del \textit{port group} VM Network. El \textit{port group} Management Network se utiliza para acceder a la configuración del host físico a través de la dirección IP que se indica. Se utiliza la interfaz vmnic0 del host como salida del tráfico generado por el vSwitch0.
      \FloatBarrier
  
      \begin{figure}[h]
        \centering
        \includegraphics[width=0.4\textwidth]{imaxes/pruebaconcepto/RouterFisicoL3.png}
        \caption{Interfaces del router Vyos.}
        \label{fig:interfaces-router-fisico-L3}
      \end{figure}
      \FloatBarrier
  
      En la imagen anterior se muestra la configuración del router VyOS. Cada una de las interfaces se debe configurar antes del despliegue de VCF. Todas usan MTU de 9000 Bytes ya que la mayoría de componentes de VCF utilizan paquetes de red \textit{jumbo frame}. En las interfaces Eth2 y Eth3 el router utiliza enrutamiento dinámico BGP donde el AS local es 65001 y el AS remoto es AS 65003, configurado para anunciar a sus vecinos la red 10.0.0.0/24 Management Network. Las direcciones configuradas como \textit{neighbour} son: 172.27.11.2, 172.27.11.3, 172.27.12.2 y 172.27.12.3. En la dirección IP 172.27.254.199 de la interfaz eth0, el router proporciona un servidor DHCP que asigna direcciones IP en el rango 172.16.254.0 - 172.16.254.100.
      \FloatBarrier
  
      \begin{figure}[h]
        \centering
        \includegraphics[width=0.6\textwidth]{imaxes/pruebaconcepto/RedDesdeDentro.png}
        \caption{Topología de las redes del entorno desplegado.}
        \label{fig:red-L3-infraestructura-fisica}
      \end{figure}
      \FloatBarrier
  
      En la imagen anterior se muestran todos los componentes de VMware Cloud Foundation desplegados por VLC y los desplegados posteriormente para completar los objetivos del proyecto, como se conectan con los distintos servicios de red y a que redes se conectan. Las redes Mgmt-xRegion01-VXLAN y Mgmt-Region01A-VXLAN se corresponden a redes virtuales gestionadas por VMware NSX-T que no requieren ninguna configuración adicional en la capa 3 de la infraestructura física (esto se verá con detalle en el apartado de diseño de VMWare NSX-T).
      \FloatBarrier
    \end{subsubsection}
    
  \end{subsection}

\begin{subsection}{Diseño y configuración del Management Domain}
En esta sección se describen las funciones y configuración de los componentes desplegados en el entorno de pruebas con la ayuda de VLC. 

\begin{subsubsection}{Diseño de VMware vCenter Server}

El componente VMware vCenter Server es el punto de acceso y de control de todas las VMs localizadas en los hosts ESXi situados bajo su dominio. VMware vCenter Server funciona sobre una VM situada en el Management Domain. Esta instancia de vCenter Server contiene un dominio con un cluster vSphere que agrupa a los cuatro hosts ESXi que forman el Magement Domain. Estos hosts se denominan respectivamente \textit{esxi-1}, \textit{esxi-2}, \textit{esxi-3} y \textit{esxi-4}, y cada uno cuenta con 64 GB de memoria RAM y 19,9 GHz de CPU. Desde VMware vCenter Server el administrador gestiona los recursos de las VMs de cada componente, monitoriza los recursos, administra la creación y asignación de roles, permisos y usuarios, gestiona los grupos de discos que forman el \textit{datastore} de VMware vSAN, determina las redes a las que se conecta cada componente, establece la configuración de disponibilidad y recuperación ante fallos proporcionada por VMware vSphere, en definitiva, VMware vCenter Server es el punto desde donde se controla y administra el uso de recursos por parte de las VMs desplegadas. Además, integra el componente PSC, el cual controla la identidad y permisos de los administradores y aplicaciones que acceden a VMware vCenter, y gestiona el almacenamiento de licencias de VCF. El acceso a VMware vCenter Server se hace a través del componente web vSphere Client.
\begin{figure}[h]
  \centering
  \includegraphics[width=0.2\textwidth]{imaxes/pruebaconcepto/clusterVCenterServer.png}
  \caption{Dominio y cluster vSphere del Management Domain.}
  \label{fig:cluster-vCenter-Server}
\end{figure}
\FloatBarrier
En la imagen anterior se muestra el dominio (\textit{vcenter-mgmt.pesci.domain}), de la instancia de VMware vCenter Server, y el cluster vSphere (\textit{mgmt-cluster}) donde se alojan los componentes del Management Domain. Este cluster incluye los cuatro hosts ESXi y cuatro \textit{resource pools}, uno de ellos contiene las VMs de los componentes dedicados al Management Domain.
%  Con vCenter Server se simplifica la escalabilidad del SDDC, la gestión de actualizaciones para los componentes es más sencilla, permite determinar roles específicos y responsabilidades y permite aislar las redes de otras instancias de vCenter Server. Además, para gestionar vSpehere SSO Domain, VMware vCenter Server contiene embebido el componente PSC con todos los servicios necesarios. 
%  En caso de que existan varios \textit{Workload Domain} se puede habilitar el modo \textit{Enhanced Linked Mode} para poder gestionar todas las instancias de vCenter Server de forma centralizada desde un único vSphere Client.
% Por lo anterior, en el \textit{management domain} se despliega una instancia de VMware vCenter Server que incluye un cluster de VMware vSphere.

\end{subsubsection}


\begin{subsubsection}{Diseño almacenamiento VMware vSAN}
  
  Los hosts del Management Domain utilizan como almacenamiento un \textit{datastore} del componente VMware vSAN. Está formado por 16 discos SDD agrupados en cuatro grupos con configuración All-Flash, cada grupo está asociado a un host. Para mantener la disponibilidad del los ficheros almacenados en el \textit{datastore}, se establece la opción \textit{Failures-To-Tolerate} (FTT) igual a uno. De esta forma, VMware vSAN mantiene dos copias de los archivos generados por las VMs y las coloca en grupos de discos distintos, de forma que si ocurre un fallo en alguno de los hosts las VM seguirán teniendo acceso a sus archivos. Esta configuración equivale a tener un sistema de almacenamiento RAID 1, pero con la ventaja de que no se ha modificado la configuración del hardware y, si fuera necesario, se podría aumentar el número de réplicas simplemente editando el valor de FTT desde el portal de VMware vCenter Server. Como se muestra en la siguiente figura, VMware vSAN mantiene una copia del mismo archivo en dos hosts/grupos de discos diferentes, mientras la configuración física de cada grupo de discos es de tipo RAID 0. Las máquinas virtuales acceden al \textit{datastore} a través de una subred que utiliza su propia VLAN y a la que todos los hosts están conectados.
  \begin{figure}[h]
    \centering
    \includegraphics[width=0.8\textwidth]{imaxes/pruebaconcepto/vSANconfig.png}
    \caption{Ejemplo de como se almacena un archivo con VMware vSAN y FTT igual a uno}
    \label{fig:vSAN-config-FTT}
  \end{figure}
  \FloatBarrier
  Utilizar el sistema de almacenamiento de VMware vSAN supone una gran mejora respecto al sistema de almacenamiento basado en LUNs, utilizado actualmente en el CPD del CITIC. VMware vSAN monitoriza los dispositivos de almacenamiento y configura la redundancia de los archivos de forma dinámica y sencilla, permitiendo establecer una configuración específica según sea necesario, sin modificar los dispositivos físicos de almacenamiento. Con el sistema basado en LUNs, es obligatorio modificar la estructura y configuración de los dispositivos físicos cada vez que se quiera establecer una configuración de redundancia diferente en el sistema de almacenamiento, lo cual supone un gran coste para el administrador y un aumento de los riesgos. Si tomamos el ejemplo de la figura anterior, la redundancia el sistema gestionado por VMware vSAN, con FTT igual a 1, podría ser aumentada estableciendo la opción de configuración FTT igual a 2. Así, se crearía una nueva copia del archivo en un tercer grupo de discos, mientras la configuración física se mantiene igual.
  % sts participantes, soporta el fallo de un host lo cual permite dejar hosts fuera de servicio para tareas de mantenimiento. Esto es posible gracias a que con FTT (\textit{Failures-To-Tolerate}) igual a 1 se mantiene la redundancia de los datos almacenados en el \textit{datasotore}, en uno de los hosts. Cada grupo de discos cuenta con cuatro discos uno de ellos para caché, 16 discos en total. Para hacer disponible este servicio de almacenamiento, todos los hosts deben estar conectados a la subred generada para VMware vSAN y utilizar una VLAN para separar su tráfico.

\end{subsubsection}
    
\begin{subsubsection}{Diseño cluster VMware vSphere}
Como ya se ha mencionado, los cuatro hosts desplegados para el Management Domain están agrupados en un cluster de VMware vSphere. Gracias a dos funcionalidades de este componente, se establece una configuración para mantener activas las VMs desplegadas\footnote{Las VMs a las que se refiere son las instancias de cada componente de VCF.} dentro de este cluster. Entonces, VMware vSphere se encarga, de forma automatizada, de balancear el consumo de recursos y de recuperar el servicio de las VMs cuando alguna sufre un fallo. Estas funciones de VMware vSphere son:
\begin{itemize}
  \item vSphere High Availability: establece una cantidad de recursos que se reserva de los disponibles en el cluster vSphere, y se encarga de reiniciar una VM cuando deja de estar operativa. Para este cluster se establece una reserva el 25\% de la CPU total y el 25\% de la memoria RAM total. De esta forma, se asegura que una cuarta parte de los recursos disponibles están reservados para reiniciar, en un host diferente, una VM que ha dejado de funcionar.
  
  \item vSphere DRS: se encarga de migrar VMs de un host a otro dentro del cluster vSphere,con el objetivo de balancear la carga de trabajo entre los hosts disponibles. Usando este servicio se garantiza que cada VM obtiene la capacidad necesaria para funcionar correctamente, y aumenta la eficiencia del cluster al hacerse un mejor uso de sus recursos. Para realizar las migraciones entre hosts, vSphere DRS utiliza la funcionalidad vMotion, el cual permite mover una VM de un host a otro manteniendo el estado en el que se encontraba, y manteniendo activo el servicio de la VM. Por ejemplo, si el consumo de recursos de un host está alrededor del 100\%, vSphere DRS lo detecta e inicia la migración de la VM mediante vMotion, a otro host con recursos disponibles. 
\end{itemize}

Combinando estas dos funcionalidades, las tareas de mantenimiento se reducen ya que es VMware vSphere quien, de forma automatizada y transparente, se encarga de monitorizar el estado de VMs y hosts, de optimizar el uso de recursos y de asegurarse de que existen suficientes recursos para la ejecución de todos los flujos de trabajo.

% Dentro de un \textit{workload domain} pueden existir varios clusters vSphere con diferentes características según su finalidad. Los hosts ESXi que lo forman pueden ser de diferentes tamaños teniendo en cuenta que se pueden usar menos hosts ESXi de mayor capacidad o más hosts con menores prestaciones, el coste de cada host ESXi, el uso que se le va a dar al cluster y las características máximas y mínimas del cluster vSphere. Debido a la limitada cantidad de recursos que ofrece el host físico donde se realiza el despliegue, para el \textit{management domain} se utiliza un único cluster vSphere con de 4 hosts de los cuales se reserva un host para proveer redundancia. Todos los hosts ESXi cuenta con 64GB de memoria RAM menos uno que tiene 32 GB, y 19.9GHz de CPU. Dentro del cluster hay que configurar los servicios vSphere HA y vSphere DRS para proteger los componentes del SDDC. La configuración que se establece en el \textit{management domain} es la siguiente:
% En caso de que el \textit{management domain} esté extendido en dos AZ entonces se requieren 4 hosts en cada AZ para proporcionar redundancia y disponibilidad en caso de caída de una de las AZ.

% \begin{itemize}
%     \item \textbf{vSphere High Availability}: en este servicio la propiedad \textit{Admission Control Policy} permite establecer la cantidad de recursos reservados en caso de fallo y como se establece el cálculo de esos recursos. En el \textit{management domain} se configura para el fallo de al menos un host y reserva de recursos según un porcentaje, reservando así el 25\% de la CPU y el 30\% de la memoria RAM ya que funciona mejor cuando las VM usan mucha CPU y memoria. La otra propiedad que se debe habilitar para el correcto funcionamiento del servicio es \textit{VM and Application Monitoring}, que se encarga de reiniciar las VM en caso de caída.
    % que puede ser según el número hosts que pueden fallar en el cluster, según un porcentaje de reserva de rescursos o especificando el host donde se recolocan las VM del host caído.  RAM.  
%     \item \textbf{vSphere DRS}: 
% este servicio permite migrar VMs de un host ESXi a otro dentro del mismo cluster vSphere para equilibrar la carga de trabajo y mantener las VMs activas en caso de caída de alguno de los hosts. Se activa usando la opción por defecto \textit{Fully Automated} ya que aporta el mejor balance entre consumo de recursos y migraciones de VM innecesarias. Adicionalmente también se pueden establecer reglas para determinar el orden de encendido de las VMs pertenecientes a un mismo grupo. 
%     %En caso de que exista más de una AZ, se deben crear grupos de VM y de hosts de cada AZ para luego implementar reglas de afinidad para que las VM de una AZ no sean migradas a otra AZ ya que esto puede afectar al rendimiento de la VM. 
% \end{itemize}
% En el modelo consolidado se debe crear un único cluster con un mínimo de cuatro hosts ESXi ya que uno de los hosts se utiliza para asegurar la disponibilidad del almacenamiento vSAN cuando hay algún host inactivo. Este modelo proporciona capacidad de un único fallo por cluster.
\end{subsubsection}

\begin{subsubsection}{Diseño de red para el cluster vSphere}
  A parte de controlar la disponibilidad de los recursos, VMware vSphere también se encarga de gestionar las redes a las que se conecta cada VM, permitiendo separar cada tipo de tráfico en subredes y asignarles unas propiedades específicas. Para llevar esto a cabo y que las VMs puedan conectarse a la red externa y comunicarse con el resto de VMs, dentro del cluster vSphere se crea un vShpere Distributed Switch (vDS), en el cual se configuran puertos a los que se conectan las VMs alojadas en el cluster vSphere.
  % que las VMs tengan conectividad con la red externa y con el resto de VMs, en el cluster vSphere, se crea un vShpere Distributed Switch (vDS), en el cual se configuran puertos a los que se conectan las VMs alojadas en el cluster vSphere.
  \begin{figure}[h]
    \centering
    \includegraphics[width=0.5\textwidth]{imaxes/pruebaconcepto/distributedSwitchEntornoFinal.png}
    \caption{Contenido de vSphere Distributed Switch \textit{sddc-vds01}.}
    \label{fig:port-groups-vSwitch-vSphere}
  \end{figure}
  \FloatBarrier

  Como se muestra en la imagen anterior, el vDS creado para el cluster vSphere del Management Domain contiene varios puertos, donde hay VMs conectadas, y dos interfaces uplink (\textit{sddc-vds01-DVUplinks-10}). Estas dos interfaces, \textit{uplink1} y \textit{uplink2}, representan las interfaces de red físicas de cada host y son las que dan salida al tráfico de las VMs hacia la red física del entorno. Cada uno de los puertos tiene una función específica, estos son, \textit{sddc-vds01-mgmt}, dedicado al tráfico de configuración y gestión que los componentes de VCF envían entre sí, \textit{sddc-vds01-vmotion}, dedicado al tráfico de las migraciones de VMs entre hosts llevadas a cabo por la funcionalidad vMotion, \textit{sddc-vds01-vsan}, usado por las VMs para acceder al datastore de VMware vSAN el cual es el sistema de almacenamiento del entorno, \textit{sddc-edge-uplink01} y \textit{sddc-edge-uplink02}, puertos usados por los componentes de VMware NSX-T para dar salida, hacia la red física, al tráfico de las redes virtuales que gestiona este componente de VCF. Los demás puertos que se muestran en la imagen son generados de forma automática por VMware NSX-T. 
  En la configuración de cada puerto, se establece la VLAN que se asigna al tráfico que circula a través de él, las interfaces uplink por las que se transmite su tráfico hacia la red física y como se balancea la carga entre cada interfaz uplink, y la prioridad, respecto al resto de puertos, que se asigna al tráfico que circula por él. Los puertos, cuyo tráfico tiene mayor prioridad son, \textit{sddc-vds01-vsan} y \textit{sddc-vds01-vmotion},  para asegurarse de que obtienen el suficiente ancho de banda y así facilitar la transmisión de archivos de gran tamaño. 
  % re cada puerto se establecen varias propiedades En cuanto a las propiedades del tráfico, dentro del vDS se determina qué tráfico tiene más prioridad sobre los recursos de red, en este caso se establece el tráfico de los puertos \textit{sddc-vds01-vsan} y \textit{sddc-vds01-vmotion} como los de mayor prioridad, para asegurarse de que obtienen el suficiente ancho de banda y así facilitar la transmisión de archivos de gran tamaño. También, para cada puerto, se establece la VLAN con la que se etiqueta el tráfico, qué interfaces de red físicas se deben utilizar y como debe balancear el tráfico entre estas.
\\
  De esta forma, cada subred utilizada por los componentes de VCF es asociada con un puerto del vDS, y por lo tanto, las propiedades del tráfico de cada subred son configuradas a través de VMware vSphere. Esto, simplifica el proceso administración y configuración de las redes del entorno, ya que una vez configurados los dispositivos de red físicos, el router VyOS en este caso, con las direcciones IP, las etiquetas VLAN y MTU correspondientes a cada subred, la monitorización de la red y la configuración de la calidad del servicio se realizan desde VMware vSphere.
  
  

% Si bien en VMware Cloud Foundation existe VMware NSX-T, un componente dedicado únicamente a la administración de la red del SDDC, es desde VMware vSphere dónde se encuentran los elementos para establecer redes que separen cada tipo de tráfico de los componentes del SDDC. Estas redes se configuran en base a los siguientes aspectos:
% \begin{itemize}
%     \item Separar el tráfico de cada servicio para mejorar la eficiencia de la red y la seguridad. Así se puede ajustar las características de cada red, como el ancho de banda o la latencia, a las necesidades de cada servicio.
%     \item Utilizar un único vSphere Distributed Switch por cluster donde se añade un \textit{port group} por cada servicio.
%     % \item Mejorar el rendimiento usando NICs de tipo VMXNET3 en las máquinas virtuales.
%     \item Las NICs físicas de cada host ESXi conectados a un mismo vSphere Distributed Switch están conectadas también a la misma red física.
%     % \item Aquellas redes que se dedican a servicios de la infraestructura deben estar configuradas con puertos tipo \textit{vmkernel}.
% \end{itemize}
% Para el \textit{management domain} del SDDC se crea un único vSphere Distributed Switch llamado \textit{sddc-vds01} con la siguiente configuración:
% \begin{itemize}
    
%     \item Se establece un MTU igual 9000 Bytes para permitir el tráfico de \textit{jumbo frames} ya que son requeridos por algunos de los servicios.
    
%     \item Se habilita el servicio \textit{Network I/O} que permite establecer un nivel de prioridad a cada tipo de tráfico. Esto se realiza estableciendo limites de ancho de banda, políticas de balanceo de carga y reserva de recursos para un tipo de tráfico asociado a un servicio. Por cada tipo de tráfico hay cuatro aspectos que se pueden configurar que son \textit{Shares} (indica el \% de ancho de banda que se le da a un tipo de tráfico, el tipo de tráfico que tenga un mayor valor en \textit{Shares} tendrá más prioridad a la hora de usar los recursos), \textit{Reservation} (indica el valor de ancho de banda que se reserva para el tipo de tráfico) y \textit{Limit} (establece un valor máximo para el ancho de banda de un tipo de tráfico). En el \textit{management domain} los tipos de tráfico más relevantes que se deben configurar son los siguientes:
%     \begin{itemize}
%       \item \textit{Management Traffic}: el valor \textit{Shares} se establece al 50\% (\textit{Normal}) lo cual le da mayor prioridad que el resto de tipos. El resto de valores no se modifican.
%       \item \textit{vSphere vMotion Traffic}: el valor \textit{Shares} se establece al 25\% (\textit{Low}) ya que durante el estado normal del entorno este tipo de tráfico no es muy importante. El resto de valores no se modifican.
%       \item \textit{vSAN Traffic}: el valor \textit{Shares} se establece al 100\% (\textit{High}) para garantizar que este servicio recibe la cantidad de ancho de banda que necesita. El resto de valores no se modifican.
%       \item \textit{Virtual Machine Traffic}: el valor \textit{Shares} se establece al 100\% (\textit{High}) para garantizar que las VMs siempre tienen acceso a la red ya que son una parte importante del SDDC. El resto de valores no se modifican.
%     \end{itemize}
    
%     \item Para detectar errores de compatibilidad entre la configuración del vSphere Distributed Switch y la red física se habilita el servicio \textit{Health Check}. Este se encarga de comprobar si la configuración de cada VLAN y MTU se adapta a la configuración de la capa física.
    
%     \item Como puertos de salida \textit{Uplink} se configuran las interfaces físicas \textit{vmnic0} y \textit{vmnic1}. Como vDS es un componente distribuído, en cada host se usarán ambas interfaces de red como \textit{uplinks}.
    
% \end{itemize}
% En este vSpehere Distributed Switch para el Management Domain se configuran los siguientes \textit{port groups}, que son de tipo \textit{Distributed port group} y de tipo \textit{Uplink port group}. Además, el vDS está configurado sobre los cuatro hosts por lo tanto todos tienen acceso a todos los \textit{port groups}:
% \begin{itemize}
       
%         \item \textbf{Management port group}: es un \textit{Distributed port group} que comunica a todos los hosts ESXi entre si y transmite el tráfico entre los diferentes componentes de VMware Cloud Foundation, es decir, por este \textit{port group} circulan los comandos de configuración y gestión que los componentes del SDDC se envían entre ellos. Tiene el nombre \textit{sddc-vds01-mgmt}, a él se conectan las VMs \textit{vcenter-mgmt}, \textit{sddc-manager}, \textit{nsx-mgmt-1},\textit{edge01-mgmt} y \textit{edge02-mgmt}. Utiliza la subred con IP 10.0.0.0, máscara de red 255.255.255.0, VLAN 10 y MTU igual a 1500 Bytes. Esta red debe ser configurada también en la infraestructura física.
        
%         \item \textbf{vMotion port group}: es un \textit{Distributed port group} que está dedicado al tráfico del componente vSphere vMotion para realizar las migraciones de máquinas virtuales de un host a otro. Tiene el nombre \textit{sddc-vds01-vmotion} y utiliza la subred con IP 10.0.4.0, máscara de red 255.255.255.0, VLAN 10 y MTU igual a 8940 Bytes.
        
%         \item \textbf{vSAN port group}: es un \textit{Distributed port group} que está dedicado al servicio de almacenamiento VMware vSAN y por él los hosts acceden al almacenamiento del SDDC. Tiene el nombre \textit{sddc-vds01-vsan} y utiliza la subred con IP 10.0.8.0, máscara de red 255.255.255.0, VLAN 10 y MTU igual a 8940 Bytes.
        
%         \item \textbf{Edge Uplink port group}: es un \textit{Distributed port group} dedicado a las conexiones del component NSX-T Edge que se dedica a dar acceso a determinados servicios y para proporcionar a otros \textit{workload domain} conexión con la red externa. Están gestionados por VMware NSX-T ya que dan servicio a sus componentes. En el entorno existen dos \textit{port groups} para proporcionar redundancia y alta dispobilidad, uno llamado \textit{sddc-edge-uplink01} cuyas instancias están configuradas bajo la red con IP 172.27.11.0 y con máscara de red 255.255.255.0, y otro llamado \textit{sddc-edge-uplink02} cuyas instancias están configuradas bajo la red con IP 172.27.12.0 y máscara de red 255.255.255.0. Ambos \textit{port groups} están configurados como VLAN Trunk (por ellos puede circular tráfico de cualquier VLAN) y tienen un MTU de 8940 Bytes. En ambos hay configuradas las dos VMs llamadas \textit{edge01-mgmt} y \textit{edge02-mgmt}. Estas dos redes también se deben configurar en la infraestructura física.
        
%         \item \textbf{Uplink port group}: se trata de un \textit{Uplink port group} al que se le asignan las NICs físicas de cada host para establecer políticas sobre el tráfico que se dirige desde los hosts y VMs hacia fuera del vSphere Distributed Switch. Con el nombre \textit{sddc-vds01-DVUplinks-10}, en él están configuradas las dos NICs físicas de cada host, cada una en una interfaz \textit{uplink}.
        
% \end{itemize}
% \begin{figure}[h]
%   \centering
%   \includegraphics[width=0.4\textwidth]{imaxes/pruebaconcepto/distributedSwitchEntornoFinal.png}
%   \caption{Contenido de vSphere Distributed Switch \textit{sddc-vds01}.}
%   \label{fig:port-groups-vSwitch-vSphere}
% \end{figure}
% \FloatBarrier
% En la imagen anterior se muestran todos los \textit{Distributed Port Groups} y \textit{Uplink port group} que se alojan en el vSphere Distributed Switch (\textit{sddc-vds01}) dedicado al \textit{management domain}. En el \textit{port group} \textit{sddc-vds01-DVUplinks-10} se muestra como cada interfaz \textit{uplink} se mapea con una interfaz física (vmnic) de cada host ESXi. Los \textit{port groups} \textit{mgmt-Region01A-VXLAN}, \textit{mgmt-xRegion01-VXLAN} y \textit{sddc-host-overlay} son generados y administrados por el componente VMware NSX-T como se explicará más adelante. Cada \textit{port group} informa de cuantas VMs y hosts ESXi tiene conectados.

% La configuración que se aplica a cada \textit{Distributed port group} descrito anteriormente es la siguiente:
% \begin{itemize}
%   \item \textit{Port binding}: permite indidcar como se gestionan los puertos de un \textit{port group} cuando se añade o elimina una VM. Tiene dos opciones de configuración, la primera se denomina \textit{Static Port Binding} y su función consiste en asignar un puerto dentro del \textit{port group} a la VM que se conecta y solo se elimina cuando la VM es borrada. La segunda opción se denomina \textit{Ephemeral Port Binding} y consiste en que el puerto se asigna a la VM cuando esta se enciende y se elimina cuando se apaga o elimina. Para los \textit{port groups} \textit{sddc-vds01-vsan} y \textit{sddc-vds01-vmotion} se configura la opción \textit{Static Port Binding} ya que así se asegura que las VMs se conectan siempre al mismo puerto lo cual permite mantener datos históricos y hacer monitoreo a nivel de puerto. Para los \textit{port group} \textit{sddc-vds01-mgmt}, \textit{sddc-edge-uplink01} y \textit{sddc-edge-uplink02} se configura la opción \textit{Ephemeral Port Binding} ya que, como el tráfico que circula por ellos es el que gestiona todos los componentes del SDDC y dan acceso a otras redes externas, se elimina la dependencia con el estado de VMware vCenter Server permitiendo que la comunicación continúe aunque VMware vCenter Server no se encuentre operativo.

%   \item \textit{Load Balancing}: indica como se distribuye el tráfico de salida de cada VM/host que se encuentran en el \textit{port group} entre las NICs físicas. Se selecciona \textit{Route based on physical NIC load}, es decir, el tráfico de una VM se transmite por una única NIC por lo que si esa NIC física está saturada, se asignará otra NIC física a la VM.
  
%   \item \textit{Network failure detection}: esta opción permite establecer como debe determinar el \textit{port group} que alguna de las NICs físicas está fuera de servicio. Se selecciona \textit{Link status only} para que esto se determine según el estado que le transmite la NIC física, así se pueden detectar los fallos que ocurren en la red física.
  
%   \item \textit{Notify switches}: se habilita para permitir a los host enviar \textit{frames} a los switches físicos para que estos conozcan la localización de las VM que están funcionando en cada host.
  
%   \item \textit{Failback}: permite determinar como se reactiva una NIC cuando esta se recupera de un fallo. Se habilita para establecer que la NIC se marcará como activa inmediatamente después de que se haya recuperado. Esta opción se debería desactivar en caso de que el estado de la NIC sea inestable.
  
%   \item \textit{Failover Order}: permite determinar que uplinks se deben utilizar, los que se seleccionan como \textit{active} son los que se utilizarán por defecto, los que se seleccionan como \textit{stand by} se usarán cuando los uplinks marcados como \textit{active} se encuentren desactivados. Se seleccionan las dos interfaces \textit{uplink} disponibles en el estado \textit{active}. Para el \textit{port group} \textit{sddc-edge-uplink01} se selecciona la interfaz \textit{uplink1} como activa y se deja sin usar la interfaz \textit{uplink2}, mientras que se configura de forma contraria en el \textit{port group} \textit{sddc-edge-uplink02}.
% \end{itemize}

\end{subsubsection}


\begin{subsubsection}{Diseño de la red del SDDC con VMware NSX-T}
    En el entorno de pruebas existe una red virtual que se define mediante software mantenida por VMware NSX-T, que al estar desacoplada de la infraestructura física se puede gestionar sin necesidad de modificar la configuración de la red física. Esto implica que se pueden aplicar diferentes configuraciones de red de forma sencilla, mejorando y simplificando su administración y seguridad. La virtualización de la red con VMware NSX-T se basa principalmente en el concepto de Segment:
    % La virtualización de la red con VMware NSX-T se basa en dos componentes, Transport Zone (TZ) y Segment
    % % proporcionando elasticidad y flexibilidad a la hora de administrar y obtener los recursos, tanto para el administrador como para el usuario final. 
    % En el Management Domain se despliega una instancia de NSX-T Manager y dos instancias del componente NSX-T Edge.
    % Para mantener la disponibilidad de VMware NSX-T y balancear su carga de trabajo, se despliegan tres instancias de NSX-T Manager Appliance, aunque para reducir el consumo de recursos, en el entorno de prueba solo se creará una instancia de este componente llamada \textit{nsx-mgmt-1}. También se despliegan dos instancias del componente VMware NSX-T Edge, llamadas \textit{edge01-mgmt} y \textit{edge02-mgmt}.
    \begin{itemize}
        % \item Transport Zone: se trata de un contenedor dentro del cual se definen Segments. A una TZ se conectan TNs\footnote{Los Transport Nodes son los hosts físicos y cada instancia de VMware NSX-T Edge.} para acceder a los Segments. Cada TN puede estar conectado a varias Transport Zones.
        \item Segment: se trata de un dominio de broadcast de capa 2 (una subred) al cual las VMs se conectan.
    \end{itemize}
    Un Segment se extiende por diferentes hosts los cuales pueden estar en la misma red a nivel físico o en distintas partes de la infraestructura. De esta forma, las VMs situadas en hosts con acceso a un Segment pueden conectarse a él y comunicarse de forma directa con otras VMs situadas en un host conectado a una red física diferente. Es decir, con un Segment se pueden comunicar diferentes puntos de la infraestructura sin cambiar la estructura de la red física, ya que VMware NSX-T se encarga de encapsular el tráfico al salir de un host cuando su destino se encuentra en una red física diferente a la de origen, haciendo creer al destinatario y al remitente que se encuentran en la misma subred (figura \ref{fig:frame-nsx-t}).
    \begin{figure}[h]
        \centering
        \includegraphics[width=0.6\textwidth]{imaxes/pruebaconcepto/frame-nsx-t.png}
        \caption{Ejemplo de cómo la comunicación entre dos VMs a través de un Segment es realizada transportando el tráfico a través de diferentes redes físicas.}
        \label{fig:frame-nsx-t}
      \end{figure}
    \FloatBarrier
    En el entorno de pruebas existen cuatro Segments, \textit{Mgmt-Region01A-VXLAN} y \textit{Mgmt-xRegion01-VXLAN}, ambos dedicados a dar acceso a la red a los componentes de VMware vRealize Suite y a las VMs desplegadas por los usuarios, y \textit{VCF-edge\_mgmt-cluster\_segment\_11} y \textit{VCF-edge\_mgmt-cluster\_segment\_12}, utilizados para dar salida hacia el router VyOS al tráfico que proviene de los Segments anteriores.

    Los encargados de gestionar el enrutamiento entre Segments y hacia la red externa son las instancias de NSX-T Edge. Para ello, internamente forman una estructura de routers virtuales que a parte de realizar las tareas de enrutamiento proporcionan servicios de red como NAT, Load Balancing, DNS, DHCP, VPN y Firewall, y mantienen rutas redundantes hacia el dispositivo físico de red.
    \begin{figure}[h]
        \centering
        \includegraphics[width=0.6\textwidth]{imaxes/pruebaconcepto/estructura_NSX_T.png}
        \caption{Segments a los que se conecta cada host del entorno y cómo estos acceden a la red física a través de las VMs de NSX-T Edge.}
        \label{fig:estructura-NSXT}
      \end{figure}
    \FloatBarrier
    \begin{figure}[h]
        \centering
        \includegraphics[width=0.6\textwidth]{imaxes/pruebaconcepto/vrealize/topology-vrops.png}
        \caption{Topología virtual de las redes virtuales construídas en VMware NSX-T.}
        \label{fig:two-tier-topology} 
    \end{figure}
    \FloatBarrier
    En la primera figura \ref{fig:estructura-NSXT}, se muestra como cada host se conecta a los dos Segments disponibles para que las VMs que se residen en ellos puedan acceder a la red física, en última instancia a través del vDS desplegado en el cluster de VMware vSphere. En la segunda figura \ref{fig:two-tier-topology}, se muestra la misma estructura pero desde el punto de vista interno de VMware NSX-T. En ella se aprecian los dos Segments donde uno de ellos tiene tres VMs conectadas (componentes de VMware vRealize Suite), y tres routers virtuales, dos de tipo Tier-1 y uno de tipo Tier-0. Los routers Tier-1 proporcionan servicios de red y enrutamiento entre Segments, \textit{services-Tier1} contiene un servidor DHCP  para las VMs que se conecten al Segment \textit{Mgmt-Region01A-VXLAN}, mientras que el router de tipo Tier-0 se encarga de dirigir el tráfico hacia la red física (router VyOS) a través de cuatro conexiones que se corresponden con las que se conectan al vDS que se muestra en la figura \ref{fig:estructura-NSXT}. Para que el router VyOS tenga conocimiento de las subredes virtuales/Segments existentes, las instancias de NSX-T Edge se las comunica mediante el protocolo de enrutamiento dinámico BGP.
    \\
    Usando VMware NSX-T el administrador puede gestionar y crear redes para ser consumidas por los usuarios de la plataforma. La creación de estas redes virtuales se hace bajo demanda y no requiere ninguna configuración adicional en los dispositivos de la red física. Su gestión se realiza siempre desde VMware NSX-T, el cual permite monitorizarlas, controlar su seguridad y establecer servicios dedicados. Además, permite extender una red virtual sobre diferentes redes físicas, permitiendo acceder a las VMs conectadas a esa red virtual desde diferentes puntos del SDDC, lo cual implica que una VM se puede migrar de una localización a otra para aumentar su disponibilidad sin necesidad de cambiar su configuración de red. En la infraestructura actual del CITIC todo esto no es posible ya que las redes que se crean dentro del entorno deben configurarse previamente sobre la red física, y todos los servicios de red necesarios deben ser proporcionados también desde dispositivos físicos, es decir, no existe actualmente en el CITIC una plataforma que permita gestionar las redes de la infraestructura de una forma dinámica y sin un alto coste en tiempo y riesgos.
    % Una TZ se extiende en diferentes hosts que pueden estar situados en la misma red a nivel físico, o en distintas partes de la infraestructura del SDDC. Los hosts que estén conectados a una TZ tendrán acceso a los Segments (cada Segment equivale a una subred) generados en esa TZ. Así, se hace posible la creación de redes accesibles desde cualquier parte de la infraestructura del SDDC sin necesidad de modificar los dispositivos de red físicos.
    % En el entorno, existen dos TZs. Una de ellas, la TZ \textit{mgmt-domain-m01-overlay-tz}, contiene dos Segments, \textit{mgmt-Region01A-VXLAN} y \textit{mgmt-xRegion01-VXLAN}, los cuales son utilizados para conectar las instancias de los componentes de VMware vRealize Suite. La otra TZ disponible, \textit{sfo01-m01-edge-uplink-tz} contiene otros dos Segments, utilizados para dar salida hacia el router VyOS al tráfico que circula por la TZ anterior (\textit{mgmt-domain-m01-overlay-tz}). Para que el tráfico de cada Segment pueda circular por la red física de la infraestructura, VMware NSX-T lo encapsula cuando sale de un host para que este pueda llegar al host destinatario.
    % Los encargados de gestionar el enrutamiento entre Segments y hacia la red externa son las instancias de NSX-T Edge. Para ello, internamente forman una estructura de routers virtuales que a parte de realizar las tareas de enrutamiento, proporcionan servicios de red como NAT, Load Balancing, DNS, DHCP, VPN y Firewall, y mantienen rutas redundantes hacia el dispositivo físico de red.

    
    
    % VMware NSX-T se encarga de encapsular el tráfico de estas redes virtuales 
    
    % Cuando el tráfico de un Segment debe salir de un TN a la red física para alcanzar su destino\footnote{Este paquete contiene la información de las VMs origen y destino que se están comunicando.}, este es encapsulado de nuevo en un paquete con la información de los TN origen y destino[Fig. \ref{fig:Frame-Segment-NSXT}]. Gracias a esta encapsulación, elementos que se encuentran en distintos entornos de la red física se pueden comunicar como si estuvieran directamente conectados el uno al otro. Así, es posible la creación de una misma red que se extienda por toda la infraestructura del SDDC, permitiendo comunicar componentes situados en distintas redes físicas, sin necesidad de modificar la configuración de los dispositivos físicos ni su topología. Esto hace necesario el uso de un protocolo de enrutamiento dinámico con BGP, tanto en la infraestructura física como en la red virtual, y así automatizar el proceso de configuración de nuevas redes virtuales.
    % El tipo de encapsulación que se realiza sobre el tráfico de los Segments se define la configuración de cada TZ, esta puede  ser de tipo VLAN o de tipo Overlay usando el protocolo Geneve.
    % \begin{itemize}
    %     \item TZ de tipo VLAN: se define una VLAN que se utilizará para identificar y encapsular el tráfico perteneciente a los Segments de una misma TZ. La VLAN que se defina debe estar configurada en la red física para que su tráfico sea aceptado.
    %     \item TZ de tipo Overlay Geneve: este protocolo se encarga de encapsular el tráfico saliente añadiendo una cabecera extra donde incluye un identificador. Cada Segment tendría su propio identificador.
    % \end{itemize}
    % \begin{figure}[h]
    %     \centering
    %     \includegraphics[width=0.8\textwidth]{imaxes/pruebaconcepto/Frame.png}
    %     \caption{Paquete de un Segment encapsulado cuando sale a la red física.}
    %     \label{fig:Frame-Segment-NSXT}
    % \end{figure}
    % \FloatBarrier
    % En la imagen anterior se muestra como se encapsula un paquete perteneciente a un Segment cuando este sale de un host/TN al medio físico. En las cabeceras del paquete correspondiente al Segment tendrá la información sobre las VMs origen y destino que se están comunicando, mientras que las cabeceras que encapsulan a ese paquete contienen la información sobre los hosts/TNs origen y destino donde se encuentran las VMs que se están comunicando. La dirección IP utilizada por los hosts/TNs para enviar el tráfico de un Segment encapsulado, se denomina Tunnel End-Point (TEP) y se asigna mediante DHCP para automatizar su configuración cuando un nuevo host/TN es añadido al entorno.

    % \begin{figure}[h]
    %     \centering
    %     \includegraphics[width=0.8\textwidth]{imaxes/pruebaconcepto/OverlayTZSegments.png}
    %     \caption{Segments de la Transport Zone \textit{mgmt-domain-m01-overlay-tz}}
    %     \label{fig:overlay-TZ-segments-NSXT}
    %   \end{figure}
    %   \FloatBarrier

    %   En la imagen anterior se muestra la Transport Zone de tipo Overlay, definida en el entorno de pruebas, con el nombre \textit{mgmt-domain-m01-overlay-tz}. A ella se conectan los seis TNs\footnote{Los seis TNs son los cuatro hosts y las dos instancias de NSX-T Edge}\footnote{Un TN utiliza el elemento NSX-T Virtual Distributed Switch (N-VDS) para conectarse a los Segments de una TZ} y contiene tres Segments. El Segment \textit{mgmt-xRegion01-VXLAN} se utiliza para desplegar aplicaciones cuyas instancias deben ser accesibles desde cada Region del SDDC. El Segment \textit{mgmt-Region01A-VXLAN} tiene como finalidad alojar aplicaciones que solo deben ser accesibles desde dentro de una misma Region. El Segment \textit{sddc-host-overlay} es utilizado por los componentes de VMware NSX-T para comunicarse con los TNs. Con cada Segment de tipo Overlay se genera un \textit{port group} con el mismo nombre en el vDS (se puede ver en la figura \ref{fig:port-groups-vSwitch-vSphere}) que se utilizan para transmitir su tráfico a la red física.
    % %    VMware NSX-T genera en el vSphere vSwitch un \textit{port group} por cada \textit{segment} para poder conectar la VM de cada componente al \textit{port group} que le corresponda.

    % \begin{figure}[h]
    %     \centering
    %     \includegraphics[width=0.8\textwidth]{imaxes/pruebaconcepto/VLANTZSegments.png}
    %      \caption{Segments de la  Transport Zone \textit{sfo01-m01-edge-uplink-tz}}
    %     \label{fig:VLAN-TZ-segments-NSXT}
    % \end{figure}
    % \FloatBarrier
    % En la imagen anterior se muestra la Transport Zone de tipo VLAN, definida en el entorno de pruebas, con el nombre \textit{sfo01-m01-edge-uplink-tz}. A ella se conectan los dos TNs que son instancias de NSX-T Edge, \textit{edge01-mgmt} y \textit{edge02-mgmt}. Contiene dos Segments, \textit{VCF-edge-mgmt-cluster-segment-11} y \textit{VCF-edge-mgmt-cluster-segment-12}, que son utilizados por las instancias de NSX-T Edge para transmitir el tráfico de todas las redes virtuales gestionadas por VMware NSX-T hacia redes físicas externas al SDDC. Para ello utilizan utilizan los \textit{port groups} \textit{sddc-edge-uplink01} y \textit{sddc-edge-uplink02} (se pueden ver en la figura \ref{fig:port-groups-vSwitch-vSphere}) del vDS para transmitir su tráfico hacia las interfaces de red físicas de cada host. Ambas instancias forman la topología de red, que se muestra en la siguiente imagen, para comunicar las redes virtuales de VMware NSX-T con el router físico.
    % \begin{figure}[h]
    %     \centering
    %     \includegraphics[width=0.4\textwidth]{imaxes/pruebaconcepto/UplinkDesign.png}
    %     \caption{Topología de red de las insterfaces \textit{uplink}.}
    %     \label{fig:Uplink-Design-Edge-NSXT} 
    % \end{figure}
    % \FloatBarrier
    % Como se puede ver en la imagen anterior, los dos Segments son utilizados por las instancias de NSX-T Edge para mantener rutas redundantes hacia la red externa y así aumentar su disponibilidad. A parte, este componente también se encarga de proporcionar un conjunto de servicios de red a los componentes que están conectados a los Segments de VMware NSX-T. Para entregar estos servicios, internamente, VMware NSX-T forma una topología con una serie de routers virtuales.
    % \begin{figure}[h]
    %     \centering
    %     \includegraphics[width=0.4\textwidth]{imaxes/pruebaconcepto/topologiaTwoTierRouting-Final.png}
    %     \caption{Topología virtual de VMware NSX-T}
    %     \label{fig:two-tier-topology} 
    % \end{figure}
    % \FloatBarrier
    % En la imagen anterior se muestra la topología que forman VMware NSX-T para proporcionar acceso a la red externa y para entregar otros servicios a los componentes situados en los dos Segments existentes. En esta topología hay tres routers, \textit{mgmt-domain-tier0-gateway}, \textit{mgmt-domain-tier1-gateway} y \textit{mgmt-domain-lb01-tier1-gateway}, de los cuales, el primero se encarga de gestionar la comunicación con los dispositivos de red físicos, y los dos restantes se encargan de enrutar el tráfico entre Segments y hacia el router de Tier 0, y de entregar distintos servicios a los componentes que residen en los Segments. Los servicios de red que se pueden habilitar en los dos routers de Tier 1 son NAT, Load Balancing, DNS y VPN.
\end{subsubsection}
\end{subsection}
\begin{subsection}{Operaciones de la Arquitectura}
    El entorno ya está configurado para funcionar como un SDDC, a partir de este punto ya no es necesario realizar ninguna modificación en la infraestructura física ya que todas las tareas que se deben realizar están dentro del alcance de los componentes de VMware Cloud Foundation. Para finalizar la construcción del SDDC y habilitar un servicio donde los usuarios puedan aprovisionar recursos bajo demanda, se instalarán sobre el entorno desplegado las aplicaciones VMware vRealize Identity Manager\footnote{También se denomina Workspace ONE Access} y VMware vRealize Automation. La primera permite al administrador conectar con el servidor de usuarios y gestionarlos de forma centralizada para entregarlos a múltiples aplicaciones, como vRealize Automation, desde un mismo punto.
    
    \begin{subsubsection}{vRealize Suite Lifecycle Manager}
        
    \end{subsubsection}
    \subsubsection{Gestión del Ciclo de Vida}
    Elementos que se encargan de administrar el ciclo de vida de los componentes:
    \begin{itemize}
    %     \item \textbf{vSphere Update Manager}: por cada instancia de VMware vCenter Server se despliega una instancia de vSphere Update Manager. Este componente utiliza el servicio \textit{Update Manager Download Service} (UMDS) para obtener las actualizaciones de la red externa al SDDC, el cual se despliega en una red AVN [Fig. \ref{fig:avnConsolidated}], permitiendo limitar el acceso a Internet de vSphere Update Manager y reduciendo el número de descargas ya que un UMDS se comparte entre varias instancias de VMware vCenter Server. Se crea una instancia de UMDS por cada \textit{region} existente [Fig. \ref{fig:UpdateManagerArc}].
    %     \begin{figure}[h!]
    %         \centering
    %         \includegraphics[width=0.5\textwidth]{imaxes/conceptosPrevios/UpdateManagerArch.png}
    %         \caption{Diseño de vSphere Update Manager en el modelo consolidado (izquierda) y en el modelo estándar (derecha)}
    %         \label{fig:UpdateManagerArc}
    %     \end{figure}
    %     \FloatBarrier
        
        \item \textbf{vRealize Suite Licfecycle Manager}: componente utilizado para desplegar, actualizar y configurar, de forma automatizada, los productos vRealize Operations, vRealize Log Insight, vRealize Automation y vRealize Business Cloud. De este componente se despliega una única instancia en una AVN accesible desde cada \textit{region} por todas las instancias de VMware vCenter Server. Se debe registrar su nombre de dominio en el servidor DNS para hacerla accesible.
        
    %     \iffalse
    %     y se puede elegir entre dos modelos de despliegue, uno en el que se usa una máquina virtual denominada  que se encarga de descargar los archivos requeridos por vSphere Update Manager mientras este se encuentra en un entorno aislado [Fig. \ref{fig:updateManager}], y otro donde es la instancia de vSphere Update Manager la que realiza la descarga de los ficheros. La primera opción incrementa la seguridad y permite compartir estos archivos entre distintas instancias de vSphere Update Manager.\\
    %     Una vez desplegado se pueden establecer diferentes configuraciones a nivel de host, máquina virtual y cluster, para que durante la instalación de actualizaciones el servicio del SDDC continúe operativo y evitar la pérdida de información y errores en los recursos.
    %     \begin{figure}[h!]
    %   \centering
    %   \includegraphics[width=0.8\textwidth]{imaxes/conceptosPrevios/vRealizeUpdateArchLifeCyle.png}
    %   \caption{Estructura de la gestión del ciclo de vida con vRealize Suite Lifecycle Manager.}
    %   \label{fig:vrealizeUpdateManager}
    % \end{figure}
    % \FloatBarrier
    %     \fi
    
    
    \end{itemize}
    
\end{subsection}
\end{section}