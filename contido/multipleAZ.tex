En caso de que exista más de una AZ, se deben crear grupos de VM y de hosts de cada AZ para luego implementar reglas de afinidad para que las VM de una AZ no sean migradas a otra AZ ya que esto puede afectar al rendimiento de la VM.

En caso de que el \textit{management domain} esté extendido en dos AZ entonces se requieren 4 hosts en cada AZ para proporcionar redundancia y disponibilidad en caso de caída de una de las AZ.

cada conjunto de instancias de cada componente forman un cluster donde cada VM está protegida por las funcionalidades vSphere HA y vSphere DRS para proveer alta disponibilidad del servicio y migrar las VMs a otra ubicación en caso de caída de una AZ o de un host. 

Si existe más de una AZ, varios de los \textit{distributed port groups} se deben extender al resto de AZs para que en caso de que la primera AZ falle sus VMs se puedan migrar a otra AZ y sigan teniendo conectividad. Los \textit{port groups} que deberían estar extendidos en todas las AZ son el \textit{port group} \textit{sddc-vds01-mgmt} de cada AZ\footnote{Cada AZ tiene su propio \textit{Management port group}, entonces en cada AZ debe ser accesible el \textit{Management port group} del resto de AZs.}, los \textit{port group} \textit{sddc-edge-uplink01}, \textit{sddc-edge-uplink02} y \textit{port group} \textit{Edge Overlay}.