\begin{chapter}{Introducción}
\label{chap:introducion}
\lettrine{S}{egún} \textit{National Institute of Standards and Technology} (NIST), \textquote{Cloud computing is a model for enabling ubiquitous, convenient, on-demand network access to a shared pool of configurable computing resources}\cite{computing}. Las principales características de este modelo son\label{nist}:
\begin{itemize}
    \item \emph{Autoservicio bajo demanda}: El usuario puede aprovisionar recursos según sus necesidades y de forma automática sin requerir ninguna interacción humana con el proveedor del servicio.
    \item \emph{Acceso por red}: El servicio está disponible para los usuarios a través de red de forma remota.
    \item \emph{Almacén de recursos}: Los recursos son accesibles por múltiples usuarios simultáneamente, y todos ellos acceden a la misma instancia del software que gestiona el servicio, siendo este un servicio \textit{multi-tenant}.
    \item \emph{Elasticidad}: Los recursos se pueden aprovisionar o liberar de forma elástica, es decir, se pueden escalar de forma rápida según las necesidades del usuario.
    \item \emph{Servicio medido}: El servicio Cloud es capaz de obtener y abstraer información acerca del consumo de recursos para monitorizarlos, controlarlos e informar al usuario y al proveedor.
\end{itemize}

 El Centro de Investigación en Tecnoloxías da Información e as Comunicacións (CITIC) de la Universidade da Coruña, cuenta con una infraestructura construida para ofrecer un servicio Cloud al personal que trabaja en sus instalaciones, y así darles acceso a hardware que no está disponible en dispositivos convencionales. Actualmente, esta infraestructura tiene instalado un software de virtualización la empresa VMware, pero que no cuenta con los elementos suficientes para ser accedido por todos los usuarios. Este servicio de virtualización permite aprovisionar recursos de un conjunto de servidores, en forma de máquinas virtuales con unas especificaciones establecidas por el usuario, para realizar tareas que requieren gran capacidad de cómputo, de almacenamiento o de red.

El sistema cuenta con una plataforma de autenticación, pero los usuarios a los que está destinado el servicio no tienen acceso. Esto se debe a que no existe una herramienta que permita gestionar perfiles de usuario ya existentes, sino que, el único modo de habilitar el acceso consiste en que el administrador cree una perfil de forma manual dentro del servicio para cada usuario. También carece de una plataforma donde cada usuario solo tenga acceso a sus recursos, en la vista actual tienen visibilidad y acceso a los recursos de otros usuarios dependiendo de los permisos que se hayan asignado al perfil. Además, en el sistema actual, el proceso de aprovisionamiento de recursos mediante la creación de máquinas virtuales es complejo, por tener una interfaz poco intuitiva y difícil de manejar para un usuario que no es administrador del sistema, a parte de que el proceso debe realizarse manualmente. Esto implica que la monitorización, control y medición de los recursos que aprovisiona cada usuario sean también complejas. La falta de automatización y simplicidad en el sistema provoca que el administrador tenga que gestionar todo el entorno de forma manual, tanto los perfiles de usuarios como los recursos y su configuración, lo cual genera un gran coste y aumenta los riesgos de la infraestructura.

Como se puede observar, el servicio no cumple con las características que definen un servicio de Cloud Computing, especialmente en lo que se refiere al aprovisionamiento bajo demanda, a la elasticidad y a la monitorización y control de los recursos. Por ello, en este proyecto se desplegará un conjunto de servicios, que juntos permitan habilitar un servicio al que los usuarios puedan acceder autenticándose con sus credenciales de la UDC, aprovisionar recursos de red, almacenamiento y cómputo, y que permita monitorizar los recursos que cada usuario posee. Además, para facilitar las tareas de administración, el servicio debe automatizar las operaciones de aprovisionamiento y permitir al administrador limitar la cantidad de recursos que un usuario puede aprovisionar para evitar que estos sean infrautilizados.
Con estas mejoras se busca construir un servicio que sea útil, dinámico, sencillo de administrar, que optimice el uso de los recursos y que aumente su eficiencia, gracias a la automatización de tareas y al aprovechamiento de elementos que ya se encuentran disponibles.
% p para facilitar, en este portal el usuario puede aprovisionar recursos en forma de máquinas virtuales, modificarlos como necesite, y monitorizarlos. Esto implica que los usuarios podrían aprovisionar gran cantidad de recursos que luego podrían ser infrautilizados no pudiendo ser aprovechados por otros usuarios, por esto también es necesario implementar un sistema que permita a los administradores medir, valorar y limitar de alguna forma la cantidad de recursos aprovisionados por un mismo usuario.

% Inicialmente, el sistema cuenta con una plataforma, a la que los usuarios no tienen acceso debido a la falta de perfiles de usuario, para obtener recursos de la infraestructura física bajo demanda. Esto tiene que ser realizado por el personal encargado de recibir sus peticiones y de activar máquinas virtuales solicitadas, un proceso no automático y lento.
% Aunque actualmente si que es posible la creación de un perfil para cada usuario, esto no es viable ya que tampoco dispondrían de un espacio propio dentro del servicio si no que tendrían visibilidad y acceso, dependiendo de sus permisos, a los recursos de otros usuarios, a parte de que la interfaz es compleja y poco intuitiva, difícil de manejar para un usuario que no sea administrador del servicio.


% Estas mejoras consiguen optimizar el uso de la infraestructura y aumentar su eficiencia debido a la automatización de gran parte de las operaciones que se repiten constantemente como el aprovisionamiento, gestión de usuarios, y creación de máquinas virtuales. Así se consigue un servicio más dinámico, útil y fácil de administrar y gestionar.

%\lettrine{P}{rimer} capítulo da memoria, onde xeralmente se exporán as
%liñas mestras do traballo, os obxectivos, etc. Incluimos un par de
%exemplos de citas~\cite{ErlangBook,ErlangWebBook} e de referencias
%internas (sección \ref{sec:mostra}, páxina \pageref{sec:mostra}).


\begin{section}{Motivación}
La motivación para realizar este proyecto es crear un servicio Cloud en el CITIC para proporcionar recursos a aquellos usuarios que necesiten equipos de grandes prestaciones, y que estos los puedan conseguir de una forma sencilla y ágil, a la vez que se mejora la gestión interna del servicio para así reducir sus costes e incidencias a largo plazo. En definitiva, hacer que la infraestructura sea eficiente, útil y capaz de dar servicio a todos sus usuarios.
\end{section}
% Con el objetivo de evitar problemas en el entorno de producción, el proyecto se desarrollará en un entorno de pruebas con una cantidad de recursos reducida, para mostrar las funcionalidades y características de la solución propuesta. El proceso se realizará siguiendo la metodología incremental Scrum, en la cual primero se analizarán las diferentes alternativas disponibles, y posteriormente se irán desplegando componentes para añadir nuevas funcionalidades que completen los objetivos del proyecto.
\begin{section}{Objetivos}

El objetivo general de este proyecto es crear un servicio piloto, desplegando sobre un entorno de pruebas, y presentar las funcionalidades y características de una plataforma que permita sacar el máximo rendimiento de la infraestructura del CITIC, y de los recursos administrativos que se encuentran disponibles, tanto en el CITIC como en la UDC. Este servicio debe ser útil, ágil y accesible.
Los objetivos concretos se pueden resumir en los siguientes puntos:
\begin{itemize}
    \item Centralizar y mejorar la gestión de usuarios integrando el sistema de autenticación de la UDC y así facilitar el acceso.
    \item Desplegar un portal de acceso para los usuarios que simplifique la gestión y aprovisionamiento de sus recursos.
    \item Limitar y controlar la cantidad de recursos que un usuario puede aprovisionar y así evitar tener recursos ociosos.
    % \item Implementar un sistema de valoración del servicio que permita limitar y controlar la cantidad de recursos que un usuario puede aprovisionar, y así evitar tener recursos ociosos.
    \item Automatizar las tareas de administración y configuración de la infraestructura.
    \item Documentar las soluciones desplegadas en el sistema para facilitar la transmisión de conocimiento a largo plazo.
\end{itemize}
\end{section}

\begin{section}{Organización}
    La documentación de este proyecto se divide en cinco capítulos. El primero es \ref{chap:estado-recursos-CITIC-VCF}.\nameref{chap:estado-recursos-CITIC-VCF} y en él se describe el hardware y el software que forman la infraestructura situada en el CITIC, la situación actual de la tecnología que se quiere implementar, las alternativas encontradas en el mercado y la descripción y componentes de la solución elegida. Posteriormente, en capítulo \ref{chap:planificacionProyecto}.\nameref{chap:planificacionProyecto} se describen las tareas y los costes de la realización del proyecto en base a la solución elegida en el capítulo anterior. Una vez expuestas las tareas del proyecto, en el capítulo \ref{chap:Metodologia}.\nameref{chap:Metodologia} se describen conceptos referidos a la infraestructura y arquitectura propios de la solución que se va a implementar, los requisitos físicos y servicios que la infraestructura debe proveer antes de realizar la implementación. Finalmente, dentro del mismo capítulo en el apartado \ref{subsect:prueba-concepto}. \nameref{subsect:prueba-concepto}, se exponen la instalación y funcionalidades de los componentes de la solución propuesta, dentro de un entorno de pruebas.
    %necesarios para cumplir los objetivos del proyecto.
\end{section}

\end{chapter}