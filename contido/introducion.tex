\chapter{Introdución}
\label{chap:introducion}
Según \textit{National Institute of Standards and Technology} (NIST), el Cloud Computing es un "modelo de recursos configurables y compartidos, accesibles através de la red bajo demanda y desde cualquier lugar en cualquier momento"\cite{DefCloudComputing}. Las principales características de este modelo son:
\begin{itemize}
    \item Autoservicio bajo demanda: El usuario puede aprovisionar recursos según sus necesidades y de forma automática sin requerir ninguna interacción humana con el proveedor del servicio.
    \item Acceso a través de red: El servicio es accesible a través de mecanismos estándar.
    \item Almacén de recursos: Los recursos se sirven a distintos consumidores al mismo tiempo siguiendo un modelo de tenencia múltiple. Estos se pueden gestionar de forma dinámica y permiten conocer su ubicación física a un nivel de abstracción alto.
    \item Elasticidad: Los recursos se pueden aprovisionar o liberar de forma elástica, es decir, que se pueden escalar de forma rápida según las necesidades del usuario.
    \item Servicio medido: El sistema Cloud es capaz de aportar información sobre los recursos que el cliente tiene aprovisionados (por ejemplo, almacenamiento, ancho de banda, procesamiento, y usuarios activos).
\end{itemize}
 
 El Centro de Investigación en Tecnoloxías da Información e as Comunicacións (CITIC) de la Universidade da Coruña ofrece un servicio de Cloud Computing para el personal que trabaja en sus instalaciones, que les permite aprovisionar recursos, en forma de máquinas virtuales con unas especificaciones, de un servidor para realizar tareas que precisan gran capacidad de cómputo, de almacenamiento, o de red. Aunque el servicio ya está en uso, tiene grandes inconvenientes en la gestión y el acceso que no permiten ofrecerlo de forma abierta a todos los usuarios del centro.\\
 
Actualmente el sistema cuenta con una plataforma donde los usuarios no pueden acceder y obtener los recursos bajo demanda de forma dinámica y sencilla, esto tiene que ser realizado por una persona encargada de recibir sus peticiones y de activar máquinas virtuales para ellos, un proceso no automático. Este problema surge por la falta de un sistema con una interfaz intuitiva, ya que el portal de acceso actual es una vista de administrador del sistema con demasiadas opciones de configuración que el usuario final no debería tener disponibles. A través de dicha vista, los usuarios tampoco disponen de un directorio personal al que solo ellos tienen acceso para administrar sus recursos, con el portal actual tendrían acceso a todas las máquinas virtuales del sistema pudiendo eliminar o modificar cualquier recurso de otro usuario. Además la plataforma tampoco incorpora un sistema de medición de consumo de los recursos obtenidos por un usuario, así como tampoco hay un sistema que permita limitar la cantidad de recursos que un usuario puede aprovisionar.\\

Este servicio Cloud no cumple con las características esenciales de un sistema de Cloud Computing, como el autoservicio bajo demanda, el acceso a través de red, la gestión dinámica y elástica, y la monitorización y medición de los recursos, por lo que se hace necesario desplegar una herramienta, o herramientas, para separar el entorno de administración del entorno de usuario y crear un servicio Cloud lo más aproximado posible al modelo NIST, con la intención de obtener el máximo potencial de la infraestructura mejorando su eficiencia para que sea un servicio útil.

%\lettrine{P}{rimer} capítulo da memoria, onde xeralmente se exporán as
%liñas mestras do traballo, os obxectivos, etc. Incluimos un par de
%exemplos de citas~\cite{ErlangBook,ErlangWebBook} e de referencias
%internas (sección \ref{sec:mostra}, páxina \pageref{sec:mostra}).


\section{Motivación}


\section{Objetivos}

