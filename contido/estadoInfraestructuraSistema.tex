\chapter{Estado de los recursos}
\label{chap:estadoInfraestructuraSistema}
\lettrine{C}{on} el fin de contextualizar los recursos que se utilizarán en este trabajo, en este capítulo se expone la situación actual de toda la infraestructura en lo relacionado al software que está en funcionamiento, a los recursos físicos de los que se compone, y al estado actual de las herramientas que rodean a dichos recursos.

\section{Infraestructura}
La infraestructura física de este servicio de Cloud Computing, se encuentra localizada en el edificio del CITIC de la UDC, dentro de un rack alojado en su Centro de Proceso de Datos (CPD) \cite{citicUDC}. Está formado por 5 nodos \textit{Lenovo NeXtScale nx360 M5} y 3 nodos \textit{Dell EMC PowerEdge R740}. Ambos componentes dan flexibilidad en cuanto a la escalabilidad y ofrecen gran rendimiento de cómputo.\\

Especificaciones principales de los nodos:
\begin{itemize}
    \item Lenovo NeXtScale nx360 M5: 
        \begin{itemize}
            \item CPU: Dos Intel Xeon E5-2650
            \item Memoria: 128 GB
            \item Tarjeta  gráfica: Tesla M60
        \end{itemize}
            Más información: \href{https://lenovopress.com/tips1195-nextscale-nx360-m5-e5-2600-v3}{Especificaciones}
    \item Dell EMC PowerEdge R740:
        \begin{itemize}
            \item CPU: Dos Xeon Gold 6146
            \item Memoria: 384 GB
            \item Tarjeta gráfica: Tesla P40
        \end{itemize}
    Más información: \href{https://www.dell.com/es-es/work/shop/servidores-almacenamiento-y-redes/smart-value-poweredge-r740-server-standard/spd/poweredge-r740/per7400m}{Especificaciones}
\end{itemize}
El almacenamiento es independiente y está colocado en una ubicación distinta a la de los nodos. Esta conformado por 13 discos duros SSD de 3.84 TB de capacidad, obteniendo así una capacidad total de casi 50 TB, pero que utilizan el sistema de almacenamiento RAID 5 lo cual permite replicar los datos para conseguir mayor  integridad, tolerancia a fallos y ancho de banda, reduciendo la cantidad de almacenamiento utilizable a 34 TB. Estos discos forman un \textit{pool} de almacenamiento que se divide en cinco LUNs (\textit{Logical Storage Unit}) de 2 TB cada una, representadas en el sistema de virtualización como cinco \textit{datastores} diferentes que utilizan el sistema de archivos VMFS el cual optimiza el almacenamiento de máquinas virtuales.\\
Estos discos están colocados en varias cabinas que son accesibles por los nodos a través de dos switches para aportar redundancia. Para ello, las cabinas incorporan dos controladores SFP+ que se conectan a cada switch mediante dos puertos que aportan conectividad 10 Gb y, además, incorporan otros dos puertos con conectividad 1 Gb para la gestión de los discos. Estas conexiones utilizan los protocolos de red Ethernet y iSCSI , formando así, junto con el resto de componentes descritos, la estructura de una SAN [\ref{fig:esquemaentornoreal}].\\

La gestión del almacenamiento se realiza en la capa física, el nivel más bajo por lo que la configuración de cada LUN que utilizan las máquinas virtuales desplegadas se tiene que hacer antes de conocer los requisitos necesarios de lo que se vaya a desplegar en la capa software. Esto provoca que si se quiere desplegar una máquina virtual con más capacidad de almacenamiento o con una estructura RAID diferente haya que crear una nueva LUN que se adapte a los requisitos.
Esta gestión hace que el uso de recursos de almacenamiento no sea el óptimo ya que no permite ajustar de forma precisa y rápida cada configuración a los requisitos necesarios generando así mayor coste.

\iffalse
Los nodos acceden a los discos de almacenamiento a través de dos switches que al mismo tiempo dan conectividad entre los nodos formando una SAN. Los 10TB de capacidad están repartidos entre cinco DataStores, de 2TB cada uno. Una máquina virtual está alojada en un DataStore concreto pero puede tener ficheros alojados en varios almacenes de datos diferentes.
Las conexiones entre dispositivos son Ethernet 10 Gigabit combinado con el protocolo de transporte ISCSI
\fi

\section{Software}
\label{subsec:softwareinstalado}
Actualmente el servicio está basado en el software de la empresa VMware, uno de los principales proveedores de software de virtualización, siendo \emph{VMware vSphere} el software desplegado sobre la infraestructura. El primero se encarga de virtualizar parte de la infraestructura física y de proporcionar las herramientas necesarias para gestionarla. Sus principales componentes son los siguientes:
\begin{itemize}
    \item ESXi: Hipervisor propio de VMware, de tipo 1 o \textit{bare metal}[\ref{itm:baremetal}]. No requiere de sistema operativo para funcionar ya que funciona directamente sobre el hardware físico\cite{Esxi}. Este hipervisor está instalado en cada uno de los cinco nodos que forman la infraestructura.
    \item VMware vCenter Server: proporciona una plataforma centralizada para la gestión, operación, aprovisionamiento de recursos, y evaluación de rendimiento de las máquinas virtuales y los nodos. Incluye otros componentes\cite{componentesCenterServer}:
        \begin{itemize}
            \item Single Sign-On: es un servicio de autenticación. Permite que los componentes de vSphere se puedan comunicar sin tener que cada uno se tenga que autenticar de forma separada gracias a que crea un dominio de autenticación. También se encarga de la autenticación de usuarios, permitiendo usar un directorio de usuarios externo.\label{itm:singlesingonEX}
            \item vSphere License Server: permite gestionar e inventariar licencias para aquellos sistemas conectados a un \emph{Platform Services Controller}.
            \item VMware Certificate Authority: provee con un certificado a cada nodo. Está firmado por esta misma autoridad (VMCA).
            \item PostgreSQL: distribución de la base de datos PostgreSQL para vSphere.
            \item vSphere Web Client y vSphere Client: interfaz que permite conectarse a una instancia de vCenter Server para gestionar la infraestructura.
            \item vSphere ESXi Dump Collector: permite configurar un host ESXi para que guarde su memoria en un servidor externo en lugar de en un disco cuando hay algún fallo crítico.
            \item vSphere Syslog Collector: habilita logs de red.
            \item vSphere Auto Deploy: herramienta que permite desplegar gran cantidad de nodos físicos de forma automática.
            \item vSphere Update Manager Extension: centraliza y gestiona las actualizaciones de vSphere. 
        \end{itemize}
    
    \item Web Client: interfaz web que permite acceder a vCenter Server de forma remota.
    \item vMotion: permite la migración de máquinas virtuales de un servidor físico a otro de forma transparente.
    \item Storage vMotion: permite migrar los ficheros de una máquina virtual de un un datastore a otro, pudiendo repartirlos en diferentes datastores.
    \item High Availability (HA): provee alta disponibilidad para las máquinas virtuales. En caso de que un componente de la infraestructura falle recoloca la máquina en otra ubicación.
    \item Distributed Resource Scheduler (DRS): se encarga de balancear la carga de cómputo entre el hardware disponible. Ayuda a reducir el consumo de energía.
    \item Storage DRS: balancea la carga de almacenamiento y las operaciones I/O entre los diferentes datastores disponibles.
    \item Fault Tolerance: provee la disponibilidad continua de las máquinas virtuales habilitadas creando una copia de cada una  para usarla en caso de que la primera falle.
    \item Distributed Switch (VDS): habilita swithces virtuales que se encargan de gestionar el tráfico de los hosts ESXi.
    \item Virtual Machine File System (VMFS): vSphere proporciona un sistema de almacenamiento optimizado para tener alto rendimiento con sus hosts ESXi. Este es el sistema de archivos utilizado en los DataStores mencionados.
\end{itemize}

\begin{figure}[hp]
  \centering
  \includegraphics[width=0.75\textwidth]{imaxes/cap2recursos/contentVSphere}
  \caption{Elementos de la plataforma VMWare vSphere\cite{fotovSphere}}
  \label{fig:componentesVSphere}
\end{figure}
\begin{figure}[hp]
  \centering
  \includegraphics[width=1\textwidth]{imaxes/cap2recursos/recursosReal.png}
  \caption{Esquema de los recuros software y hardware del entorno}
  \label{fig:esquemaentornoreal}
\end{figure}

\FloatBarrier
\subsection{Estado de la tecnología}

En los últimos tiempos los servicios de IaaS (\textit{Infrastructure as a Service}) se han extendido de forma considerable con la aparición de software que permite la gestión de un sistema de Cloud Computing, como pueden ser VMware Cloud Foundation (2011), OpenStack (2010), o Apache CloudStack (2012). Estas herramientas construyen una infraestructura virtual sobre un entorno físico estandarizado que les permite administrar y automatizar la escalabilidad, el sistema de almacenamiento, la disponibilidad del servicio, la red, y la seguridad del servicio, con lo que se consigue reducir el coste y el tiempo de gestión y configuración, haciendo la infraestructura física se hace más eficiente.

A la hora de alcanzar los objetivos descritos en este proyecto se nos plantea la duda de que solución software desplegar ya que actualmente existen tres principales alternativas en el mercado, VmWare Cloud Foundation, OpenStack y Apache CloudStack. Cada una de ellas ofrece diferentes características con diferentes requisitos que se pueden adaptar mejor o peor al entorno de despliegue, pero después de comprobar esos aspectos tenemos claro que la solución elegida es \emph{VmWare Cloud Foundation}.

\subsubsection{VmWare Cloud Foundation}
Esta solución virtualiza todas las capas de la infraestructura[\ref{fig:infraCloudFound}] (red, computación y almacenamiento) combinando cuatro componentes principales, vSphere para gestionar el cómputo, vSAN para la gestión del almacenamiento, NSX para la gestión de la red, y vRealize para gestionar todas las operaciones que tienen lugar en el servicio, integrando todos los componentes para que la gestión de la infraestructura sea lo más simple posible. Este cojunto de herramientas convierten el CPD en un \emph{Software Defined Datacenter} (SDDC), un entorno donde todas las partes físicas de la infraestructura pasan a estar controladas a través de software haciendo más flexible, independiente y menos costosa la configuración de estos componentes[\ref{fig:sddcoverview}]. Las principales características de Cloud Foundation son:
\begin{itemize}
    \item \emph{Integración nativa}: todos sus componentes se integran de forma nativa entre ellos, así como con otros componentes de la empresa VmWare, minimizando las tareas de configuración y administración.
    \item \emph{Experiencia de usuario simple}: gracias a la gran cantidad de procesos automatizados.
    \item \emph{Escalabilidad modular}: el sistema y la infraestructura se puede escalar de forma sencilla.
    \item \emph{Cloud híbrida}: da la posibilidad de conectar una Cloud pública con una Cloud privada y así tratar ambas como una única Cloud.
\end{itemize}

\begin{figure}[h!]
  \centering
  \includegraphics[width=1\textwidth]{imaxes/cap2recursos/SDDCoverview.png}
  \caption{Partes virtualizadas en un SDDC.}
  \label{fig:sddcoverview}
\end{figure}
\\
\begin{figure}[h!]
  \centering
  \includegraphics[width=1\textwidth]{imaxes/cap2recursos/overviewCF.png}
  \caption{Cloud Foundation virtualiza toda la infraestructura.}
  \label{fig:infraCloudFound}
\end{figure}

Cloud Foundation permite reducir el tiempo de mantenimiento ya que todo está controlado por el software que integra todos los componentes, automatizando gran parte de las operaciones y el ciclo de vida de todos los elementos desde su creación, como puede ser el control de versiones de cada elemento, los perfiles de usuario, y las máquinas virtuales creadas, además de proporcionar una plataforma de acceso para que cada usuario pueda gestionar sus recursos.
También permite separar las cargas de trabajo mediante Dominios según el tipo de trabajo que se va a realizar, pudiendo acceder a cada uno de ellos de forma separada.




Para poder usar este software es necesaria la adquisición de licencias, estas se organizan por componente y por número de hosts sobre los que se va a instalar el producto, haciendo que el coste sea elevado, pero, a pesar de eso, se ha elegido este paquete principalmente por su integración nativa con los componentes ya instalados y porque su mantenimiento es más sencillo. \\
Si bien VMWare ofrece plugins para conectar sus componentes con otras soluciones, como es el caso de OpenStack \cite{opestackintegrated}, estos no ofrecen el rendimiento que da la integración nativa, además, en caso de recibir actualizaciones, habría que actualizar cada componente de forma individual aumentando el riesgo de incompatibilidades con el resto de elementos del sistema, mientras que Cloud Foundation gestiona todo el ciclo de vida de cada actualización para cada componente, permitiendo comprobar si existe alguna incompatibilidad con el resto de versiones antes de aplicar una actualización. En definitiva, VmWare Cloud Foundation simplifica el proceso de instalación, configuración, gestión, y mantenimiento, tanto para los usuarios como para el administrador del sistema.

\subsubsection{Componentes de VmWare Cloud Foundation \cite{componentesCloudFound}}
\label{subsubsect:cfcomponents}
\begin{itemize}
    \item \textbf{SDDC Manager}: gestiona el ciclo de vida de todos los componentes del sistema, incluyendo el proceso inicial de despliegue de Cloud Foundation, su configuración y aprovisionamiento, y las actualizaciones. Monitoriza los recursos físicos y lógicos de la infraestructura, facilita su configuración y permite añadir nuevos recursos cuando sea necesario. \iffalse Todo esto lo realiza mediante flujos de trabajo que facilitan la detección de orígenes de errores.\fi
    \item \textbf{vSphere}: ya está incluído en el servicio actual [\ref{subsec:softwareinstalado}].
    \item \textbf{vSAN}: componente clave que virtualiza el almacenamiento. Como ya se ha explicado, el almacenamiento del servicio actual está configurado con LUNs que se deben gestionar individualmente en una capa distinta a los componentes software, provocando que su configuración sea más compleja y costosa. Esto es lo que gestiona vSAN, trata toda la capacidad de almacenamiento como un único elemento, eliminando así la necesidad de tener que crear LUNs aisladas, consiguiendo abstraer la configuración de almacenamiento de la capa física en la capa de software y permitiendo establecer políticas desde cada máquina virtual para adecuarlo a las necesidades de cada una sin tener que editar la configuración real de los discos. Así el uso de almacenamiento es más eficiente, flexible y su configuración está integrada dentro del mismo servicio. \\
    En vSAN, en lugar de tratar el almacenamiento de forma independiente este pasa a estar ligado a cada host, es decir, cada uno de los nodos tiene asignados hasta cinco grupos de discos. Estos grupos de discos pueden ser híbridos, donde se combinan discos duros SSD y HDD, o All-Flash, donde todos los discos son SSD. Dentro de cada grupo hay \underline{dos tipos de discos} con distintas funciones, el disco de caché y el disco de capacidad\cite{operacionesVSAN}:
        \begin{itemize}
            \item \textbf{Caché}: Hay uno en cada grupo. Realiza la función de memoria caché y se encarga de escribir los datos persistentes en los discos de capacidad.
            \item \textbf{Capacidad}: Puede haber hasta siete discos en cada grupo. Almacena los datos persistentes del entorno.
        \end{itemize}
    En cada grupo de discos la gestión de la \underline{lectura y escritura} de datos se hace de la siguiente forma:
        \begin{itemize}
            \item \textbf{Lectura}: En el caso de la solución híbrida, si el dato que se busca no está en el disco de caché entonces se busca en los discos de capacidad y después se incorpora al disco de caché. Con la solución all-flash, los datos se leen siempre directamente de los discos de capacidad sin que estos sean escritos en el disco de caché dejando a este completamente libre para las operaciones de escritura. Gracias a esto, la estructura all-flash ofrece mayor rendimiento respecto a la híbrida.
            \item \textbf{Escritura}: Tanto en la solución híbrida con en la all-flash, el host ESXi primero escribe en el disco de caché, este le responde con una confirmación de escritura y más tarde vSAN se encarga de escribir ese dato en los discos de capacidad cuando el disco de caché está casi completo o cuando el dato lleva un tiempo sin ser utilizado.
        \end{itemize}
        \begin{figure}[h!]
            \centering
            \includegraphics[width=1\textwidth]{imaxes/cap2recursos/rendimientoVSAN.png}
            \caption{Almacenamiento All-Flash vs. Híbrido en vSAN}
            \label{fig:rendimientoVSAN}
        \end{figure}
        \FloatBarrier
    El acceso al almacenamiento desde cada nodo se hace a través de IP en una red donde están todos los nodos. \\ 
    Con esto se reducen las tareas de gestión del almacenamiento físico ya que ya no es necesario hacer ajustes en la capa física para cumplir unos requisitos en la capa software.
    \item \textbf{NSX}: otro de los componentes clave. Tiene un papel similar a vSAN, pero en este caso se encarga de virtualización de los componentes físicos de la red de la infraestructura, es decir, abstrae los componentes físicos de nuestro entorno en software, dando más libertad a la hora de establecer la topología y componentes físicos de la red. Incluye servicios como firewall, DNS, DHCP, VPN, NAT, enrutamiento, balanceo de carga, o switching, que permiten reducir la cantidad de dispositivos físicos de red.\\
    Principales \underline{componentes internos}\cite{componentesNSX}:
    \begin{itemize}
        \item \textbf{NSX Manager}: permite crear, configurar y administrar el resto de recursos de NSX. Su interfaz está integrada en vSphere.
        \item \textbf{NSX Controller}: contiene las tablas de ARP, MAC, VTEP y de enrutamiento.
        \item \textbf{NSX Virtual Switch}: gestiona los vSphere vSwitch Distibuted desplegados, proporcionando switching a los nodos ESXi.
        \item \textbf{NSX Logical Router Control}: se despliega cuando se crea un Distributed Logical Router. Se encarga de buscar adyacencias para generar tablas de enrutamiento que envía a NSX Manager y a los NSX Cotrollers para que informen a cada Distributed Logical Router en cada nodo ESXi.
        \item \textbf{NSX Edge}: proporciona servicios múltiples servicios como firewall perimetral de capa 2 y 3, SSL, NAT, DHCP, VPN, balanceo de carga y alta disponibilidad.
    \end{itemize}
    \item \textbf{vRealize Log Insight}: realiza la gestión de logs del servicio, dando visibilidad a todas las operaciones del servicio, generando análisis del sistema. Esto permite tener mayor conocimiento de los riesgos, eficiencia y uso de recursos, a parte de facilitar la búsqueda de orígenes de errores.\\
    Para su correcto funcionamiento \underline{requiere desplegar los siguientes componentes}:
    \begin{itemize}
        \item \textbf{Nodo Master}: cuando se despliega con el modo \textit{standalone} este el responsable de todas las actividades, incluyendo las consultas y gestión de logs. También participa en la gestión del ciclo de vida del cluster actualizando, eliminando y añadiendo nodos \textit{Worker}. 
        \item \textbf{Nodo Worker}: se generan para proporcionar alta disponibilidad facilitando la escalabilidad del entorno. En estos nodos se delegan las tareas de consulta y gestión de logs.
        \item \textbf{Load Balancer}: se encarga de centralizar y asegurar la entrada de logs en Log Insight simplificando la configuración para habilitar la alta disponibilidad. También balancea el tráfico de logs entrante entre los nodos existentes.
    \end{itemize}
\end{itemize}

**Decir como se puede implementar un sistema de facturación.*******\\
**Como se puede conectar los usuarios de la UDC.********












\iffalse
\subsubsection{OpenStack}
Es una plataforma constituída por tres proyectos, uno dedicado a la gestión de la red, otro a la gestión del cómputo, y otro a la gestión del almacenamiento. Para poder desplegarlo sobre los componentes de VmWare instalados en nuestra infraestructura son necesarios tres plugins específicos para el software de VmWare. OpenStack aporta una API a través de la cual es posible gestionar los recursos virtuales y el aproivisionamiento.

\subsubsection{Apache CloudStack}
\fi


