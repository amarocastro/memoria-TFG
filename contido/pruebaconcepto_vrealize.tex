\begin{subsection}{Operaciones de la Arquitectura}
    El entorno ya está configurado para funcionar como un SDDC, a partir de este punto ya no es necesario realizar ninguna modificación en la infraestructura física ya que todas las tareas que se deben realizar están dentro del alcance de los componentes de VMware Cloud Foundation. Para finalizar la construcción del SDDC y habilitar un servicio donde los usuarios puedan aprovisionar recursos bajo demanda, se instalarán sobre el entorno desplegado las aplicaciones Workspace ONE Access \footnote{VMware vRealize Identity Manager} (WSA) y VMware vRealize Automation (vRA). La primera permite al administrador conectar con el servidor de usuarios Active Directory y gestionarlos para proveer un servicio de autenticación centralizado a múltiples aplicaciones como VMware vRealize Automation. La segunda aplicación permite a los usuarios aprovisionar recursos de forma automatizada desde un catálogo de recursos. VMware vRealize Suite Licfecycle Manager (vRSLCM) es el componente que permite administrar vRA y WSA, su instalación y actualizaciones, las contraseñas de administrador y sus certificados, para ello necesita comunicarse con VMware vCenter Server. Se desplegará una instancia de cada componente en el \textit{management domain} creado anteriormente y estarán conectadas al \textit{segment}/subred \textit{Mgmt-xRegion01-VXLAN}.
    
    % \begin{subsubsection}{vRealize Suite Lifecycle Manager}
        
    % \end{subsubsection}
    \begin{subsubsection}{Workspace One Access}
        Los usuarios que necesiten acceder a vRA deben estar registrados en el directorio de Workspace One Access. Este componente centraliza el acceso de todos los productos de VMware vRealize. Cuando se despliega se debe configurar un Active Directory que en el caso del entorno está situado en la VM con Windows Server 2016. Dentro del Active Directory existen grupos de seguridad y perfiles de usuario, un perfil de usuario contiene información como nombre, apellidos, dirección e-mail, nombre de usuario y contraseña\footnote{Se pueden configurar más campos pero los que se describen son los obligatorios a la hora de crear un usuario.}, y este puede formar parte de varios grupos de seguridad. Una vez configurado, cada aplicación se conectará a WSA y se podrán asignar roles para los grupos de seguridad y usuarios estableciendo así un nivel de acceso. Además, cada usuario registrado tendrá disponible un catálogo de aplicaciones en el portal de WSA cuyo administrador establecerá que aplicaciones están habilitadas para cada usuario o grupo, eso sí, para que el usuario pueda acceder a ella previamente se debe establecer un rol para ese usuario dentro de la aplicación.

        %*****USUARIOS QUE HAY EN EL ENTORNO Y EL ACCESO A CADA APLICACIÓN*******%
        \begin{figure}[h]
            \centering
            \includegraphics[width=0.4\textwidth]{imaxes/vRealize_pruebaconcepto/usuariosDefinidos.png}
            \caption{Muestra los usuarios definidos en el Active Directory sincronizados en Workspace One Access.}
            \label{fig:users-defined-AD}
        \end{figure}
        \FloatBarrier
        En la Figura \ref{fig:users-defined-AD} se muestran los dos usuarios definidos en el Active Directory y dos usuarios que se corresponden a los perfiles de administración de WSA, no se utilizarán grupos de seguridad para reducir la complejidad pero su configuración en las aplicaciones de VMware es igual que para los perfiles de usuario. En un entorno real existen usuarios que controlan a otros usuarios y establecen su nivel de acceso, a parte de los perfiles de administrador de cada aplicación. Para el entorno se define el perfil \textit{adminuser} que será el encargado de gestionar el acceso de dos usuarios (\textit{baseuser1} y \textit{baseuser2}) que serán los que consuman a las aplicaciones desplegadas (vRSLCM y vRA). El primero tendrá acceso y permisos de edición en las aplicaciones vRSLCM y vRA, mientras que los dos usuarios base solo podrán acceder a vRA y dentro de este el usuario admin definirá que servicios están habilitados para cada uno.

        %************************************************************************%

    \end{subsubsection}

    \begin{subsubsection}{VMware vRealize Automation}
        El punto a través del cual los usuarios pueden aprovisionar sus recursos es vRealize Automation. Este producto provee el servicio cloud. 
        \begin{figure}[h]
            \centering
            \includegraphics[width=0.8\textwidth]{imaxes/vRealize_pruebaconcepto/ComponentesVRA.png}
            \caption{Componentes de VMware vRealize Automation y tareas que realiza cada rol de usuario.}
            \label{fig:vra-components}
        \end{figure}
        \FloatBarrier
        Internamente vRA se divide en varios servicios que permiten gestionar los diferentes aspectos de la cloud. Para centrarse en los objetivos de este proyecto solo se hace referencia a dos de esos servicios, el primero es Cloud Assembly el cual permite administrar la infraestructura disponible controlar el uso que se hace de esos recursos, y el segundo es Service Broker, utilizado por los usuarios para aprovisionar los recursos desde un catálogo de plantillas. La obtención de los recursos por parte del usuario se hace desplegando una serie de plantillas llamadas Blueprints diseñadas previamente, en donde se define un conjunto de VMs y recursos de red y de almacenamiento incluyendo otros aspectos como la configuración de cada uno de los recursos, como redes de la infraestructura que se utilizan, cantidad de almacenamiento, o la ubicación del despliegue en la infraestructura. Son ficheros de código con extensión \textit{.yaml} donde se indican etiquetas, aunque también se pueden diseñar con un editor gráfico. Estas plantillas están relacionadas con proyectos, una plantilla pertenece a uno o varios proyectos donde existe un coordinador de proyecto que se encarga de diseñar Blueprints y de administrar los usuarios miembros de ese proyecto. Los proyectos de vRA permiten limitar los recursos para que un conjunto de usuarios pueda desplegar los componentes definidos en las Blueprints disponibles, como la cantidad de memoria RAM, cantidad de instancias que se pueden desplegar y cantidad de almacenamiento, también aquellas redes que se pueden utilizar. Desde el punto de vista de vRA, la infraestructura se divide en Cloud Zones, las cuales son conjuntos de recursos situados en distintos proveedores Cloud que pueden ser públicos como AWS o Azure, o privados que solo pueden ser clusters vSphere. En el caso del entorno desplegado solo se tendrá una única Cloud Zone de tipo vSphere. En cada Cloud Zone se define como se deben distribuir los recursos aprovisionados sobre la infraestructura. 
        Finalmente será el administrador de la infraestructura el que se encargue de proveer los recursos, administrar los proyectos disponibles, gestionar los coordinadores de cada proyecto y controlar y limitar el uso de los recursos.
        % Finalmente, vRA permite configurar tarjetas donde se puede definir el coste del aprovisionamiento de CPU, almacenamiento y memoria RAM, además del coste de uso de otros elementos como sistemas operativos, el uso de una determinada red o el uso de una determinada Cloud Zone. Estas tarjetas se asignan por proyecto para determinar el coste que tendrá el consumo de recursos por mes.


    \end{subsubsection}

    
\end{subsection}