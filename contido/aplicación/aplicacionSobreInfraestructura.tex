\begin{chapter}{Aplicación de la solución}

En este capítulo se detalla como sería la implementación de la solución desplegada en el entorno de pruebas sobre la infraestructura disponible en el CITIC.

\begin{section}{Arquitectura del entorno}

Cuando se despliegan los componentes de VMware Cloud Foundation se crea el primer \textit{workload domain} que es el Management Domain. Este WD se despliega inicialmente sobre cuatro de los hosts pero como el entorno cuenta con ocho hosts, todavía quedarían otros cuatro hosts sin aprovisionar por lo tanto es necesario pensar que arquitectura se va a implementar. Existen dos posibilidades, el modelo estándar y el modelo consolidado, y para ambas el entorno cuenta con los recursos suficientes.
El modelo estándar está pensado para entornos grandes con un mínimo de 7 hosts. Para el caso de la infraestructura del CITIC primero se crearía el Management Domain con cuatro hosts y después se añadiría un Virtual Infrastructure Domain con los cuatro hosts restantes, así, de esta forma, las operaciones del SDDC estarían separadas ya que el Management Domain se dedicaría a dar soporte a sus propios componentes y al resto de Workload Domains, permitiendo gestionar las actualizaciones, monitorizar y resolver conflictos, gestionar la seguridad y administrar las operaciones, mientras el VI Domain contendría los recursos que los usuarios aprovisionan desde el componente VMware vRealize Automation.
Con el modelo consolidado solo existiría un único \textit{workload domain} que contendría ocho hosts de la infraestructura. Este sería un Management Domain donde se comparten los mismos recursos para los componentes que gestionan el SDDC como en el modelo estándar y para las operaciones de aprovisionamiento de recursos, aunque la capacidad de uso de recursos de cada componente se podría limitar colocándolos en \textit{resorce pools}.
Viendo las diferencias entre ambos, el primer modelo proporciona mayor aislamiento y seguridad de los recursos ya que los dedicados a la administración están separados de los dedicados a las operaciones de los usuarios de la plataforma Cloud. Al mismo tiempo, esto limita la cantidad de recursos disponibles y reduce la disponibilidad del servicio ya que cada WD está limitado por el número de hosts que lo forman. En cambio, con el modelo consolidado se reduce la capacidad de aislamiento de cada flujo de trabajo pero todos los hosts comparten las operaciones de administración y de los usuarios aumentando así la disponibilidad del servicio y la cantidad de recursos. El modelo recomendado por VMware en este caso es el modelo estándar porque tiene mejores medidas de seguridad y los recursos se administran de una forma más sencilla. 
En el momento de implementar esta solución es necesario decidir cuantos hosts y cuales de los disponibles se van a asignar a cada \textit{workload domain} ya que no todos aportan la misma cantidad de recursos. El el objetivo es balancear los recursos entre los WD para proporcionar suficientes recursos y que sus operaciones se ejecuten correctamente.
Todos los hosts y componentes de almacenamiento de la infraestructura están situados en una misma ubicación, dentro del CITIC, por lo tanto el entorno de producción estaría formado por una única \textit{region} con solo una una AZ en su interior la cual agruparía todos los componentes.
\end{section}

\begin{section}{Cumplimiento de requisitos}
    En esta sección se detallará si los recursos físicos ya disponibles en la infraestructura son suficientes para implementar VMware Cloud Foundation o si por el contrario sería necesario aumentarlos.
    La infraestructura estará formada por dos \textit{workload domains}, un Management Domain y un VI Domain, el primero requiere al menos cuatro hosts mientras que el segundo debe contener al menos tres\footnote{Tres es la cantidad mínima de hosts para implementar VMware vSAN que será el tipo de almacenamiento que se utilice en el VI Domain.}, por lo tanto hay suficientes hosts, ocho, para implementarlo. Aparte, los requisitos físicos descritos para los hosts del Management Domain se aplican también a los hosts del VI Domain. Teniendo en cuenta esto, un host con los requisitos mínimos debería contar con dos interfaces de red, un grupo de discos con dos discos duros con 4 TB para capacidad y 200 GB de caché y 128 GB de memoria RAM. Los hosts disponibles cumplen con todos los puntos pero se disponen de suficientes discos duros, hay trece discos disponibles pero se necesitan al menos dieciséis\footnote{Dos discos cada uno de los ocho hosts.}. El ancho de banda de las conexiones de red existentes también cumple con el mínimo, 10 Gbit.

    % Los recursos disponibles deberían ser suficientes para soportar la instalación de todos los componentes de VMware Cloud Foundation. En total esta cuenta con ocho hosts, 3072 GB de memoria RAM, 53,2 GHz de CPU y 34 TB de almacenamiento disponible, lo cual supera ampliamente los requisitos mínimos especificados anteriormente, por ello no es necesario aumentar la capacidad del entorno para implementar la solución. En cuanto a los requisitos de red, los hosts también cuentan con suficientes interfaces físicas y las conexiones de red cumplen con los requisitos de ancho de banda.
    \end{section}

\begin{section}{Diseño y configuración del VI Domain}
    Una vez desplegado el MD en la infraestructura cuya configuración sería similar a la descrita en el capítulo anterior, se debería desplegar un VI domain con los cuatro hosts restantes en el entorno de producción. En esta sección se describirán aquellos aspectos más relevantes el diseño y configuración de ese segundo \textit{workload domain} una vez se decida instalar el servicio Cloud en las máquinas del CITIC.

    Los componentes de este WD, como el Management Domain, también deben tener acceso al servidor DNS, al servidor DHCP y al servidor NTP para su correcto funcionamiento y para que todos puedan estar sincronizados entre si. Además, también se debe proveer acceso al directorio de usuarios de la UDC para establecer roles y habilitar usuarios que se puedan conectar a los componentes que gestionan este VI Domain, y al router o switches de la capa física para que los componentes puedan acceder a la red externa.

    \begin{subsection}{Diseño de los componentes}
        Si bien el diseño de los componentes que VMware Cloud Foundation despliega para controlar un VI Domain es muy similar al que se ha descrito en el capítulo anterior es necesario resumirlo y destacar aquellas diferencias.

        La instancia del componente VMware vCenter Server que controla el VI Domain estaría alojada dentro del Management Domain y sería el encargado de controlar los cuatro hosts pertenecientes al WD. Además, sería necesario activar la opción \textit{Enhanced Link Mode}, al igual que en la instancia de VMware vCenter que controla el Management Domain, para que ambas instancias compartan sus respectivos PSCs y así formen un único dominio de autenticación SSO con el que poder gestionar ambas desde una misma interfaz de vSphere Web Client.

        El almacenamiento para un VI Domain puede ser de distintos tipos, SAN, NAS o VMware vSAN, incluso se pueden combinar, pero para para la realización de este proyecto solo se tiene en cuenta la configuración de almacenamiento con VMware vSAN ya que reduce la complejidad de administración de la infraestructura. Este VI Domain debería tener su propio VMware vSAN \textit{datastore} diferente del utilizado por el Management Domain. El \textit{datastore} debería tener el tamaño suficiente para soportar las operaciones de aprovisionamiento y despliegue de recursos que realicen los usuarios. Además, la configuración de FTT debería ser igual a 1 para soportar la caída de alguno de los hosts. Finalmente, también requiere de una subred dedicada para proporcionar el servicio de almacenamiento mediante el protocolo IP como se describe para el Management Domain.

        Dentro del VI Domain se crearía un cluster de VMware vSphere donde se incluyen los cuatro hosts que forman el WD. Su configuración de vSphere High Availability y de vSphere DRS sería la misma para el Management Domain, es decir, para el primer servicio se configura la reserva del 25\% de CPU y el 30\% de memoria RAM del WD y se activa la propiedad \textit{VM and Application Monitoring}, y para el segundo servicio se configura la opción \textit{Fully Automated}. La red de este cluster estaría formada por un único vSphere Distributed Switch que contendría un \textit{Management Port Group}, un \textit{vMotion Port Group}, un \textit{vSAN Port Group}, dos \textit{Edge Uplink Port Groups} (todos ellos de tipo \textit{Distributed Port Group}) y dos \textit{Uplink Port Groups} que dan salida al tráfico hacia la red física. Las funciones y configuración de las propiedades para cada \textit{port group} son las mismas que las descritas en el Management Domain para los mismos \textit{port groups} a excepción de la propiedad \textit{Port Binding} que se establecería como \textit{Static} para todos porque la opción \textit{Emphemeral} ya no es necesaria al estar vCenter Server en otro WD, eliminándose así la dependencia con su estado. Además, las subredes que se deberían configurar tendrían que ser diferentes ya que sus servicios serían dedicados a este WD, excepto para el \textit{Management Port Group} que se debería conectar a la misma subred que el \textit{Management Port Group} del Management Domain y así poder comunicarse y controlar el VI Domain. La configuración de los servicios \textit{Network I/O} y \textit{Health Check} de este vDS es la misma que en el Management Domain. El tamaño del MTU también debe ser de 9000 Bytes.

        En cuanto al componente VMware NSX-T, se desplegarían tres instancias de VMware NSX-T Manager Applicance en el Management Domain y dos instancias de Vmware NSX-T Edge dentro del propio VI Domain. Su configuración sería la misma que la descrita para el Management Domain, una TZ de tipo VLAN con su \textit{Uplink Policy} correspondiente y con dos \textit{segments} conectados a los dos \textit{Edge Uplink Port Groups} del vDS que serán los que utilicen las instancias de VMware NSX-T Edge para dirigir el tráfico hacia el dispositivo de red físico, y otra TZ de tipo Overlay pero esta vez con un \textit{segment} para comunicar los componentes de VMware NSX-T y adicionalmente tantos \textit{segments} se quieran crear para que sean usados por los usuarios del servicio Cloud. Estos \textit{segments} formarían parte de una topología con dos routers virtuales, uno de \textit{Tier-1} y otro de \textit{Tier-0} donde los \textit{segments} de tipo VLAN se conectarían a \textit{Tier-0} y los creados para entregar el servicio Cloud, al \textit{Tier-1} donde además se podrían proporcionar otros servicios como DNS o DHCP. En la red física, este WD también requiere la configuración de BGP y ECMP para el correcto funcionamiento de los componentes de VMware NSX-T y el aprovechamiento de todas las funcionalidades que ofrece una red definida por software.

    \end{subsection}

\end{section}


\end{chapter}
