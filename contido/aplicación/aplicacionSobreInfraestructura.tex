\begin{chapter}{Aplicación de la solución}

En este capítulo se detalla como sería la implementación de la solución desplegada en el entorno de pruebas sobre la infraestructura disponible en el CITIC.

\begin{section}{Cumplimiento de requisitos}
Los recursos disponibles deberían ser suficientes para soportar la instalación de todos los componentes de VMware Cloud Foundation. En total esta cuenta con ocho hosts, 3072 GB de memoria RAM, 53,2 GHz de CPU y 34 TB de almacenamiento disponible, lo cual supera ampliamente los requisitos mínimos especificados anteriormente, por ello no es necesario aumentar la capacidad del entorno para implementar la solución. En cuanto a los requisitos de red, los hosts también cuentan con suficientes interfaces físicas y las conexiones de red cumplen con los requisitos de ancho de banda.
\end{section}

\begin{section}{Arquitectura del entorno}

Cuando se despliegan los componentes de VMware Cloud Foundation se crea el primer \textit{workload domain} que es el Management Domain. Este WD se despliega inicialmente sobre cuatro de los hosts pero como el entorno cuenta con ocho hosts, todavía quedarían otros cuatro hosts sin aprovisionar por lo tanto es necesario pensar que arquitectura se va a implementar. Existen dos posibilidades, el modelo estándar y el modelo consolidado, y para ambas el entorno cuenta con los recursos suficientes.
El modelo estándar está pensado para entornos grandes con un mínimo de 7 hosts. Para el caso de la infraestructura del CITIC primero se crearía el Management Domain con cuatro hosts y después se añadiría un Virtual Infrastructure Domain con los cuatro hosts restantes, así, de esta forma, las operaciones del SDDC estarían separadas ya que el Management Domain se dedicaría a dar soporte a sus componentes y al resto de Workload Domains, permitiendo gestionar de forma dedicada sus actualizaciones, monitorizar y resolver los conflictos, gestionar su seguridad y administrar sus operaciones, mientras el VI Domain estaría dedicado a realizar las tareas de aprovisionamiento que los usuarios realizarían desde el componente VMware vRealize Automation.
Con el modelo consolidado solo existiría un único \textit{workload domain} que contendría ocho hosts de la infraestructura. Este sería un Management Domain donde se comparten los mismos recursos para los componentes que gestionan el SDDC como en el modelo estándar y para las operaciones de aprovisionamiento de recursos, aunque la capacidad de uso de recursos de cada componente se podría limitar colocándolos en \textit{resorce pools}.
Viendo las diferencias entre ambos, el primer modelo proporciona mayor aislamiento y seguridad de los recursos ya que los usados para administración están separados de los usados por los usuarios de la plataforma Cloud. Al mismo tiempo, esto limita la cantidad de recursos disponibles y reduce la disponibilidad del servicio. En cambio, con el modelo consolidado se reduce la capacidad de aislamiento de cada flujo de trabajo pero todos los hosts comparten las operaciones de administración y de los usuarios aumentando así la disponibilidad del servicio y la cantidad de recursos. A pesar de esto, el modelo recomendado por VMware cuando se tienen más de siete hosts es el modelo estándar por lo que es el que se debería utilizar en la implementación. En el momento de implementar esta solución es necesario decidir cuantos hosts y cuales de los disponibles se van a asignar a cada \textit{workload domain} ya que no todos aportan la misma cantidad de recursos, el objetivo es balancear los recursos y así cada WD tenga suficientes recursos para llevar a cabo sus operaciones.
Además, toda la infraestructura está situada en una misma localización, por lo tanto el entorno de producción estará formado por una única \textit{region} con solo una una AZ.

\begin{subsection}{Diseño y configuración del VI Domain}
    Una vez desplegado el MD en la infraestructura cuya configuración sería similar a la descrita en el capítulo anterior, se debería desplegar un VI domain con los cuatro hosts restantes en el entorno de producción. En esta sección se describirán aquellos aspectos más relevantes el diseño y configuración de ese segundo \textit{workload domain} una vez se decida instalar el servicio Cloud en las máquinas del CITIC.

    Los componentes de este WD, como el Management Domain, también deben tener acceso al servidor DNS, al servidor DHCP y al servidor NTP para su correcto funcionamiento y para que todos puedan estar sincronizados entre si. Además, también se debe proveer acceso al directorio de usuarios de la UDC para establecer roles y habilitar usuarios que se puedan conectar a los componentes que gestionan este VI Domain, y al router o switches de la capa física para que los componentes puedan acceder a la red externa.

    \begin{subsubsection}{Diseño de los componentes}
        Si bien el diseño de los componentes que VMware Cloud Foundation despliega para controlar un VI Domain es muy similar al que se ha descrito en el capítulo anterior es necesario resumirlo y destacar aquellas diferencias.

        La instancia del componente VMware vCenter Server que controla el VI Domain estaría alojada dentro del Management Domain y sería el encargado de controlar los cuatro hosts pertenecientes al WD. Además, sería necesario activar la opción \textit{Enhanced Link Mode}, al igual que en la instancia de VMware vCenter que controla el Management Domain, para que ambas instancias compartan sus respectivos PSCs y así formen un único dominio de autenticación SSO con el que poder gestionar ambas desde una misma interfaz de vSphere Web Client.

        Dentro del VI Domain se crearía un cluster de VMware vSphere donde se incluyen los cuatro hosts que forman el WD. Su configuración de vSphere High Availability y de vSphere DRS sería la misma para el Management Domain, es decir, para el primer servicio se configura la reserva del 25\% de CPU y el 30\% de memoria RAM del WD y se activa la propiedad \textit{VM and Application Monitoring}, y para el segundo servicio se configura la opción \textit{Fully Automated}. La red de este cluster estaría formada por un único vSphere Distributed Switch que contendría un \textit{Management Port Group}, un \textit{vMotion Port Group}, un \textit{vSAN Port Group}, dos \textit{Edge Uplink Port Groups} (todos ellos de tipo \textit{Distributed Port Group}) y dos \textit{Uplink Port Groups} que dan salida al tráfico hacia la red física. Las funciones y configuración de las propiedades para cada \textit{port group} son las mismas que las descritas en el Management Domain para los mismos \textit{port groups} a excepción de la propiedad \textit{Port Binding} que se establecería como \textit{Static} para todos porque la opción \textit{Emphemeral} ya no es necesaria al estar vCenter Server en otro WD eliminándose así la dependencia con su estado. Además, las subredes que se deberían configurar tendrían que ser diferentes excepto para el \textit{Management Port Group} que tiene que estar conectado con el \textit{Management Port Group} del Management Domain para que este pueda comunicarse y controlar el VI Domain. La configuración de los servicios \textit{Network I/O} y \textit{Health Check} de este vDS es la misma que en el Management Domain. El tamaño del MTU también debe ser de 9000 Bytes.
    \end{subsubsection}

\end{subsection}


\end{section}

\end{chapter}
