%%%%%%%%%%%%%%%%%%%%%%%%%%%%%%%%%%%%%%%%%%%%%%%%%%%%%%%%%%%%%%%%%%%%%%%%%%%%%%%%

\begin{abstract}\thispagestyle{empty}
    El Cloud Computing es un modelo que ofrece un conjunto de recursos (por ejemplo, redes, almacenamiento, aplicaciones, servicios y potencia de cálculo) que pueden ser aprovisionados bajo demanda de una forma rápida y sencilla, y minimizando el coste del servicio y el esfuerzo en cuanto a gestión de los recursos.\\\\
    Desde hace varios meses* Centro de Investigación en Tecnoloxías da Información e as Comunicacións (CITIC) de la Universidade da Coruña, ofrece un servicio de Cloud Computing al personal que allí trabaja. Este servicio consiste en que los usuarios pueden aprovisionar un conjunto de recursos, que se traducen en máquinas virtuales, del tamaño que necesiten para realizar tareas que no serían posibles en dispositivos convencionales.\\\\
    Actualmente el servicio es limitado y no está abierto a todos los usuarios del CITIC debido a la falta de una plataforma que centralice los accesos y que controle todo el ciclo de vida de los recursos aprovisionados. Los usuarios deben hacer una petición a una persona encargada con las máquinas virtuales que necesitan y este debe activarlas. \\\\
    El objetivo principal de este proyecto es desplegar una plataforma sobre el servicio actual Cloud del CITIC, donde los usuarios, de forma automática, puedan obtener recursos según sus necesidades, facilitar la gestión de las credenciales de usuarios enlazándolas con las credenciales de la UDC, y controlar y limitar, en cierta medida, el uso que hacen de los recursos y así evitar tener recursos ociosos, en definitiva, el objetivo es mejorar la eficiencia de esta Cloud y obtener todo su potencial.
    \item 
  \vspace*{25pt}
\begin{multicols}{2}
\begin{description}
\item [\palabraschaveprincipal:] \mbox{} \\[-20pt]
  \blindlist{itemize}[7] % substitúe este comando por un itemize
                         % que relacione as palabras chave
                         % que mellor identifiquen o teu TFG
                         % no idioma principal da memoria (tipicamente: galego)
\end{description}
\begin{description}
\item [\palabraschavesecundaria:] \mbox{} \\[-20pt]
  \blindlist{itemize}[7] % substitúe este comando por un itemize
                         % que relacione as palabras chave
                         % que mellor identifiquen o teu TFG
                         % no idioma secundario da memoria (tipicamente: inglés)
\end{description}
\end{multicols}

\end{abstract}

%%%%%%%%%%%%%%%%%%%%%%%%%%%%%%%%%%%%%%%%%%%%%%%%%%%%%%%%%%%%%%%%%%%%%%%%%%%%%%%%
