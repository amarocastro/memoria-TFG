%%%%%%%%%%%%%%%%%%%%%%%%%%%%%%%%%%%%%%%%%%%%%%%%%%%%%%%%%%%%%%%%%%%%%%%%%%%%%%%%

\begin{abstract}\thispagestyle{empty}
  El Cloud Computing es un modelo que permite acceder a un conjunto de recursos, como por ejemplo redes, almacenamiento y cómputo, los cuales pueden ser aprovisionados bajo demanda de forma automatizada y dinámica, reduciendo el coste del servicio para el usuario y el esfuerzo en cuanto a la administración de los recursos. El Centro de Investigación en Tecnoloxías da Información e as Comunicacións (CITIC) de la Universidade da Coruña cuenta con una infraestructura ideada para ofrecer un servicio de Cloud Computing a la comunidad universitaria. Este servicio consiste en que los usuarios pueden aprovisionar un conjunto de recursos del tamaño que requieran para realizar tareas que no serían posibles en dispositivos convencionales. Actualmente ese servicio está activo pero de forma limitada y no abierta a todos los usuarios del CITIC debido a que no existe una plataforma que permita gestionar los perfiles de usuario y su autenticación, ni un portal de acceso para aprovisionar recursos y gestionarlos de forma automatizada. En la actualidad, las tareas de aprovisionamiento y gestión de usuarios se realizan bajo petición previa, al administrador del sistema, que las ejecuta de forma manual, lo cual produce gran coste en tiempo y recursos y aumenta los riesgos del servicio.\\

  El objetivo principal de este proyecto consiste en desplegar un servicio Cloud en el CITIC, usando como base la infraestructura y herramientas ya existentes. El servicio debe proveer un sistema de autenticación para que cada usuario pueda acceder con sus credenciales de la UDC a una plataforma, la cual le permita aprovisionar y gestionar recursos de forma automatizada. Además, también debe automatizar la gestión de todos los componentes de la infraestructura, incluyendo la gestión de perfiles de usuarios, el control de la cantidad de recursos disponibles para los usuarios y el despliegue de aplicaciones, con el fin de liberar a los administradores de las tareas más redundantes y repetitivas, para así obtener el máximo rendimiento de la infraestructura disponible en el CITIC.
  %  El objetivo principal de este proyecto consiste en desplegar un servicio Cloud en el CITIC usando como base las herramientas que ya se encuentran sobre la infraestructura, donde cada usuario pueda tener su un espacio donde gestionar la obtención de recursos y que permita reducir las tareas de administración al integrar el sistema de autenticación de la UDC y al automatizar los procesos de aprovisionamiento, permitiendo establecer unos límites y controlar los recursos que utiliza cada usuario. Todo esto con el fin de mejorar la eficiencia de la infraestructura.\\
   Con el objetivo de evitar problemas en el entorno de producción, el proyecto se desarrollará en un entorno de pruebas para mostrar las funcionalidades y características de la solución implementada pero que tendrán menor rendimiento que el entorno real por contar con recursos reducidos. El proceso se realizará siguiendo la metodología incremental Scrum en la cual primero se analizarán las diferentes alternativas disponibles y posteriormente se irán desplegando componentes para añadir nuevas funcionalidades que completen los objetivos del proyecto.
    % El Cloud Computing es un modelo que permite acceder a un conjunto de recursos, como por ejemplo, redes, almacenamiento, aplicaciones, servicios y potencia de cálculo, que pueden ser aprovisionados bajo demanda de una forma rápida sencilla, reduciendo el coste del servicio para el usuario y el esfuerzo en cuanto a gestión de los recursos.\\
  %  El Centro de Investigación en Tecnoloxías da Información e as Comunicacións (CITIC) de la Universidade da Coruña cuenta con una infraestructura ideada para ofrecer un servicio de Cloud Computing a la comunidad universitaria. Este servicio consiste en que los usuarios pueden aprovisionar un conjunto de recursos, que se traducen en máquinas virtuales, del tamaño que necesiten para realizar tareas que no serían posibles en dispositivos convencionales.\\
    % Actualmente el servicio está activo pero de forma limitada y no abierta a todos los usuarios del CITIC debido que no existe una plataforma que habilite el acceso a cada usuario a sus recursos, que gestione la autenticación de los usuarios, y que automatice y simplifique los procesos de aprovisionamiento de recursos. Hasta ahora, las tareas de aprovisionamiento y gestión de usuarios se realizan bajo petición previa al administrador del sistema y se ejecutan de forma manual lo cual consume más tiempo, recursos, y aumenta los riesgos sobre el servicio.\\
    % El objetivo principal de este proyecto es desplegar un servicio Cloud en el CITIC usando como base las herramientas que ya se encuentran sobre la infraestructura, donde cada usuario pueda tener su propio espacio donde poder gestionar sus recursos según sus necesidades y de forma simple y dinámica, que simplifique la gestión de usuarios integrando el sistema de autenticación de la UDC en el servicio, y establecer un sistema de valoración de los recursos para poder controlar y limitar la cantidad de recursos que un usuario puede aprovisionar para mejorar el uso y funcionamiento del servicio. En definitiva, el objetivo es desplegar un entorno Cloud que haga la infraestructura física más eficiente y para entregar todo su potencial.
  \vspace*{20pt}
\begin{multicols}{2}
\begin{description}
\item [\palabraschaveprincipal:] \mbox{} \\[-20pt]
  \blindlist{itemize}[7] % substitúe este comando por un itemize
                         % que relacione as palabras chave
                         % que mellor identifiquen o teu TFG
                         % no idioma principal da memoria (tipicamente: galego)
\end{description}
\begin{description}
\item [\palabraschavesecundaria:] \mbox{} \\[-20pt]
  \blindlist{itemize}[7] % substitúe este comando por un itemize
                         % que relacione as palabras chave
                         % que mellor identifiquen o teu TFG
                         % no idioma secundario da memoria (tipicamente: inglés)
\end{description}
\end{multicols}

\end{abstract}

%%%%%%%%%%%%%%%%%%%%%%%%%%%%%%%%%%%%%%%%%%%%%%%%%%%%%%%%%%%%%%%%%%%%%%%%%%%%%%%%
