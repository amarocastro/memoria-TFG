%%%%%%%%%%%%%%%%%%%%%%%%%%%%%%%%%%%%%%%%%%%%%%%%%%%%%%%%%%%%%%%%%%%%%%%%%%%%%%%%

\begin{abstract}\thispagestyle{empty}
    El Cloud Computing es un modelo que permite acceder a un conjunto de recursos, como por ejemplo, redes, almacenamiento, aplicaciones, servicios y potencia de cálculo, que pueden ser aprovisionados bajo demanda de una forma rápida sencilla, reduciendo el coste del servicio para el usuario y el esfuerzo en cuanto a gestión de los recursos.\\
   El Centro de Investigación en Tecnoloxías da Información e as Comunicacións (CITIC) de la Universidade da Coruña cuenta con una infraestructura ideada para ofrecer un servicio de Cloud Computing a la comunidad universitaria. Este servicio consiste en que los usuarios pueden aprovisionar un conjunto de recursos, que se traducen en máquinas virtuales, del tamaño que necesiten para realizar tareas que no serían posibles en dispositivos convencionales.\\
    Actualmente el servicio está activo pero de forma limitada y no abierta a todos los usuarios del CITIC debido que no existe una plataforma que habilite el acceso a cada usuario a sus recursos, que gestione la autenticación de los usuarios, y que automatice y simplifique los procesos de aprovisionamiento de recursos. Hasta ahora, las tareas de aprovisionamiento y gestión de usuarios se realizan bajo petición previa al administrador del sistema y se ejecutan de forma manual lo cual consume más tiempo, recursos, y aumenta los riesgos sobre el servicio.\\
    El objetivo principal de este proyecto es desplegar un servicio Cloud en el CITIC usando como base las herramientas que ya se encuentran sobre la infraestructura, donde cada usuario pueda tener su propio espacio donde poder gestionar sus recursos según sus necesidades y de forma simple y dinámica, que simplifique la gestión de usuarios integrando el sistema de autenticación de la UDC en el servicio, y establecer un sistema de valoración de los recursos para poder controlar y limitar la cantidad de recursos que un usuario puede aprovisionar para mejorar el uso y funcionamiento del servicio. En definitiva, el objetivo es desplegar un entorno Cloud que haga la infraestructura física más eficiente y que entregue todo su potencial.
  \vspace*{25pt}
%\begin{multicols}{1}
\begin{description}
\item [\palabraschaveprincipal:] \mbox{} \\[-20pt]
\begin{itemize}
    \item Cloud Computing
    \item CITIC
    \item Virtualización
    \item SDDC
    \item Aprovisionamiento
\end{itemize}
\end{description}
%\end{multicols}

\end{abstract}

%%%%%%%%%%%%%%%%%%%%%%%%%%%%%%%%%%%%%%%%%%%%%%%%%%%%%%%%%%%%%%%%%%%%%%%%%%%%%%%%
