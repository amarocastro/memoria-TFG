\chapter*{\nomeglosariotermos}
\addcontentsline{toc}{chapter}{\nomeglosariotermos}
\label{chap:glosario-termos}

%%%%%%%%%%%%%%%%%%%%%%%%%%%%%%%%%%%%%%%%%%%%%%%%%%%%%%%%%%%%%%%%%%%%%%%%%%%%%%%%
% Obxectivo: Lista de termos empregados no documento,                          %
%            xunto cos seus respectivos significados.                          %
%%%%%%%%%%%%%%%%%%%%%%%%%%%%%%%%%%%%%%%%%%%%%%%%%%%%%%%%%%%%%%%%%%%%%%%%%%%%%%%%

\begin{description}
   % \item [Active Directory]: servicio de directorio de usuarios propio de Microsoft.
   % \label{itm:active-directory}
  %  \item [Appliance]: archivo que contiene una máquina virtual con un sistema operativo con el propósito de entregar una única aplicación preconfigurada.
  %  \label{itm:appliance}
  %  \item [BGP]\cite{bgp}: protocolo de enrutamiento que se utiliza para el intercambio de rutas entre Autonomous Systems (AS) de forma dinámica y así evitar configurarlas manualmente.
  %  \label{itm:bgp}
  %  \item [BUM]: se refiere al tráfico de red Broadcast, Unknown unicast y Multicast. El primero es tráfico que se transmite a todos los dispositivos disponibles en la red, Unknown Unicast es tráfico enviado a un único destinatario para el que no se conoce su dirección MAC dentro de una misma VLAN y Multicast es tráfico que se envía a los dispositivos que pertenecen a un grupo dentro de una red.
  %  \label{itm:bum}
  %  \item [Cluster]\cite{cluster-vmware}: agrupación de recursos de múltiples hosts que se gestionan como una única colección.
  %  \label{itm:cluster}
   \item [Conector SFP+]: interfaz modular que permite conectar cables de fibra óptica a un dispositivo.
   \label{itm:sfp} 
  \item [CPD]: luegar donde se sitúan un conjunto recursos con gran capacidad de cómputo necesarios para procesar información, normalmente en grandes cantidades.
  \label{itm:cpd}
  \item [Datastore]: dentro de VMware vSphere, un datastore es un contenedor lógico que abstrae los componentes físicos de almacenamiento y provee un modelo uniforme para almacenar máquinas virtuales, plantillas o imágenes ISO.
  \label{itm:datastore}
  \item [Hipervisor baremetal]: software instalado sobre el hardware de un servidor que permite instalar aplicaciones que funcionan sobre entornos virtuales directamente sobre el hardware.
  \label{itm:baremetal}
  \item [Host]: servidor físico en el que se ejecuta un hipervisor.
  \label{itm:host}
  \item [IaaS]: servicio Cloud en el que se provee capacidad de aprovisionamiento de recursos de cómputo, almacenamiento y red, sobre los cuales se puede desplegar software \cite{computing}.
  \label{itm:iaas}
  \item [iSCSI]: estándar que implementa el protocolo de transporte SCSI para transmitir datos entre dispositivos.
  \label{iscsi}
  % \item [Jumbo Frame]\cite{jumbo-frames}: paquetes de red cuyo MTU es mayor que el valor definido en el estándar Ethernet, 1500.
  % \label{itm:jumboframe}
  \item [LUN]: identifica una colección de dispositivos de almacenamiento que se presentan como un único volumen.
  \label{itm:lun}
  \item [Multi-tenant]: "Multi-tenancy is a relatively new software architecture principle in the realm of the Software as a Service (SaaS)" \cite{multi-tenant}.

%  \item [Multi-tenant]\cite{multi-tenant}: principio de la arquitectura de software donde la misma instancia de una aplicación se comparte con diferentes usuarios pero que se comporta como si cada usuario obtuviera su propia instancia.
%  \label{itm:tenenciamultiple}
%  \item [NIC]: componente físico que conecta un dispositivo a una red y permite compartir sus recursos.
%  \label{itm:nic}
 \item [Pool de almacenamiento]: agrupación de volúmenes de almacenamiento que se administran de forma conjunta.
 \label{itm:pool-almacenamiento}
%  \item [Port Group]: puertos que se añaden en el componente vSphere Distributed Switch y que agrupan las conexiones de múltiples máquinas virtuales sobre las cuales se pueden establecer una configuración determinada.
%  \item [QoS]: medida de rendimiento que se asigna a un servicio en la red. Utiliza el campo Differentiated Services Code Point (DSCP) en la cabecera de capa 3, y el campo Class of Service (CoS) en la cabecera de capa 2 para indicar la prioridad del tráfico.
 \label{itm:qos}
 \item [Rack]: armario metálico destinado a alojar servidores físicos.
 \label{itm:rack}
 \item [RAID 5]: conjunto de discos duros que funciona como una única unidad de almacenamiento para aumentar el rendimiento y la eficiencia. RAID 5 necesita como mínimo tres discos duros, y distribuye de paridad en todos los discos para poder recuperar datos corruptos.
 \label{itm:raid5}
%  \item [Red Overlay]\cite{overlay}: abstracción de una red sobre una red física implementada por un conjunto de nodos situados en diferentes localizaciones y conectados entre si.
%  \label{itm:overlay}
 \item [SAN]: red dedicada a proveer acceso a los dispositivos de almacenamiento.
 \label{itm:san}
%  \item [SDDC]\cite{sddc}: Software-Defined Datacenter es un modelo de  arquitectura de infraestructura para virtualizar los recursos de cómputo, almacenamiento y red.
%  \label{itm:sddc}
%  \item [UDP]: protocolo de red de la capa de transporte que permite enviar paquetes sin establecer previamente una conexión.
%  \label{itm:udp}
\item [Virtual Machine]: máquina que se ejecuta en un entorno virtualizado con hardware virtual dentro de un hipervisor.
\label{itm:vm}
 \item [VLAN]: método para aislar múltiples dominios de broadcast sobre una misma red física.
 \label{itm:vlan}
%  \item [VLAN trunk]: enlace que permite la circulación del tráfico de diferentes redes VLAN.
%  \label{itm:trunk}

%   \item [CoS]: \textit{Class of Service} es un campo de la cabecera de un paquete Ethernet que determina su prioridad cuando se utiliza etiquetado VLAN. Es un protocolo QoS de capa 2.
%  \label{itm:cos}
%   \item [DSCP]: \textit{Differentiated Services Code Point} es campo de la cabecera IP que forma parte del protocolo QoS de capa 3 \textit{DiffServ}, y que sirve para clasificar el tráfico según servicios.
%  \label{itm:dscp}



%   \item [DHCP helper address]: elemento que permite retransmitir el tráfico broadcast de un servidor DHCP por múltiples redes.
%  \label{itm:dhcpHelper}
\end{description}