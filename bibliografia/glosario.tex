\chapter*{\nomeglosariotermos}
\addcontentsline{toc}{chapter}{\nomeglosariotermos}
\label{chap:glosario-termos}

%%%%%%%%%%%%%%%%%%%%%%%%%%%%%%%%%%%%%%%%%%%%%%%%%%%%%%%%%%%%%%%%%%%%%%%%%%%%%%%%
% Obxectivo: Lista de termos empregados no documento,                          %
%            xunto cos seus respectivos significados.                          %
%%%%%%%%%%%%%%%%%%%%%%%%%%%%%%%%%%%%%%%%%%%%%%%%%%%%%%%%%%%%%%%%%%%%%%%%%%%%%%%%

\begin{description}
 \item [Tenencia múltiple]: principio de la arquitectura de software donde una aplicación se sirve a varios clientes desde una misma instancia.
 \label{itm:tenenciamultiple}
 \item [SDDC]: \textit{Software Defined DataCenter} es un modelo de infraestructura donde se virtualiza la abstracción, gestión y automatización de todos los recursos y servicios de un centro de datos.
 \item [Hipervisor baremetal]: software instalado sobre el hardware de un servidor que permite instalar aplicaciones que funcionan sobre entornos virtuales directamente sobre el hardware.
  \label{itm:baremetal}
 \item [Máquina virtual]: software que emula un conjunto de recursos físicos para ejecutar otro software de forma aislada.
  \label{itm:vm}
 \item [Datastore]: contenedores que VMware vSphere utiliza para el almacenamiento archivos en un único lugar o a través de una red. Suelen utilizarse para almacenar ficheros de máquinas virtuales y puedent tener el formato VMFS, NFS o NAS.
 \label{itm:datastore}
 \item [Modelo multi-tenant]: es un modelo de desarrollo donde una misma aplicación se entrega a distintos usuarios sin hacer un desarrollo específico para cada uno de ellos.
 \label{itm:multitenant}
 \item [RAID 5]: Es un conjunto de discos duros que funciona como una única unidad de almacenamiento para aumentar el rendimiento y la eficiencia. RAID 5 necesita como mínimo tres discos duros, y distribuye la información de paridad en todos los discos (esta información permite recuperar datos corruptos a partir del resto de información no perdida).
 \label{itm:raid5}
 \item [Almacén de datos]
 \label{itm:almacendatos}
 \item [LUN]: \textit{Logical Unit Number} es un identificador que agrupa un agrupa un conjunto o subconjunto de almacenamiento físico o virtual. Puede asignarse a un disco completo o solo a una parte.
 \label{itm:lun}
 \item [Controlador SFP+]: módulo transceptor óptico que se utiliza en las telecomunicaciones y aplicaciones de transmisión de datos. Soportan Sonet, canal de Fibra y Gigabit Ethernet.
 \label{itm:sfp}
 \item [SAN]: \textit{Storage Area Network} es una red dedicada al almacenamiento, de alta velocidad con canal de Fibra o iSCSI, con equipos de conexión dedicados (p.e. switches) y con dispositivos de almacenamiento (discos duros). 
 \label{itm:san}
 \item [VMFS]: sistema de archivos de alto rendimiento nativo de VMware vSphere. Se utiliza para implementar los almacenes de datos y está optimizado para el almacenamiento de máquinas virtuales.
 \label{itm:vmfs}
 \item [Platform Services Controller (PSC)]: componente de la infraestructura que agrupa los servicios de infraestructura de un entorno vSphere. Estos servicios son la concesión de licencias, administración de certificados y la autenticación con vCenter Single Sign-On.
 \label{itm:psc}
 \item [Cluster]: Conjunto de dos o más Hosts para aprovisionar recursos.
 \label{itm:cluster}
 \item [Servicio LBT]: servicio que se encarga de balancear el tráfico que entra en cada interfaz de un switch.
 \item [vCPU]
 \label{itm:vcpu}
 \item [Jumbo Frame]: son los paquetes que se transmiten por una red y cuyo MTU es mayor a 1500.
 \label{itm:jumboframe}
 \item [VTEP]: \textit{VXLAN Tunnel End Point} es un componente del protocolo VXLAN cuya función es encapsular y desencapsular las tramas correspondientes a una VXLAN. Este componente se encuentra al principio y al final del camino que sigue una trama.
 \label{itm:vtep}
 \item [NIC]: \textit{Network Interface Controller} es un componente físico que conecta el host con una red.
 \label{itm:nic}
 \item [Log]: registro que muestra información sobre un evento que afecta a un proceso en particular.
 \label{itm:log}
  \item [CPD]: Centro de Procesamiento de Datos es un espacio donde se encuentran los recursos necesarios para procesar información.
 \label{itm:cpd}
 \item [Pool de recursos]: representa una partición de recursos disponibles que no se crean ni se eliminan bajo demanda.
 \label{itm:poolResources}
  \item [CoS]: \textit{Class of Service} es un campo de la cabecera de un paquete Ethernet que determina su prioridad cuando se utiliza etiquetado VLAN. Es un protocolo QoS de capa 2.
 \label{itm:cos}
  \item [DSCP]: \textit{Differentiated Services Code Point} es campo de la cabecera IP que forma parte del protocolo QoS de capa 3 \textit{DiffServ}, y que sirve para clasificar el tráfico según servicios.
 \label{itm:dscp}
  \item [VLAN trunk]: enlace que permite comunicar distintas VLANs.
 \label{itm:trunk}
  \item [ARP]: protocolo responsable de obtener la dirección MAC que corresponde a una dirección IP.
 \label{itm:arp}
  \item [Rack]: armario metálico destinado a alojar servidores físicos.
 \label{itm:rack}
  \item [DHCP helper address]: elemento que permite retransmitir el tráfico broadcast de un servidor DHCP por múltiples redes.
 \label{itm:dhcpHelper}
\end{description}
